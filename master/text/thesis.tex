\documentclass[11pt,twoside,utf8,final]{fithesis} %twoside
\usepackage{amsmath,amssymb,amsthm, amsfonts}
\usepackage{booktabs}
\usepackage{graphics}
\usepackage{caption} 
\usepackage[hang]{subfigure}
\usepackage[pdftex]{graphicx}
\usepackage[czech]{babel}
\usepackage[utf8]{inputenc}
\usepackage[usenames,dvipsnames]{color}
\usepackage{url}
\usepackage{tikz}
\usepackage{amsthm}
\usepackage{float}
\usepackage{algorithm}
\usepackage{algpseudocode}
\usepackage{hyperref}
\usepackage{listings}
\usepackage{comment}
\usetikzlibrary{calc,trees,positioning,arrows,chains,shapes.geometric,%
    decorations.pathreplacing,decorations.pathmorphing,shapes,%
    matrix,shapes.symbols, decorations.markings}

\floatname{algorithm}{Pseudokód}
\renewcommand{\algorithmicrequire}{\textbf{Vstup:}}
\renewcommand{\algorithmicensure}{\textbf{Výstup:}}
\renewcommand{\lstlistingname}{Zdrojový kód}

\theoremstyle{plain}
\newtheorem{thm}{Theorem}[chapter] % reset theorem numbering for each chapter

\theoremstyle{definition}
\newtheorem{defn}[thm]{Definice} % definition numbers are dependent on theorem numbers
\newtheorem{exmp}[thm]{Example} % same for example numbers

\definecolor{src-bck}{RGB}{240, 240, 240}
\hypersetup{
    bookmarks=true,         % show bookmarks bar?
    unicode=true,           % non-Latin characters in Acrobat bookmarks
    pdftoolbar=true,        % show Acrobat toolbar?
    pdfmenubar=true,        % show Acrobat menu?
    pdffitwindow=false,     % window fit to page when opened
    pdfstartview={FitH},    % fits the width of the page to the window
    pdftitle={Bakalářská práce},    
                            % title
    pdfauthor={Jan Papoušek},% author
    pdfsubject={Paralelizace metod pro analýzu dynamických systémů pomocí grafické karty},
                            % subject of the document    
    pdfkeywords={keyword1} {key2} {key3}, % list of keywords
    pdfnewwindow=true,      % links in new window
    colorlinks=false,       % false: boxed links; true: colored links
    linkcolor=blue,         % color of internal links
    citecolor=green,        % color of links to bibliography
    filecolor=magenta,      % color of file links
    urlcolor=cyan,          % color of external links
    linkbordercolor={1 1 0}
}

\thesislang{cs}
\thesistitle{Analýza robustnosti spojitých dynamických systémů v~distribuovaném prostředí}
\thesissubtitle{Diplomová práce}
\thesisstudent{Jan Papoušek}
\thesiswoman{false}
\thesisfaculty{fi}
\thesisyear{Jaro 2013}
\thesisadvisor{RNDr. David Šafránek, Ph.D.}

\setcounter{tocdepth}{2}


\definecolor{javared}{rgb}{0.6,0,0} % for strings
\definecolor{javagreen}{rgb}{0.25,0.5,0.35} % comments
\definecolor{javapurple}{rgb}{0.5,0,0.35} % keywords
\definecolor{javadocblue}{rgb}{0.25,0.35,0.75} % javadoc
\definecolor{sourcebackground}{RGB}{250, 250, 250} % javadoc

\lstdefinestyle{Java}{
	language=Java,
	basicstyle=\ttfamily,
	keywordstyle=\color{javapurple}\bfseries,
	stringstyle=\color{javared},
	commentstyle=\color{javagreen},
	morecomment=[s][\color{javadocblue}]{/**}{*/},
	numbers=left,
	frame=leftline,
	backgroundcolor=\color{sourcebackground},
	numberstyle=\tiny\color{black},
	stepnumber=1,
	tabsize=4,
	showspaces=false,
	showstringspaces=false,
	captionpos=b
}

\lstdefinestyle{Bash}{
	language=Bash,
	basicstyle=\ttfamily,
	numbers=left,
	frame=leftline,
	numberstyle=\tiny\color{black},
	stepnumber=1,
	tabsize=4,
	showspaces=false,
	showstringspaces=false,
	captionpos=b
}

\begin{document}

\FrontMatter
\ThesisTitlePage

\begin{ThesisDeclaration}
\DeclarationText

\medskip
\begin{flushright}
\begin{minipage}{0.36\textwidth}
V Brně dne \today\\
Jan Papoušek
\end{minipage}
\end{flushright}

\AdvisorName
\end{ThesisDeclaration}

\begin{ThesisThanks}
Děkuji vedoucímu své práce Davidu Šafránkovi za odborné vedení
a~pos\-kyt\-nu\-tí cenných rad, Svenu Dražanovi za inspirativní konzultace, Tomáši Vejpustkovi za spolupráci 
při implementaci nástroje Parasim a celé Laboratoři sys\-té\-mo\-vé biologie za poskytnutí
technického zázemí.

\vspace{0.5cm}
\noindent
Děkuji svým rodičum Janu a Evě Papouškovým za podporu, kterou mi poskytli během studia a vůbec v celém
mém dosavadním životě. Děkuji své nastávající manželce Tereze Doležalové za naše dlouhé
rozpravy formující mé životní směřování, jenž vy\-ústi\-lo nejen v tuto práci.

\vspace{0.5cm}
\noindent
Bez Vás by tato práce nevznikla.

\end{ThesisThanks}

\begin{ThesisAbstract}
Práce prezentuje algoritmus pro analýzu dynamických systémů zadaných pomocí soustavy
obyčejných diferenciálních rovnic vzhledem k~cho\-vání charakterizovanému vlastností
temporální logiky signálů. Výsled\-kem a\-na\-lý\-zy je představa o tom,
jakým způsobem ovlivňují změny modelu jeho chování. Popisovaná metoda se zakládá
na již existujícím algoritmu a výpoč\-tu lokální robustnosti.

Algoritmus byl implementován v programovacím jazyce Java do podoby nástroje Parasim tak,
aby analýzu bylo možno spustit v prostředí se sdílenou nebo distribuovanou pamětí.
Výpočetní model a architektura nástroje umožňují komponenty odpovídající jednotlivým
částem algoritmu snadno nahrazovat, případně použít tuto platformu pro jiný typ
vý\-poč\-tu.

Vlastnosti algoritmu a škálovatelnost implementace pro sdílenou i distribuovanou paměť
byly ověřeny spuštěním analýzy nad vybranými modely.


\begin{comment}
The work presents an algorithm for the analysis of dynamic systems specified by the ordinary differential equations.
The system behavior is characterized by the formula of signal temporal logic. The result of the analysis is the idea of
how the changes affect the model's behavior. The described method is based on an existing algorithm and calculation of local robustnes.

The algorithm has been implemented in the Java programming language into the tool called Parasim
so that the analysis could be executed in an environment with shared or distributed memory.
Computational model and tool architecture allow components corresponding to parts of the algorithm
easy to replace, or use this platform to another type of calculation.

Properties of the algorithm implementation and scalability for both shared and
distributed memory has been verified by running analyzes on selected models.
\end{comment}

\end{ThesisAbstract}

\begin{ThesisKeyWords}
dynamický systém, soustava diferenciálních rovnic, STL, monitoring, analýza vlastnosti, robustnost
\end{ThesisKeyWords}

\MainMatter
\tableofcontents

%------------------------------- CHAPTERS --------------------------------------

\chapter{Úvod}\label{chapter:introduction}

Okolo nás se vyskytují různé systémy rozličných vlastností, k jejichž
pochopení se stále častěji používá modelování. Růst významu modelů se stupňuje
s rozšiřujícím se používáním výpočetní techniky, která je schopná s těmito modely
efektivně pracovat. Modelování se již nepoužívá pouze v~tradičních oblastech, jako je například
předpověď počasí, ale s postupem času proniká i~do~oblastí, jakými je například
modelování procesů v~živých organismech, kte\-ré\-mu se věnuje systémová biologie~\cite{westerhoff2005}.

S rostoucím významem modelů se zdá být stále důležitější dokázat formulovat
vlastnosti, které od modelů očekáváme, za použití nějaké for\-mál\-ní logiky a následně
je automatizovaným způsobem nad těmito modely o\-vě\-řo\-vat. Není však vhodné spokojit se s ověřením
vlastnosti pro jedno konkrétní nastavení hodnot proměnných a parametrů modelu. Analýza
by měla jít dál a nahlížet na model obecněji. Model je možné vychýlit z jeho ideálního
nastavení nebo naopak jeho ideální nastavení vzhledem k dané vlastnosti najít. V kontextu
modelů živých organismů takový druh analýzy otevírá možnost studovat systém v méně příznivých
nebo dokonce pro daný organismus životu nebezpečných podmínkách.

Cílém této diplomové práce je naimplementovat algoritmus právě pro takový druh analýzy,
který bude schopen z různých ohodnocení parametrů a proměnných efektivně najít ta ohodnocení,
která splňují požadovanou vlastnost. Implementovaný algoritmus vychází z algoritmu
prezentova\-né\-ho v~di\-plo\-mo\-vé práci Svena Dražana \cite{drazan2011} a principu lokální robustnosti použitého v~ná\-stro\-ji Breach~\cite{donze2010breach},
který se tímto té\-ma\-tem rovněž zabývá. Narozdíl od algoritmu pou\-ži\-té\-ho v~ná\-stro\-ji Breach však
implementovaný algoritmus není takovu mírou pro\-vá\-zán s numerickou simulací a používá
jednodušší způ\-sob pokrytí množiny ohodnocení parametrů a proměnných.

Výsledkem práce je volně dostupná aplikace Parasim, která vznikla ve spolupráci s Tomášem Vejpustkem,
jenž zajistil uživatelské rozhraní. Tvor\-ba této aplikace proběhla v rámci Laboratoře
systémové biologie\footnote{\url{http://sybila.fi.muni.cz/}} pod dohledem Svena Dražana.
Parasim umožňuje provést výpočet analýzy nejen na jednom počítači, ale i v distribuovaném
prostředí, a tím analýzu urychlit.

Následující text se nejprve v kapitole \ref{chapter:preliminaries} zabývá základními pojmy, jako je definice modelu a logiky
pro vyjádření vlastností. Následuje kapitola \ref{chapter:algorithm} popisující řešený problém, původní
algoritmus, ze kterého práce vychází, pojem lokální robustnosti a provedené úpravy v algoritmu.
Kapitola \ref{chapter:implementation} se věnuje implementaci modulárního systému a výpočetního modelu.
Kapitola \ref{chapter:evaluation} popisuje evaluaci, která zahrnuje provedení několika analýz
a~vy\-hod\-no\-ce\-ní naměřených dat. Na závěr, v kapitole \ref{chapter:conclusion}, je práce
shrnuta a nastíněny směry, kterými se lze ubírat dále v budoucnu.

% systemova biologie jak
% analyza modelu
% proc je to tema v ramci systemove biologie zajimave
% naimplementovat konkretni algoritmus
% proc tento algoritmus
% jak budu postupovat pri implementaci
% jaka struktura prace
% jak zjistim, ze se to povedlo

\chapter{Pojmy a východiska}\label{chapter:preliminaries}

Tato kapitola se věnuje základním pojmům nutným k~pochopení
algoritmu pro analýzu dynamických systémů představeného dále v kapitole \ref{chapter:algorithm}
a~kontextu, v jakém byl navržen. Jedná se zejména o popis reprezentace zkou\-ma\-ných
modelů a požadovaných vlastností.

%\section{Systémová biologie}

\section{Modelování}

Před popisem samotného modelování pomocí obyčejných diferenciálních rovnic je
nutné říci, co se od vytvářených modelů zpravidla očekává. Model má zpro\-středko\-vat
zjednodušený pohled na zkoumaný systém a umožnit tím systém snáze pochopit a
případně předpovídat některá jeho chování. Fakt, že model je jen zjednodušením, znamená,
že se vždy od reality liší a~reflektuje jen některé aspekty chování zkoumaného
systému~\cite[str. 48]{pelanek2012}.

% Priklad modelu mapa, modelujeme problem ne system

Jednoduchým příkladem modelu je mapa. Zřejmě nemůžeme od mapy očekávat, aby obsahovala
všechny aspekty zahrnuté ve skutečném světě. Svým způsobem je mapa již od počátku
\uv{špatně}, nabízí pouze jistou abstrakci systému a může i zkreslovat náš pohled.
Přesto nelze popírat její užitečnost. Z příkladu mapy je také zřejmé, že se
nemodeluje systém, ale problém. Existuje celá řada druhů map od turistických, automap
až po mapy podloží a každá z nich má svůj specifický účel~\cite[str. 47 -- 58]{pelanek2012}.

Pro tuto práci jsou důležité modely, které lze simulovat. Model definuje pravidla, podle
kterých se systém chová a simulace umožňuje podívat se na chování systému v čase, ať už
diskrétním či spojitém. Pro účely simulace je sa\-mozřej\-mě potřeba znát stav systému v~počátečním
čase, od kterého se jeho další chování odvíjí.

\subsection{Modelování pomocí obyčejných diferenciálních rovnic}

Hojně užívaným způsobem modelování, kde vystupuje spojitý čas, jsou obyčejné diferenciální rovnice.
Stav systému se vyjádří pomocí stavových pro\-měn\-ných $\mathbf{X} = (\mathit{x_1}, \mathit{x_2}, \dots \mathit{x_n})$.
Každé stavové proměnné přísluší diferenciální rovni\-ce prvního řádu, ve které vystupuje
Lipschitzovsky spojitá \cite[str. 149 -- 163]{eriksson2004} funkce
$f_i: [t, \infty) \times \mathbb{R}^n \rightarrow \mathbb{R}$, která popisuje,
jak se hodnota stavové proměnné mění v čase. Tvar takové rovnice je vidět ve sché\-ma\-tu~\ref{eq:ode:one}.
Celý systém označujeme zkráceně funkcí $f$ danou předpisem \ref{eq:ode:all}.

\begin{align}
\label{eq:ode:one}
\frac{d\mathit{x_i}}{dt} &= f_i(t, \mathbf{X})									\\
\label{eq:ode:all}
f(t, \mathbf{X})		 &= \big(f_0(t, \mathbf{X}), \ldots, f_n(t, \mathbf{X})\big)
\end{align}

Pro účely simulace není nutné znát úplné řešení této soustavy rovnic, ale postačuje pouze znalost
vývoje systému od počatečního času $t_0$, kterému odpovídá počáteční stav $\mathbf{X}(t_0)$.
V praxi se setkáváme s tím, že neznáme ani tento přesný vývoj, nýbrž pouze jeho aproximaci,
kterou poskytují metody pro řešení problému výchozích podmínek~\cite{iserles1996}.

Tyto metody hledají aproximaci v diskrétním čase a chyba, s níž se vy\-po\-čí\-ta\-ná
aproximace liší od skutečného řešení, je shora ohraničená uživatelem danou hodnotou.
Fakt, že si může uživatel takto zvolit toleranci chy\-by, je jednou z nejdůležitějších
vlastností těchto metod. Nastavení chyby může samozřejmě v případě nízké tolerance
a některých systémů vyústit ve vý\-kon\-nost\-ní problémy.

Jestliže je dána numerická metoda $\mathcal{M}_\epsilon(f, \mathbf{X}(t_0), \Delta t)$,
kde $\epsilon$ je relativní chyba a $\Delta t$ požadovaný časový krok,
pak zpravidla pracujeme se sekvencí bodů danou předpisem \ref{eq:ode:aprox},
ve kterém $\tau$ představuje délku numerické simulace.

\begin{align}
\begin{array}{ll}
\label{eq:ode:aprox}
\mathcal{M}^\tau_\epsilon(f, \mathbf{X}(t_0), \Delta t) = (\mathbf{X}_0, \mathbf{X}_1, \ldots \mathbf{X}_k), \\
\textrm{~~~~~~~kde~~~}\mathbf{X}_i \sim \mathbf{X}(t_i), t_i = t_0 + i \cdot \Delta t,	\\
\textrm{~~~~~~~~~~~~~~~~~}k \cdot \Delta t \leq \tau \wedge (k+1) \cdot \Delta t > \tau
\end{array}
\end{align}

% Priklad modelu ve spojitem case

\subsection{Příklad modelu}\label{section:lotkav}

Známým příkladem modelu využívajícího soustavy diferenciálních rovnic je model popisující vztah predátora
a kořisti \cite{lotka1925} definovaný rovnicemi \ref{eq:lotkav}. Obsahuje stavové pro\-měn\-né
pro množství kořisti ($x$) a počet predátorů ($y$), dále parametry pro přirozený
přírůstek kořisti ($\alpha$), \uv{žravost} predátorů ($\beta$), přirozený úbytek
predátorů ($\gamma$) a schopnost reprodukce predátorů ($\delta$).

\begin{align}\label{eq:lotkav}
\begin{array}{ll}
\frac{dx}{dt} &= x\cdot(\alpha - \beta \cdot y)			\\
\frac{dy}{dt} &= -y \cdot (\gamma - \delta \cdot x)
\end{array}
\end{align}

Model je samozřejmě zjednodušením reality. V systému se nachází pou\-ze dva druhy
zvířat, bíložravá kořist a masožravý predátor. U kořisti se před\-po\-klá\-dá bezproblémový
přístup k potravě, a proto přirozeně příbývá. Naproti tomu reprodukce predátora je
závislá na přísunu masité potravy, tzn. množství kořisti v systému.

I na tomto jednoduchém modelu je však možné pozorovat netriviální chování, o čemž se lze přesvědčit
při pohledu na grafy  \ref{fig:lotkav:timeserie} a \ref{fig:lotkav:oscil}. Systém má
tendenci oscilovat. Periodicky dochází k nárustu populace predátora, to vyústí v pokles
populace kořisti, následně u predátorů v důsledku nedostatku potravy převáží úmrtnost nad rozsahem reprodukce,
množství kořisti opět naroste a tento cyklus se znovu opakuje.

\begin{figure}[h!]
\begin{center}
\subfigure[vývoj systému v čase]{
	\includegraphics[width=.48\textwidth]{../images/generated/lotkav-timeserie.pdf}\label{fig:lotkav:timeserie}
}
\subfigure[stavy, kterými systém prochází]{
	\includegraphics[width=.48\textwidth]{../images/generated/lotkav-oscil.pdf}\label{fig:lotkav:oscil}
}
\end{center}
\caption{Model systému obsahující predátora a kořist s ohodnocením parametrů $\alpha = 2$,  $\beta = 0.5$, $\gamma = 0.2$ a $\delta = 0.6$.}
\end{figure}

Síla modelů a simulace se ukazuje být v tom, že tento druh informace získáme, aniž bychom prováděli
experiment se skutečným systémem se  skutečnými liš\-ka\-mi a zajíci. Samozřejmě se na závěr neobejdeme
bez validace výsledků simulace oproti realitě, ale tomu může předcházet velké množství
experementů na počítači, které by v reálných podmínkách stály mnoho prostředků nebo by ani 
nebyly uskutečnitelné. Za zmínku stojí na\-pří\-klad model popisující šíření nákazy
populací~\cite{kermack1927}, což je téma, u něhož si opravdu lze jen těžko představit
experimentování na reálném systému.

\subsection{Modelování chemických reakcí}

Modelování pomocí obyčejných diferenciálních rovnic je natolik obecný a~ú\-čin\-ný nástroj,
že jej lze použít pro popis jevů z mnoha oblastí. Jednou z typických oblastí, kde
se rovnice používají, jsou chemické reakce. Pro systém elementárních chemických reakcí
lze za jistých před\-po\-kla\-dů automatizovaně zkonstruovat systém diferenciálních rovnic,
kde stavovými proměnnými jsou koncentrace jednotlivých látek. Elementárními che\-mic\-ký\-mi
reakcemi se rozumí ty chemické reakce, u nichž dochází k přímé přeměně reaktantů
v produkty bez reakčních mezikroků nebo v jejichž pří\-pa\-dě je možné tyto mezikroky zanedbat \cite{horn1972}.
Schémata \ref{eq:chemeq} ukazují, jak tento převod konkrétně vypadá pro jednotlivé
elementární chemické reakce a specifickou kinetickou konstantu $k$. Koncentraci látky
$\textrm{X}$ značíme $[\textrm{X}]$.

\begin{align}\label{eq:chemeq}
\begin{array}{ll}
\xrightarrow{k} A			&\leadsto \frac{d[\textrm{A}]}{dt} = k						\\
A \xrightarrow{k} 			&\leadsto \frac{d[\textrm{A}]}{dt} = - k \cdot [\textrm{A}]	\\
A \xrightarrow{k} B			&\leadsto \frac{d[\textrm{A}]}{dt} = - k \cdot [\textrm{A}], \frac{d[\textrm{B}]}{dt} = k \cdot [\textrm{A}] \\
AB \xrightarrow{k} A + B	&\leadsto \frac{d[\textrm{AB}]}{dt} = - k \cdot [\textrm{AB}], \frac{d[\textrm{A}]}{dt} = \frac{d[\textrm{B}]}{dt} = k \cdot [\textrm{A}] \\
A + B \xrightarrow{k} AB	&\leadsto \frac{d[\textrm{A}]}{dt} = \frac{d[\textrm{B}]}{dt} = - k \cdot [\textrm{A}] \cdot [\textrm{B}], \frac{d[\textrm{AB}]}{dt} = k \cdot [\textrm{A}] \cdot [\textrm{B}] \\
A + A \xrightarrow{k} AA	&\leadsto \frac{d[\textrm{A}]}{dt} = - 2k \cdot [\textrm{A}]^2, \frac{d[\textrm{AB}]}{dt} = k \cdot [\textrm{A}]^2 \\
\end{array}
\end{align}

Pro obecné reakce abstrahující určitou kaskádu elementárních reak\-cí univerzální
převod neexistuje. Zde je již nutné přihlédnout k typu reakce. V~dů\-sled\-ku toho, že se
zanedbají mezireakce s meziprodukty, může vý\-sled\-ný systém diferenciálních rovnic obsahovat
méně proměnných a je tak snazší jej simulovat.

Je vhodné poznamenat, že nástroj, který popisuje tato práce v~kapitole \ref{chapter:implementation},
načítá model zapsaný v jazyce SBML \cite{hucka2003,drager2011}. Tento jazyk představuje standard
v oboru systémové biologie pro sdílení modelů a lze z něj automatizovaným způsobem získat model
ve formě systému diferenciálních rovnic.

\section{Vlastnosti modelovaných systémů}

Aby bylo možné modely automatizovaně analyzovat, je nutné vyjadřovat se o jejich
vlastnostech exatně. Jazyk, který je k tomuto účely nutné použít, musí být schopen 
popsat chování systému v čase. V kontextu obyčejných di\-fe\-ren\-ci\-ál\-ních rovnic je
chováním trajektorie, v případě simulace sekvence bodů s~časovým razítkem. Například u modelu
uvedeného v~sekci \ref{section:lotkav} je vhodné popsat oscilaci populace kořisti
nebo predátora. To lze provést tak, že budeme požadovat, aby množství jedinců
daného druhu opakovaně pře\-sáh\-lo maximální a minimální mez. Jak konkrétně
tento po\-ža\-da\-vek zformulovat, je ukázáno v následující části této kapitoly.

K vyjádření vlastností nad lineárními běhy systémů se nejčastěji používá lieární
temporální logika~\cite{strejcek2007} (\textit{linear temporal logic}, LTL), případně logiky z~ní odvozené.
LTL u\-mož\-ňu\-je se zjednodušeně vyjadřovat o sta\-vech
sys\-té\-mu v čase formulacemi typu \uv{v budoucnu nastane \ldots}, \uv{vždy platí \ldots} apod.
Tato logika se definuje nad nekonečnými běhy, a proto je v~této práci použita temporální logika
signálů~\cite{maler2004} (\textit{signal temporal logic}, STL) založená na temporální logice metrických intervalů~\cite{alur1996} (\textit{metric interval temporal logic}, MITL),
která se od LTL liší zejména přidáním ča\-so\-vých intervalů u~temporálních operátorů. STL tedy
u\-mož\-ňu\-je formulovat výroky, které jsou čás\-teč\-ně zá\-vis\-lé na čase, jako
je například \uv{za hodinu až dvě nastane \ldots} či \uv{za třicet minut bude celé
čtyři hodiny platit \ldots}.

\subsection{Signál}
Pro účely vyjadřování se o chování modelu pomocí STL se zavádí pojem signálu~\cite{maler2004}.
Zvolme si časovou doménu $\mathbb{T} = \mathbb{R}_{\geq 0}$ a signál konečné délky
nad libovolnou doménou $\mathbb{D}$ jako parciální zobrazení $s: \mathbb{T} \rightarrow \mathbb{D}$.
Definičním oborem tohoto zobrazení nechť je interval $l = [0, r)$, kde $r \in \mathbb{Q}_{\geq0}$ nazáváme
délkou signálu a značíme ji $|s| = r$. Pro všechna $t \geq r$ položíme $s[t] = \bot$. Takto definované
signály lze sdružovat pomocí párovací funkce $||$.

\begin{align}\label{eq:signals:pairing}
\begin{array}{ll}
s_1: \mathbb{T} \rightarrow \mathbb{D}_1, s_2: \mathbb{T} \rightarrow \mathbb{D}_2	\\
s_1 || s_2: \mathbb{T} \rightarrow \mathbb{D}_1 \times\mathbb{D}_2		\\
s_1 || s_2= s_{12}\textrm{, kde }\forall t\in\mathbb{T}. s_{12}[t] = (s_1[t] \times s_2[t])
\end{array}
\end{align}

Pro případ, že se délky skládaných signálů liší, definujeme výslednou délku složeného
signálu jako $|s_{12}| = min(|s_1|, |s_2|)$. Standardní cestou lze na těchto sdružených
signálech definovat projekční funkce.

\begin{align}\label{eq:signals:pairing}
\begin{array}{ll}
s_1 = \pi_1(s_{12})		& s_2 = \pi_2(s_{12})
\end{array}
\end{align}

Pro libovolnou funkci $f: \mathbb{D} \rightarrow \mathbb{D}'$ a signál $s: \mathbb{T} \rightarrow \mathbb{D}$ je zápisem $f(s_1)$ myšleno
skládání funkcí $f \circ s_1 : \mathbb{T} \rightarrow \mathbb{D}'$, kde $f(\bot) = \bot$.

Je dobré si povšimnout, že definice signálu je konzistentní s tím, jak chápeme chování modelu,
tedy jako trajektorii v $\mathbb{R}^n$, kde $n$ je počet sta\-vo\-vých proménných. Zároveň
je však třeba si uvědomit, že výstupem numerické simulace není spojitá trajektorie, nýbrž pouze
sekvence bodů s~ča\-so\-vým razítkem. Z praktických důvodů je dále v této kapitole tato sekvence
chá\-pá\-na jako po částech konstantní funkce.

\begin{figure}[h!]
\begin{center}
\subfigure[nasimulovaná data]{
	\includegraphics[width=.48\textwidth]{../images/generated/piecewise-constant-a.pdf}
}
\subfigure[po částech konstantní funkce]{
	\includegraphics[width=.48\textwidth]{../images/generated/piecewise-constant-b.pdf}
}
\end{center}
\caption{Příklad převodu sekvence bodů na po částech konstantní funkci.}
\end{figure}

\subsection{Temporální logika signálů}\label{section:stl}
Důležitým aspektem zde použité logiky jsou uzavřené časové intervaly $I = [a, b]$, kde $a, b \in \mathbb{Q}_{\geq0}$,
jimiž jsou omezeny všechny temporální operátory. Konečnost intervalů je jedním z rozdílů
oproti klasické temporální logice metrických intervalů. Toto omezení je plně v souladu s tím,
že modely porovnáváme s reálnými systémy, které pozorujeme konečný čas. Tento předpoklad značně
zjedno\-du\-šuje další analýzu.

Nechť $U = \{\mu_1, \mu_2, \mu_3, \ldots, \mu_k\}$ je množina efektivně vyčíslitelných funk\-cí
$\mu_i: \mathbb{R}^n \rightarrow \{T, F\}$, které zpravidla odpovídají predikátům tvaru
$f(\mathbf{X})~\geq~k$ nebo $f(\mathbf{X})~\leq~k$. Všimněme si, že nemá smysl v predikátech
používat samotnou rovnost, protože numerická simulace vrací data s~určitou chybou.
K těmto vyčíslitelným funkcím přísluší složený $s = \mu_1(x)|| \ldots|| \mu_k(x)$,
indexy skládaných signálů představují atomické propozice $p$.

Gramatiku temporální logiky signálů definujeme podle předpisu \ref{eq:stl:grammar},
ve kterém $p$ značí atomickou propozici.

\begin{align}\label{eq:stl:grammar}
\varphi := T~|~p~|~\neg\varphi~|~\varphi_1 \wedge \varphi_2~|~\varphi_1\mathcal{U}_{[a,b]}\varphi_2
\end{align}

Ze základních formulí lze odvodit další standardně používané výrokové a temporální operátory.
Nejpoužívanějšími jsou výrokový operátor $\vee$ a temporální operátory $\mathcal{F}$ a $\mathcal{G}$,
které intuitivně odpovídají už zmíněným výrokům \uv{v budoucnu nastane \ldots} a \uv{vždy platí \ldots}.

\begin{align}\label{eq:stl:other}
\begin{array}{ll}
\varphi_1\vee\varphi_2 		&\equiv \neg\varphi_1 \wedge \neg\varphi_2		\\
\mathcal{F}_{[a,b]}\varphi 	&\equiv T\mathcal{U}_{[a,b]}\varphi				\\
\mathcal{G}_{[a,b]}\varphi 	&\equiv \neg\mathcal{F}_{[a,b]}\neg\varphi		\\
\end{array}
\end{align}

Re\-la\-ce $(s, t) \models \varphi$ značí, že daný signál $s$ splňuje formuli $\varphi$
počínaje pozicí v čase $t$, a je definována induktivně předpisem \ref{eq:stl:semantics}.
Signál $s$ splňuje formuli $\varphi$ právě tehdy, když $(s, 0) \models \varphi$.

\begin{align}\label{eq:stl:semantics}
\begin{array}{ll}
(s, t) \models p				&\Longleftrightarrow \pi_p(s)[t] = T			\\
(s, t) \models \neg \varphi		&\Longleftrightarrow (s, t) \not\models \varphi	\\
(s, t) \models \varphi_1 \wedge \varphi_2	&\Longleftrightarrow (s, t) \models \varphi_1 \textrm{ a současně } (s, t) \models \varphi_2	\\
(s, t) \models \varphi_1 \mathcal{U}_{[a,b]} \varphi_2 	&\Longleftrightarrow \exists t' \in [t+a, t+b] . (s, t') \models \varphi_2			\\														
&~~~~~~~~\textrm{ a současně } \forall t'' \in [t, t'] . (s, t'') \models \varphi_1
\end{array}
\end{align}

Operátor $\mathcal{U}$ je zde definován s trochu jinou sé\-man\-ti\-kou,
než je běžné. Požaduje se zde silnější podmínka -- aby existoval stav (čas),
pro který platí obě vlastnosti $\varphi_1$ a $\varphi_2$, tedy aby existoval
čas $t' \in [t + a, t + b]$ takový, že $(s, t') \models \varphi_1$
a současně $(s, t') \models \varphi_2$. To však nemění sémantiku ostatních známých
temporálních operátorů $\mathcal{F}$ a $\mathcal{G}$.

\begin{align}\label{eq:stl:semantics}
\begin{array}{ll}
(s, t) \models \mathcal{F}_{[a,b]}\varphi &\longleftrightarrow	\exists t'\in[t+a, t+b].(s,t') \models \varphi		\\
(s, t) \models \mathcal{G}_{[a,b]}\varphi &\longleftrightarrow	\forall t'\in[t+a, t+b].(s,t') \models \varphi
\end{array}
\end{align}

Standardní sémantika temporálních logik obecně neumožňuje ověření platnosti temporálních operátorů
na konečných signálech kromě některých případů, jako je splněnost $\mathcal{F}\varphi$
nebo nesplněnost $\mathcal{G}\varphi$, jejichž platnost může být ověřena v konečném čase.
Tento problém naštěstí odpadá zavedením časových intervalů. I přesto však existují
formule a signály, u kterých o platnosti rozhodnout nelze. Příkladem může být formule $F_{[a, b]}\varphi$
a signál o délce kratší než $b$.

Z tohoto důvodu se definuje délka nezbytná k ověření platnosti dané formule nad daným signálem,
opět induktivně.

\begin{align}\label{eq:stl:min:length}
\begin{array}{ll}
||p||									&= 0									\\
||\neg\varphi||							&= ||\varphi||							\\
||\phi_1 \wedge \phi_2||				&= max(||\varphi_2||,||\varphi_2||) 	\\
||\phi_1 \mathcal{U}_{[a,b]} \phi_2||	&= max(||\varphi_2||,||\varphi_2||) + b	\\
\end{array}
\end{align}

\subsection{Příklad vlastnosti}\label{section:stl:example}

Temporální logika signálů umožňuje formulovat celou řadu vlastností. Me\-zi ty
nejznámější a nejčastěji používané patří oscilace. Je dobré si u\-vě\-do\-mit, že oscilaci
lze chápat mnoha různými způsoby. V~praxi se nevyplatí klást přísná omezení na přesnost
periody, velikost amplitudy či přesnost stavu, kterým systém periodicky prochází. 

Představme si systém o jedné stavové proměnné $x$, ve které se vzrů\-sta\-jí\-cí intenzitou
osciluje hodnota této proměnné. Perioda oscilace je konstantní, avšak její amplituda ne. V~této práci
použitá logika se vyjadřuje o hodnotách proměnné $x$, tudíž oscilaci budeme
chápat jako periodické pře\-kra\-čo\-vá\-ní dolní a horní meze. Atomickými propozicemi jsou tedy
predikáty $[x] \geq k$ a $[x] \leq -k$, jejichž platnost lze vidět na obrázcích
\ref{fig:stl:example:geq:limit} a \ref{fig:stl:example:leq:limit}.

\begin{figure}[t]
\begin{center}
\subfigure[{$[x] \geq k$}]{
	\includegraphics[width=.48\textwidth]{../images/generated/stl-example-geq-limit.pdf}\label{fig:stl:example:geq:limit}
}
\subfigure[{$[x] \leq -k$}]{
	\includegraphics[width=.48\textwidth]{../images/generated/stl-example-leq-limit.pdf}\label{fig:stl:example:leq:limit}
}
\subfigure[{$[x] \geq k \wedge \mathcal{F}_{[0, \frac{1}{2}]}[x] \leq -k$}]{
	\includegraphics[width=.48\textwidth]{../images/generated/stl-example-geq-limit-future-leq-limit.pdf}\label{fig:stl:example:geq:limit:future:leq:limit}
}
\subfigure[{$\mathcal{F}_{0, \frac{1}{2}}([x] \geq k \wedge \mathcal{F}_{[0, \frac{1}{2}]}[x] \leq -k)$}]{
	\includegraphics[width=.48\textwidth]{../images/generated/stl-example-future-geq-limit-future-leq-limit.pdf}\label{fig:stl:example:future:geq:limit:future:leq:limit}
}
\caption{Pravdivostní hodnota atomických propozic v čase.}
\end{center}
\end{figure}

Jeden cyklus lze popsat tak, že se hodnota sledované proměnné nachází nad horní mezí a zároveň někdy
v budoucnu klesne pod dolní mez, tedy $[x] \geq k \wedge \mathcal{F}_{[0, \frac{1}{2}]}[x] \leq -k$.
Přidání operátoru $\mathcal{F}$ zajistí, že se v daném času do určité doby provede jeden oscilační cyklus,
viz obrázek \ref{fig:stl:example:future:geq:limit:future:leq:limit}. To, že systém
osciluje znamená, že toto platí pro každý časový okamžik, což vyjádříme operátorem $\mathcal{G}$.

Ve zde uvedeném příkladu systém neosciluje s požadovanou amplitudou již od začátku,
tudíž je potřeba do formule přidat ještě nějaké čekání v~podobě operátoru $\mathcal{F}$.
Výsledná formule tedy vypadá následovně:

\begin{align}\label{eq:stl:lotkav:oscil}
\mathcal{F}_{[0, \textrm{čekání}]}\mathcal{G}_{[0, \textrm{doba oscilace}]}\mathcal{F}_{[0, \textrm{perioda}]}\Big([x] \geq k \wedge \mathcal{F}_{[0, \textrm{perioda}]}[x] \leq -k\Big)
\end{align}

Z uvedeného příkladu je zřejmě, že platnost formule nad daným signá\-lem
lze ověřit algoritmem, jehož průběh bude kopírovat strukturu formule. Jak
přesně ověřovací algoritmus vypadá, ukáže sekce \ref{section:robustness},
která navíc lehce rozšíří chápání pravdivosti jako takové.

\chapter{Algoritmus pro analýzu dynamických systémů}\label{chapter:algorthm}

V této kapitole ukážeme jednu z možností, jak přistoupit k analýze modelů zadaných
pomocí obyčejných diferenciálních rovnic. Zde uvedená analýza se snaží řešit následující
problémy:
\begin{enumerate}
	\item\label{item:init}	Máme k dispozici již hotový model, jehož chování sedí se skutečným
			systémem pro jedno konkrétní nastavení iniciálních hodnot stavovým
			proměnným. Splňuje tento model požadované vlastnosti i pro jiná nastavení?
	\item	Máme kostru modelu, v němž se vyskytuje několik parametrů, jejichž hodnota
			není známá. Jak nastavit parametry modelu tak, aby splňoval dané chování~\cite{aster2012}?
\end{enumerate}

V~bodu \ref{item:init} lze jít ještě dál než ke kontrole modelů. Můžeme si představit
poměrně přesný model, který využijeme k analýze v~podmínkách, které nelze
navodit u reálného systému. Typickým příkladem může být živý organismus v~toxickém
prostředí nebo extrémně vysoká nákaza šířící se ce\-lo\-svě\-to\-vou populací.

\section{Definice problému}\label{section:initial:condtion:problem:definition}

Nechť je dynamický systém $\mathcal{DS} = (\mathbf{X}, f)$, kde $\mathbf{X} = (x_1, \ldots, x_n)$
je vektor stavových proměnných a dynamiku systému popisují obyčejné diferenciální rovnice $\frac{dx_i}{dt} = f_i(\mathbf{X})$.
Tento systém rovnic souhrnně označíme jednou rovnicí $\frac{d\mathbf{X}}{dt} = f(\mathbf{X})$. Narozdíl
od obecného modelu zadaného pomocí systému obyčejných diferenciálních rovnic budeme předpokládat,
že funkce stojící na pravé straně nezávisí na čase. Nechť $\mathbf{P}  = (p_1, \ldots, p_m)$ značí
parametry dostupné v~systému rovnic. Systém rovnic s konkrétním ohodnocení parametrů $[\mathbf{P}]$
označíme $f_{\mathbf{P} \leftarrow [\mathbf{P}]}$.

Pro každou stavovou proměnnou $x_i$ je dán interval $\mathcal{I}_{x_i} = [\iota_{x_i}^{min}, \iota_{x_i}^{max}]$
a pro každý parametr $p_j$ interval $\mathcal{I}_{p_j} = [\iota_{p_j}^{min}, \iota_{p_j}^{max}]$
Tyto intervaly omezují nastavení iniciálních hodnot $[x_i]_0$ stavových proměnných $x_i$, respektive o\-hod\-no\-ce\-ní $[p_j]$
parametrů $p_j$, a určují prostor iniciálních podmínek \cite[str. 23]{drazan2011}.

\begin{align}
\mathcal{I} &= \mathcal{I}_{x_1} \times \mathcal{I}_{x_n} \times \ldots \times \mathcal{I}_{p_1} \times \ldots \times \mathcal{I}_{p_m}
\end{align}

$\mathcal{I} \models [\mathbf{X}]_0$ značí, že množina iniciálních hodnot $[\mathbf{X}]_0$ splňuje
omezení prostoru iniciálních podmínek, podobné označení $\mathcal{I} \models [\mathbf{P}]$ zavedeme
pro ohodnocení parametrů $\mathbf{P}$.

Dále je dána numerická metoda $\mathcal{M}_\varepsilon (f, [\mathbf{X}]_{t_0}, \Delta t) = [\mathbf{X}]_{t_0 + \Delta t}$,
která pro daný systém diferenciálních rovnic $f$,
ohodnocení stavových proměnných $[\mathbf{X}]_{t_0}$ v čase $t_0$ a časový krok $\Delta t$
vrátí stav $\mathbf{X}_{t_0 + \Delta t}$ v čase $t_0 + \Delta t$ s relativní chybou $\varepsilon$.
Díky numerické metodě lze sestrojit již dříve zmíněnou posloupnost vektorů
$\mathcal{M}^{\tau}_\varepsilon(f, [\mathbf{X}]_{t_0}, \Delta t) = [\mathbf{X}]_{t_0}, [\mathbf{X}]_{t_0 + \Delta t}, \ldots, [\mathbf{X}]_{t_0 + \tau}$, kde $\tau$
je množství času, po~který model simulujeme.

Je-li dáná formule temporální logiky signálů $\varphi$, dynamický systém $\mathcal{DS}$
a prostor iniciálních podmínek $\mathcal{I}$, pak řešeným problémem je najít
části prostoru $\mathcal{S}, \mathcal{N} \subseteq \mathcal{I}$ takové, že platí
vztah \ref{eq:initial:value:problem} \cite[str. 23]{drazan2011}.

\begin{align}\label{eq:initial:value:problem}
\forall [\mathbf{X}]_0 \forall [\mathbf{P}] . (\mathcal{S} \models [\mathbf{X}]_0 \wedge \mathcal{S} \models [\mathbf{P}])
\Rightarrow \mathcal{M}^\tau_\varepsilon(f_{\mathbf{P} \leftarrow [\mathbf{P}]}, [\mathbf{X}]_0, \Delta t) \models \varphi \\
\forall [\mathbf{X}]_0 \forall [\mathbf{P}] . (\mathcal{N} \models [\mathbf{X}]_0 \wedge \mathcal{N} \models [\mathbf{P}])
\Rightarrow \mathcal{M}^\tau_\varepsilon(f_{\mathbf{P} \leftarrow [\mathbf{P}]}, [\mathbf{X}]_0, \Delta t) \not\models \varphi
\end{align}

Tyto části prostoru popisují ohodnocení počátečních stavů a parametrů, ze kterých se daný systém vyvíjí
s požadovanou vlastností $\varphi$ a bez po\-ža\-do\-vané vlastnosti. Naivní algoritmus řešící
tento problém do prostoru iniciálních podmínek vloží určité množství bodů a tyto body
použije pro simulaci chování modelu, nad kterým se následně provede ověření vlastnosti.
Počet bodů, tedy míra zahuštění prostoru iniciálních podmínek, závisí na požadované přesnosti
analýzy. I přes nesporné výhody tohoto přístupu snadno narazíme na výpočetní limity
potřebného množství bodů, a proto je vhodné pokusit se počet bodů omezit.


\section{Původní algoritmus}

Původní algoritmus, na kterém staví tato práce vychází z velice důležitého předpokladu.
\uv{Většina řešení začínajících v iniciálních bodech blízko sebe zůstávají blízko sebe
i v průběhu času~\cite[str. 25]{drazan2011}.} Předpokládá se tedy, že chování určená blízkými
hodnotami z prostoru iniciálních podmínek mají i blízkou míru platnosti dané formule.
Není třeba tedy zjišťovat chování pro všechny body prostoru inicálních podmínek,
ale jen pro určitou množinu reprezentantů, pro kterou platí \cite[str. 25]{drazan2011}:

\begin{enumerate}
	\item	chování blízké reprezentatovi zůstane blízké na celém časovém intervalu,
			na kterém je daná formule ověřována,
	\item	množina reprezenantů pokrývá celý prostor iniciálních podmínek.
\end{enumerate}

Algoritmus do prostoru iniciálních podmínek vkládá body tak dlouho, dokud si trajektorie chování
určených blízkými body jsou vzdálenější než daná vzdálenost $\delta$. Nad simulovanými chováními
se ověří platnost formule. Výsledkem je určité množství bodů, u kterých dostáváme informace,
zda z nich simulované chování splňuje či nesplňuje danou vlastnost. Tyto body nastíní
hranice regionů platnosti a neplatnosti se zvolenou přes\-nos\-tí~$\delta$.

\begin{algorithm}
\caption{Analýza prostoru inicálních podmínek}
\begin{algorithmic}
\Require 	$\mathcal{DS} = (\mathbf{X}, f), \mathcal{I}, \varphi, \Delta t, \delta, \varepsilon$
\Ensure 	$\textsc{Result} = \{([\mathbf{F}_0]_0, s_1), \ldots ([\mathbf{F}_k]_0, s_k)\}$ -- body a splněnost $\varphi$
\State		$\textsc{M}_{new} 	\gets $ počáteční zahuštění $\mathcal{I}$
\State		$\textsc{Result} \gets \emptyset$
\While{$\textsc{M}_{new} \neq \emptyset$}
	\State $\textsc{M}_{old} \gets \textsc{M}_{new}, \textsc{M}_{new} \gets \emptyset$
	\For{$([\mathbf{X}_{main}]_0, [\mathbf{P}_{main}]) \in \textsc{M}_{old}$}
		\State $\textsc{Trajectory}_{main} \gets \mathcal{M}^{||\varphi||}_\varepsilon(f_{\mathbf{P} \gets [\mathbf{P}_{main}]}, [\mathbf{X}_{main}]_0, \Delta t)$
		\State $\textsc{Satisfied}_{main} \gets $ splněnost $\varphi$ nad $\textsc{Trajectory}_{main}$
		\State $\textsc{Result} \gets \textsc{Result} \cup \{(\textsc{Satisfied}_{main}, [\mathbf{X}_{main}]_0, [\mathbf{P}_{main}])\}$
		\State $\textsc{Neigh} \gets $ body sousedící s bodem $([\mathbf{X}]_0, [\mathbf{P}])$
		\For{$([\mathbf{X}_{neigh}]_0, [\textbf{P}_{neigh}]) \in \textsc{Neigh}$}
			\State $\textsc{Trajectory}_{neigh} \gets \mathcal{M}^{||\varphi||}_\varepsilon(f_{\mathbf{P} \gets [\mathbf{P}_{neigh}]}, [\mathbf{X}_{neigh}]_0, \Delta t)$
			\State $\textsc{Satisfied}_{neigh} \gets $ splněnost $\varphi$ nad $\textsc{Trajectory}_{neigh}$
			\State $\textsc{Result} \gets \textsc{Result} \cup \{(\textsc{Satisfied}_{main}, [\mathbf{X}_{main}]_0, [\mathbf{P}_{main}])\}$
			\State $\textsc{Distance} \gets $ vzdálenost $\textsc{Trajectory}_{main}$ a $\textsc{Trajectory}_{neigh}$
			\If{$\textsc{Distance} > \delta$}
				\State	$\textsc{M}_{new} \gets \textsc{M}_{new} \cup \{(\frac{[\mathbf{X}_{main}]_0 + [\mathbf{X}_{neigh}]_0}{2}, \frac{[\mathbf{P}_{main}] + [\mathbf{P}_{neigh}]}{2})\}$
			\EndIf
		\EndFor
	\EndFor
\EndWhile
\end{algorithmic}
\end{algorithm}

Je samozřejme otázkou, jakým způsobem konkrétně probíhá počáteční zahuštění prostoru
iniciálních podmínek, co přesně znamená obsahuje mno\-žina sousedů daného bodu a jak
se se změří vzdálenost dvou chování. Poslední zmíněné otázce se podrobně zabývá \cite{drazan2011}.
Zbytek bude ještě rozveden v sekci \ref{section:algorithm:updated} společně s tím,
jak zvolit hodnotu $\delta$.

\section{Robustnost}\label{section:robustness}

Platnost formule lze chápat trochu šířeji než prosté \uv{platí}/\uv{neplatí}.
Nemusíme se pouze ptát, zda dané chování splňuje danou formuli, otázku lze posunout dál.
Jak moc dané chování splňuje danou formuli? Jak moc je vlastnost nesplňena? Pro účely
této práce tyto otázky zcela postačují, ale samozřejmě je možné požadovat ještě víc.
Jak moc je daný systém robustní vůči změně podmínek, změnám teplot nebo koncentrací
chemických látek?

V této sekci zavedeme pojem \textit{lokální a globální robustnosti}. Lokální robustnosti
rozumíme míru, do jaké je daná vlastnost splněna na jednom chování. Globální robustnost
na druhé straně vztáhneme na celý systém, tzn. že zahrnuje míru platnosti dané vlastnosti
nad větším množství chování, která vzniknou tzv. \textit{perturbacemi}. Existuje jedno
referenční chování za ideálních podmínek, a pak mnoho perturbovaných chování za podmínek
ne tak i\-deál\-ních. Perturbace lze chápat různě, v této práci odpovídají prostoru i\-ni\-ciál\-ních
podmínek definovaného v sekci \ref{section:initial:condtion:problem:definition}. 

Jednou z věcí, které je třeba předem zmínit, je, že pro účely vylepšení algoritmu pro
analýzu dynamických systémů, se na robustnost díváme z pohledu chování systému \cite{donze2011}.
To znamená, že se při výpočtu lokální robustnosti snažíme ohraničit prostor okolo jednoho chování,
ve kterém má daná vlastnost stejnou platnost. Podobně lze k problému přistoupit z opačné strany \cite{rizk2009}.
Ve formuli identifikovat parametry, napočítat podprostor parametrického prostoru, ve kterém je formule
pro dané chování splněna splněna a určit vzdálenost již konkrétní formule s dosazenými parametry
od tohoto podprostoru.

\subsection{Lokální robustnost}

Nechť $s_1$, $s_2$ jsou signály nad doménou $\mathbb{D}$ a $d: \mathbb{D}^n \rightarrow \mathbb{R}^{+}$
funkce určující vzdálenost dvou bodů v prostoru $\mathbb{R}^n$. 
Vzdálenost dvou signálů zavedeme předpisem \ref{eq:signal:distance}.

\begin{align}\label{eq:signal:distance}
\sigma(s_1, s_2) = {\displaystyle \sup_{t \in \mathbb{R}^{+}}} \{d(s_1(t), s_1(t))\}
\end{align}

Od robustnosti $\rho$ budeme požadovat konzistentní chování s již zavedenou dvouhodnotovou platností
formule. 

\begin{align}\label{eq:signal:distance}
\rho(\varphi, s) = \left\{\begin{array}{r@{\quad}c}
\inf\{\sigma(s, s') | \forall s'. s' \not\models \varphi \}	& \textrm{pokud } s \models \varphi	\\
- \inf\{\sigma(s, s') | \forall s'. s' \models \varphi \}	& \textrm{pokud } s \not\models \varphi
\end{array} \right.
\end{align}

Oborem hodnot funkcí $\mu_i$ odpovídajících atomickým prozicím tentokrát není množina $\{T, F\}$,
nýbrž množina reálných čísel $\mathbb{R}$. Toto chápání je spíše technického charakteru a převod
ze starého chápání je přímočarý.

\begin{align}\label{eq:stl:semantics}
\begin{array}{ll}
\mu_i = f(\mathbf{X}) \geq k		\leadsto \mu_i = f(\mathbf{X}) - k							\\
\mu_i = f(\mathbf{X}) \leq k		\leadsto \mu_i = k - f(\mathbf{X})
\end{array}
\end{align}

Obdobně jako při výpočtu dvouhodnotové platnosti formule i u robustnosti je potřeba se odkazovat
na robustnost v určitém čase $\rho(\varphi, s, t)$, která se definuje induktivně ke struktuře
formule. Výslednou robustnost dostaneme položením $\rho(\varphi, s) = \rho(\varphi, s, 0)$.

\begin{align}\label{eq:stl:semantics}
\begin{array}{ll}
\rho(p, s, t)											&= \pi_p(s)[t]											\\
\rho(\neg\varphi, s, t)									&= - \rho(\varphi, s, t)								\\
\rho(\varphi_1 \wedge \varphi_2, s, t)					&= \min\Big(\rho(\varphi_1, s, t), \rho(\varphi_1, s, t)\Big)	\\
\rho(\varphi_1 \mathcal{U}_{[a, b]} \varphi_2, s, t)	&= {\displaystyle \max_{t' \in [t + a, t + b]}} min\Big(\rho(\varphi_2, s, t'), {\displaystyle\min_{t'' \in [t, t']}}\rho(\varphi_1, s, t'')\Big)
\end{array}
\end{align}

Pro odvozené temporální operátory $\mathcal{F}$ a $\mathcal{G}$ lze definovat robustnost
samostatně, což umožňuje naimplementovat její výpočet efektivněji.

\begin{align}\label{eq:stl:semantics}
\begin{array}{ll}
\rho(\mathcal{F}_{[a, b]}\varphi)		&= {\displaystyle \max_{t' \in [t + a, t + b]}} \Big(\rho(\varphi, s, t')\Big)		\\
\rho(\mathcal{G}_{[a, b]}\varphi)		&= {\displaystyle \min_{t' \in [t + a, t + b]}} \Big(\rho(\varphi, s, t')\Big)		
\end{array}
\end{align}

Jestliže máme k dispozici analyzovaný primární signál, lze použit funkci robustnosti pro definici tzv. sekundárních
signálů příslušejících každé podformuli $\varphi'$ dané formule $\varphi$, kde $s_{\varphi'}[t] = \rho(\varphi', s, t)$.
a $|s_{\varphi'}| = ||\varphi'||$. Tyto se\-kun\-dár\-ní signály nám dávají informaci o tom, jak moc se lze od primárního
signálu vzdálit, aby daná podformule zůstala ještě platná v případě klad\-ných hodnot sekundárního signálů
nebo neplatná v případě hodnot zá\-por\-ných. Krajní hodnotou je signál nulový, který značí hranici platnosti formule.

%\subsection{Příklad sekundárního signálu}

Vraťme se k příkladu formule popisující oscilaci systému s jednou stavovou proměnnou ze 
sekce \ref{section:stl:example}. Ukážeme si, jak vypadá sekundární signál pro podformuli
$\mathcal{F}_{[0, \frac{1}{2}]}x \leq -k$.

\begin{figure}[h!]
\begin{center}
\subfigure[$x \leq -k$]{
	\includegraphics[width=.48\textwidth]{../images/robustness-example-leq-limit.pdf}
}
\subfigure[$\mathcal{F}_{[0, \frac{1}{2}]}x \leq -k$]{
	\includegraphics[width=.48\textwidth]{../images/robustness-example-future.pdf}
}
\caption{Znázornění sekundárních signálů pro některé z podformulí vlastnosti oscilace. Zelené podbarvení
značí platnost dané formule v daném čase, červené naopak neplatnost.}
\end{center}
\end{figure}

\subsection{Výpočet lokální robustnosti}

Za zmínku stojí způsob, jakým lze výpočet robustnosti naimplementovat. Od nejvíce zanořených
podformulí se počítají sekundární signály, které se použijí pro výpočet sekundárních signálů
pro nadřazené podformule. Není překvapením, že sekundární signál pro základní predikátové operátory
$\neg$ a $\wedge$ lze spočítat v lineárním čase vzhledem k délce signálu. Pro hledání maxima,
respektive minima pro všechny podsekvence dané sekvence

prvků existuje algoritmus s lineární časovou složitostí~\cite{lemire2006}. Díky tomu lze
výpočet sekundárního signálu i odvozených operátorů $\mathcal{F}$ a $\mathcal{G}$ pro\-vést
rovněž v lineárním čase. Použitím pomocné funkce, jejíž struktura je popsaná v algoritmu \ref{algorithm:unary:signal}
získáme předpis 
\begin{align}\label{eq:future:gloabally:robustness:impl}
\begin{array}{ll}
s_{\mathcal{F}_{[a,b]}\varphi} = \textsc{Signal}([a,b], \varphi, >)				\\
s_{\mathcal{G}_{[a,b]}\varphi} = \textsc{Signal}([a,b], \varphi, <)
\end{array}
\end{align}

Výpočet sekundárního signálu pro operátor $\mathcal{U}$ má kvadratickou časovou složitost,
čehož lze docílit přímočarou implementací.

\begin{algorithm}
\caption{datová struktura \textsc{Lemire-Queue}\cite{lemire2006}}
\begin{algorithmic}
\Require 	$\prec \subseteq \mathbb{R}^2$ \Comment{ostré uspořádání}
\State $deque$ \Comment{fronta s přístupem k oběma koncům}
\Function{Lemire-Queue.offer}{$time, value$}
	\While{$\neg deque.\textsc{isEmpty()} \wedge value \prec \textsc{Dequeue.getLast()}.value$}
		\State $deque$.\textsc{removeLast()}
	\EndWhile
	\State $deque$.\textsc{offer(}$time, value$\textsc{)}
\EndFunction
\Function{Lemire-Queue.peek}{~}
	\State\Return $deque$.\textsc{getFirst()}
\EndFunction
\Function{Lemire-Queue.Poll}{~}
	\State\Return $deque$.\textsc{removeFirst()}
\EndFunction
\end{algorithmic}
\end{algorithm}

\begin{algorithm}\label{algorithm:unary:signal}
\caption{pomocná funkce pro sekundárního signálu}
\begin{algorithmic}
\Function{Signal}{$[a, b]$, $s_\varphi$, $\prec$}
	\State	$queue \gets $ nová \textsc{Lemire-Queue} s uspořádáním $\prec$
	\State	$monitor \gets ()$
	\State	$t' \gets t_0$
	\While{$t'\leq t_0 + b - \Delta t$}
		\State $queue$.\textsc{Offer}($s_\varphi[t']$)
		\State $t' \gets t' + \Delta t$
	\EndWhile
	\State	$t' \gets t_0$
	\While{$t' \leq |s|$}
		\State	$qeue$.\textsc{offer}($s_\varphi[t']$)
		\State	$monitor[t'] \gets queue$.\textsc{peek()}
		\If{$queue$.\textsc{peek()}.$time = t'$}
			\State	$queue$.\textsc{poll()}
		\EndIf
		\State	$t' \gets t' + \Delta t$
	\EndWhile
	\State\Return $monitor$
\EndFunction
\end{algorithmic}
\end{algorithm}

\subsection{Globální robustnost}

\begin{align}\label{eq:future:gloabally:robustness:impl}
\begin{array}{ll}
R_{\varphi, P}^\mathcal{S} = {\displaystyle\int_{p \in P}}prob(p) \cdot D_\varphi^\mathcal{S}dp
\end{array}
\end{align}

\begin{align}\label{eq:future:gloabally:robustness:impl}
\begin{array}{ll}
R_{\varphi, P}^\mathcal{S} = {\displaystyle\int_{p \in P}}prob(p) \cdot \rho(\varphi, s_p)dp
\end{array}
\end{align}

% Kitano - globalni robustnost

% STL - lokalni robustnost

% Priklad

% Vypocet robustnosti

\section{Upravený algoritmus}\label{section:algorithm:updated}

\chapter{Implementace}
Následující kapitola se věnuje aplikaci Parasim \cite{TODO}, která implementuje algoritmus pro
analýzu dynamických systémů zmíněný v sekci \ref{section:algorithm:updated}. Aplikace Parasim
vznikla na základě prototypu dostupného v \cite{drazan2011}. Cílem prototypu je názorně
zobrazit průběh výpočtu původního algoritmu. Uživateli se ukazuje, jakým způsobem
se počítá vzdálenost mezi trajektoriemi chování, kde je nutné zahušťovat a jak dopadlo
ověření platnosti. Vzhledem k tomu, že prototyp například používá pouze jednoduchou metodu
numerické simulace k nalezení trajektorií chování, není vhodné jej použít k analýze složitějších modelů.

V nové implementaci bylo oproti prototypu třeba zahrnout následující:

\begin{enumerate}
	\item	úprava algoritmu dle sekce \ref{section:algorithm:updated},
	\item	paralelní výpočet ve sdílené nebo distribuované paměti,
	\item	rozšiřitelnost, modularita a otevřenost k rozdílné implementaci již naimplementovaných částí,\label{item:features:extensibility}
	\item	zobrazení výsledků analýzy pro vícedimenzionální prostory ini\-ciál\-ních podmínek.\label{item:features:visualisation}
\end{enumerate}

Tato práce se zabývá všemi zmíněnými body kromě bodu \ref{item:features:visualisation},
který je však v Parasimu již také vyřešen. Vzhledem k monolitické implementaci prototypu
nedošlo k jeho úpravám a rozšíření, ale vytvořila se zcela od základu nová aplikace Parasim.
Bod \ref{item:features:extensibility} je důležitý z několika důvodů. Pro různé části algoritmu
existuje více způsobů, jak je naimplementovat. Existuje mnoho nástrojů umožňující
numerickou simulací na základě systému diferenciálních rovnic získat trajektorii chování.
Mnoho věcí zmíněných dříve se vyvíjí. V případě výpočtu robustnosti se nemusíme omezit
pouze na temporální logiku signálů, ale můžeme zavést logiku novou, expresivněj\-ší.

\section{Architektura}

Parasim je aplikace napsaná v Javě 7 složená z většího množství artefaktů pro
sestavovací nástroj Maven. Zdrojové kódy jsou pod licencí GNU GPL verze 3~\cite{gpl} k dispozici
v Git \cite{chacon2009} repozitáři \url{https://github.com/sybila/parasim}. Artefakty sestavené
z poslední verze těchto zdrojových jsou publikovány do Maven repozitáře
\url{http://repository-sybila.forge.cloudbees.com/snapshot/}, o což se stará veřejná
instance nástroje Jenkins \cite{jenkins} \url{http://www.cloudbees.com/}. Artefakty
vydaných verzí jsou k nalezení v Maven repozitáři \url{http://repository-sybila.forge.cloudbees.com/release}.
Hlavní aplikaci Parasimu představuje artefakt \texttt{org.sybila.parasim.application:parasim}.
Pou\-ži\-tí aplikace je popsáno v příloze \ref{appendix:usage}

Jádro Parasimu tvoří artefakt \texttt{org.sybila.parasim:core}, který řídí ži\-vot\-ní
cyklus všech modulů a zajišťuje základní funkcionalitu. To zajišťuje kontejner,
jehož instance je vytvořena a spuštěn v rámci aplikace. Kontejner je velkou mírou inspi\-ro\-ván
konceptem \textit{Context and Dependency Injection} zavedeným v~Java~EE~6~\cite{jendrock2010} specifikací
JSR-299 \cite{jsr299}. V žádném případě se nejedná o~implementaci tohoto standardu,
nýbrž jen o volnou inspiraci některými zá\-klad\-ní\-mi myšlenkami.

Existují zde kontexty, které mají různou délku a je možné je za\-no\-řo\-vat.
Rozšíření pro jednotlivé kontexty nabízejí služby a aplikace definuje na různých
místech v kódu závislosti na těchto službách. Kontejner se stará o to,
aby v případě požadavku na nějakou službu bylo zavoláno příslušné rozříření,
které ji poskytuje, aby vytvořilo objekt služby, a v případě vy\-pr\-še\-ní kontextu,
aby objekt služby vhodným způsobem zničilo.

Za vytvoření a zničení objektu služby je zodpovědný autor rozšíření, který také definuje rozhraní
služby. Aplikace není téměř žádným způsobem závislá na implementaci rozhraní,
a jednotlivé implementace lze tak jednoduše zaměňovat bez toho, aniž bychom
v~aplikačním kódu cokoliv mě\-ni\-li. Kontejner obsahuje i některé základní služby,
mezi které patří například jednotný přístup ke konfiguraci nebo způsob, jak injektovat
služby do atributů daných objektů.

\subsection{Životní cyklus}

Aby bylo možné jádro Parasimu používat, je nutné nastartovat jeho životní cyklus pomocí třídy
\href{https://github.com/sybila/parasim/blob/master/core/src/main/java/org/sybila/parasim/core/impl/ManagerImpl.java}{\texttt{ManagerImpl}}.
Tato třída představuje vstupní bod pro použití všech dále popsaných služeb.
Životní cyklus řízený vytvořením, nastartováním a zničením manažera je
znázorněn diagramem \ref{figure:parasim:lifecycle}. Jakmile je manažer vytvořen,
jsou načtena zá\-klad\-ní rozšíření, kterým je oznámena událost \href{https://github.com/sybila/parasim/blob/master/core/src/main/java/org/sybila/parasim/core/event/ManagerProcessing.java}{\texttt{ManagerProcessing}}.

\begin{figure}[h!]
\label{figure:parasim:lifecycle}
\begin{center}
\resizebox{\textwidth}{!}{
\begin{tikzpicture}  	[node distance=.5cm,start chain=going below]
\tikzset{>=stealth',
	event/.style={
    	rectangle, 
    	rounded corners, 
    	fill=green!20,
    	draw=black, very thick,
    	text width=9em, 
    	minimum height=3em, 
    	text centered, 
    	on chain},
  	line/.style={draw, thick, <-},
	class/.style={
    	rectangle, 
    	rounded corners, 
    	fill=black!10,
    	draw=black, very thick,
    	text width=8em, 
    	minimum height=3em, 
    	text centered,
		on chain},
	package/.style={
		rectangle,
		draw=black!50, dashed,
		rounded corners,
		inner sep=0.3cm,
		on chain},
  	every join/.style={->, thick,shorten >=1pt}, 	
 	scope/.style={decorate},
	code/.style={
		rectangle,
		draw=black!50, dashed,
		rounded corners,
		text width=15em,
    	minimum height=3em, 
    	text centered,
		node distance=7cm}
}
	\node[event, join] (before-app-context) {BEFORE\\kontext aplikace};
	\node[event, join] (processing) {PROCESSING};
	\node[code, left of=before-app-context] (code-create) {Manager m = Manager.create()};
	\node[package] (processing-package) {
		\begin{tikzpicture}
		\begin{scope}[solid, start branch=venstre, every join/.style={->, thick, shorten <=1pt}]
			\node[class] (enrichment) {obohacování};
			\node[class, on chain=going left] (configuration) {konfigurace};
			\node[class, on chain=going below] (lifecycle) {životní cyklus};
			\node[class, on chain=going right] (extension-loader) {načtení rozšíření};
			\node[class, on chain=going right] (remote) {vzdálená správa};
			\node[class, on chain=going above] (logging) {logování};
		\end{scope}
		\end{tikzpicture}
	};
	\node[event, join] (started) {STARTED};
	\node[code, left of=started] (code-start) {m.start()};
	\node[package, join, inner sep=0.5cm, text width=15em, text centered] (other-extensions) {interakce s načtenými rozšířeními};
	\node[package, join, inner sep=0.5cm, text width=15em, text centered] (main-app) {hlavní kód aplikace};
	\node[event, join] (stopping) {STOPPING};
	\node[event, join] (after-app-context) {AFTER\\kontext aplikace};
	\node[code, left of=stopping] (code-shutdown) {m.destroy()};

	\begin{scope}[->, thick, shorten <=1pt] 
		\draw	(processing)	-> (processing-package);
	\end{scope}

	\begin{scope}[->, dashed, shorten <=1pt] 
		\draw	(code-create)	-> (before-app-context);
		\draw	(code-start)	-> (started);
		\draw	(code-shutdown)	-> (stopping);
	\end{scope}
\end{tikzpicture}}
\end{center}
\caption{Životní cyklus Parasimu. Zelené obdelníky představují události, které se propagují napříč rozšířeními.
Šedými obdelníkymi je znázorněna základní funkcionalita. Čerchovaně ohraničeně jsou vyznačeny části Java kódu.}
\end{figure}

Poté je možné manažera nastartovat, což vyústí ve vytvoření aplikačního kontextu.
S vytvořením každého kontextu je spojena událost \href{https://github.com/sybila/parasim/blob/master/core/src/main/java/org/sybila/parasim/core/event/Before.java}{\texttt{Before}}, která je propagovaná
do všech rozšíření náležejícím tomuto kontextu a v případě jiného než
aplikačního kontextu i do rozšíření kontextu rodičovského.
Po vytvoření aplikačního kontextu následuje událost \href{https://github.com/sybila/parasim/blob/master/core/src/main/java/org/sybila/parasim/core/event/ManagerStarted.java}{\texttt{ManagerStarting}},
na kterou mohou reagovat rozšíření definovaná mimo jádro Parasimu. Jakmile aplikace skončí,
je nutné manažera zničit. Následuje vyvolání poslední u\-dá\-los\-ti, na kterou mohou reagovat načtená rozšíření, \href{https://github.com/sybila/parasim/blob/master/core/src/main/java/org/sybila/parasim/core/event/ManagerStopping.java}{\texttt{ManagerStopping}}
a zničení aplikačního konextu spojené s událostí \href{https://github.com/sybila/parasim/blob/master/core/src/main/java/org/sybila/parasim/core/event/After.java}{After}, případně všech další dosud nezničených kontextů.

\subsection{Kontexty}

Manažer a objekty kontextů umožňují vytvářet nové kontexty skrze rozhraní \href{https://github.com/sybila/parasim/blob/master/core/src/main/java/org/sybila/parasim/core/api/ContextFactory.java}{ContextFactory}.
Aby mohl Parasim od se\-be rozlišit jednotlivé kontexty, používá speciální anotace
rozsahů. Deklarace takové anotace je samotná označená anotace \href{https://github.com/sybila/parasim/blob/master/core/src/main/java/org/sybila/parasim/core/annotation/Scope.java}{\texttt{Scope}},
jak je ukázáno ve zdrojovém kódu \ref{code:scope}.

\begin{lstlisting}[label={code:scope}, caption={Anotace rozsahu}]
@Scope
@Documented
@Retention(RetentionPolicy.RUNTIME)
@Target(ElementType.TYPE)
public @interface Application {}
\end{lstlisting}

Když je k dispozici anotace rozsahu rozlišující kontext, je možné nový kontext vytvořit
nový, jak je ukázáno v ukázce kódu \ref{code:context}. Vývojář je zodpovědný za to,
aby kontext zničil, když už jej nebude potřebovat, aby roz\-ší\-ře\-ním umožnil uvolnit zdroje.

\begin{lstlisting}[label={code:context}, caption={Vytvoření kontextu}]
// vytvori kontext Scope1
// s rodicovskym kontextem Application
Context context1 = manager.context(Scope1.class); 
// vytvori kontext Scope2
// s rodicovskym kontextem Scope1
Context context2 = context1.context(Scope2.class);
...
context2.destroy();
context1.destroy();
\end{lstlisting}

\subsection{Služby}

Základní funkcionalitou jádra Parasimu je poskytovat instance služeb, které jsou
vytvořeny pomocí rozšíření. Služby se definují pomocí rozhraní a může pro ně existovat
více implementací. Aby byl Parasim schopen od sebe rozlišit jednotlivé implementace
různých vlastností používá kvalifikátory. Kva\-li\-fi\-ká\-tor je anotace, v jejíž deklaraci byla použita anotace
\href{https://github.com/sybila/parasim/blob/master/core/src/main/java/org/sybila/parasim/core/annotation/Qualifier.java}{\texttt{Qualifier}}.

Pokud na daném místě aplikace není důležité, jakou typ implementace služby požadovat,
případě není známo, jaké kvalifikátory jsou vůbec k dispozici, je možné použít kvalifikátor
\href{https://github.com/sybila/parasim/blob/master/core/src/main/java/org/sybila/parasim/core/annotation/Default.java}{\texttt{Default}}
pro výchozí implementaci. Je na autorovi rozšíření, aby poskytl smysluplnou výchozí implementaci
poskytované služby.

Příkladem vhodného použití kvalifikátorů je rozšíření poskytující
vý\-počet robustnosti pro danou formuli nad danou trajektorii chování. Zde je možné zavést
různé kvalikátory pro různé temporální logiky. Výchozí implementace takové služby
by měla být schopna spočítat robustnost pro všechny podporované typy temporálních logik,
Avšak je možné, že se v takovéto implementaci bude nacházet méně efektivní algoritmus,
případě nějaká analýza předložené formule, což zbrzdí výpočet.

\begin{lstlisting}[label={code:qualifier}, caption={Kvalikátor}]
@Qualifier
@Target({
	ElementType.FIELD,
	ElementType.METHOD,
	ElementType.PARAMETER})
@Retention(RetentionPolicy.RUNTIME)
@Documented
public @interface Default {}
\end{lstlisting}

Manažer i kontexty implementují rozhraní \href{https://github.com/sybila/parasim/blob/master/core/src/main/java/org/sybila/parasim/core/api/Resolver.java}{\texttt{Resolver}},
které umožňuje na základě rozhraní a kvalifikátoru získat instance dané služby.
Pokud se v daném kontextu nenachází žádné rozšíření poskytující danou službu,
je zavolán rodičovský kontext. 

\begin{lstlisting}[label={code:resolve}, caption={Získání instance služby}]
Manager manager = ...
Enrichment enrichment = manager.resolve(
		Enrichment.class,
		Default.class);
\end{lstlisting}

Parasim používá ještě jeden jednodušší typ služeb. Tyto služby jsou dostupné pouze
pomocí manažera a nelze je od sebe odlišit pomocí kvalifikátoru. Manežer poskytuje
všechny dostupné implementace daného rozhraní v jedné kolekci. Tento typ služeb
primárně  slouží k ovlivňování cho\-vání rozšíření. Lze pomocí nich například
poslouchat událostem z logování.

\subsection{Rozšíření}

Parasim je schopen načíst rozšíření, která jsou v době vytvoření manažera na \texttt{Java class path}.
Všechna zatím použitá rozšíření byla přibalena do souboru JAR aplikace pomocí sestavovacího
nástroje Maven, a proto jsou k dispozici bez dalšího nasta\-vo\-vá\-ní. Rozšíření musí obsahovat
následující části:

\begin{itemize}
	\item	alespoń jednu třidu, která implementuje rozhraní \href{https://github.com/sybila/parasim/blob/master/core/src/main/java/org/sybila/parasim/core/spi/LoadableExtension.java}{LoadableExtension},
	\item	soubor \texttt{META-INF/services/org.sybila.parasim.core.spi.\\LoadableExtension} obsahující plné názvy všech tříd z daného roz\-ší\-ření implementující \href{https://github.com/sybila/parasim/blob/master/core/src/main/java/org/sybila/parasim/core/spi/LoadableExtension.java}{LoadableExtension}, jeden název na jednom řád\-ku. 
\end{itemize}

Rozhraní \href{https://github.com/sybila/parasim/blob/master/core/src/main/java/org/sybila/parasim/core/spi/LoadableExtension.java}{LoadableExtension}
obsahuje jedinou metodu, ve které je pře\-dán objekt pro registraci tříd rozšíření. Je možné
registrovat služby ne\-zá\-vi\-slé na kontextu a tzv. pozorovatele, což jsou třídy poskytující
služby a naslouchající událostem. Podrobnější příklad toho, jak rozšíření může vypadat,
se nachází v příloze \ref{appendix:extension}. Tato sekce obsahuje pouze vysvětlení
nej\-dů\-le\-ži\-těj\-ších konceptů.

Již zmínění pozorovatelé jsou schopni pomocí metod označených anotací \href{https://github.com/sybila/parasim/blob/master/core/src/main/java/org/sybila/parasim/core/annotation/Provide.java}{\texttt{Provide}}
poskytovat instance služeb. V deklaraci metody se mohou na\-chá\-zet parametry,
které je manažer jádra Parasimu schopen vyhodnotit a injektovat. Tyto parametry
lze označovat kvalifikátory. V případě, že není žádný kvalifikátor k dispozici,
je pro vyhodonocení hodnoty parametru použit kvalifikátor \href{https://github.com/sybila/parasim/blob/master/core/src/main/java/org/sybila/parasim/core/annotation/Default.java}{\texttt{Default}}.

\begin{lstlisting}[label={code:provide:required}, caption={První metoda poskytující službu \texttt{Functionality}}]
/**
 * selze, pokud neni v aktualnim kontextu k dispozici 
 * retezec pro dosazeni za parametr metody
 */
@Provide
public Functionality provideFunctionality(
		String required) {

	return new FunctionalityImpl(required);
}
\end{lstlisting}

V případě, že je manažer požádán, aby poskytl instanci služby, metoda poskytující danou službu je zavolána,
poskytnutý objekt propagován jako událost do všech dostupných rozšíření, a poté uložen pro případ,
že by byla služba požadována znovu. Výjimku tvoří poskytující metody, pro jejichž
návratový typ nelze vytvořit proxy\footnote{Typickým příkladem typů, pro které nelze vytvořit proxy, jsou finální třídy.}.
Takové poskytující metody jsou zavolány již během načítání rozšíření. Pokud v čase volání
poskytující metody nelze najít hodnoty pro všechny její parametry, dojde k vyhození výjimky a
pravděpodobně k pádu aplikace.

\begin{lstlisting}[label={code:provide:optional}, caption={Druhá metoda poskytující službu \texttt{Functionality}}]
/**
 * pokud neni v aktualnim kontextu k dispozici 
 * retezec pro dosazeni za parametr metody,
 * bude pouzita hodnota null
 */
@Provide
public Functionality provideFunctionality(
		@Inject(required=false) String optional) {

	return new FunctionalityImpl(optional);
}
\end{lstlisting}

Jestliže chce autor rozšíření záviset na jiných službách pouze volně, lze použít 
u parametrů poskytujících metod anotace \href{https://github.com/sybila/parasim/blob/master/core/src/main/java/org/sybila/parasim/core/annotation/Inject.java}{\texttt{Inject}} způsobem u\-ká\-za\-ným ve zdrojovém
kódu \ref{code:provide:optional}.

Druhým typem metod v pozorovatelských třídách jsou metody na\-slou\-cha\-jí\-cí událostem.
Aby bylo možné tyto metody rozlišit, je první parametr těchto metod je označen anotací \href{https://github.com/sybila/parasim/blob/master/core/src/main/java/org/sybila/parasim/core/annotation/Observes.java}{\texttt{Observes}}
a typ tohoto parametru určuje typ naslouchané události. Pro další parametry těchto metod platí
stejná pravidla jako pro parametry poskytujících metod. Pro vytvoření vlastních událostí
je k dispozici služba \href{https://github.com/sybila/parasim/blob/master/core/src/main/java/org/sybila/parasim/core/api/EventDispatcher.java}{\texttt{EventDispatcher}}.

\subsection{Konfigurace}

Jádro Parasimu nabízí jednoduchý způsob konfigurace jednotný pro všech\-na rozšíření
a tím vybízí autory rozšíření, aby ve svých rozšířeních umožnili konfigurací změnit
co nejvíce věcí. Zároveň je však kladen důraz na to, aby rozšíření bylo funkční 
samo o sobě bez toho, aby jej uživatel musel konfigurovat.

Pro autory rozšíření je k dispozici služba \href{https://github.com/sybila/parasim/blob/master/core/src/main/java/org/sybila/parasim/core/api/configuration/ParasimDescriptor.java}{\texttt{ParasimDescriptor}},
která je schopná na základě názvu rozšíření vrátit hodnoty konfiguračních pro\-měn\-ných
ve formě řetězců, dále služba \href{https://github.com/sybila/parasim/blob/master/core/src/main/java/org/sybila/parasim/core/api/configuration/ExtensionDescriptorMapper.java}{\texttt{ExtensionDescriptorMapper}} schopná namapovat hodnoty konfiguračních pro\-měn\-ných
namapovat do atributů u objektu v Javě. Mapování umí automaticky převést řetězce
do nejpoužívanějších datových typů.

Autor rozšíření tedy vytvoří konfigurační třídu s atributy, jejíchž názvy se shodují
s požadovanými názvy konfiguračních proměnných Výchozí hodnoty těchto atributů jsou
zároveň výchozími hodnotami pro konfiguraci rozšíření. Ukázka toho, jak může vypadat
zpřístupnění konfigurace v rozšíření je k dispozici v~příloze \ref{appendix:configuration}.

Při užívání aplikace lze konfiguraci lze změnit dvěma způsoby:

\begin{itemize}
	\item	Pomocí systémové proměnné \texttt{parasim.config.file} v Javě je mož\-né nastavit cestu k XML souboru.
			Výchozí cesta k tomuto souboru je nastavena na \uv{parasim.xml}. Tento soubor
			obsahuje pojmenované sekce a v těchto sekcích nastavení pro jednotlivé konfigurační
			pro\-měn\-né. Názvy proměnných se shodují s názvy atributů konfiguračních objektů.

	\item	Pro každou konfigurační proměnnou lze přepsat její hodnotu pomocí systémové
			proměnné v Javě, jejíž název je \texttt{parasim.<název roz\-ší\-ření>.<název konfugurační proměnné>}.
			Název rozšíření i konfigurační promměné se zde uvádí v tečkové notaci, například pro atribut \texttt{timeUnit}
			a rozšíření \texttt{example} se bude systémová proměnná nazývat \texttt{parasim.example.time.unit}.
\end{itemize}

Pokud je nutné nějakou část konfigurace zpřístupnit
změnám nejen při startu aplikace, ale i při jejím běhu skrze systémové proměnné, je potřeba mít na paměti
okamžik, ve kterém se vytváří konfigurační objekt. Jakmile je totiž konfigurační objekt vytvořen,
již nelze hodnoty konfigurace jemu náležící změnit. Po změně systémové proměnné,
je možné si znovunačtení konfiguračního objektu vynutit zničením příslušnoho kontextu
a vy\-tvo\-ře\-ním nového. Na to však není vhodný aplikační kontext, protože jeho
zničení prakticky znamená vypnutí aplikace.

\subsection{Obohacování}\label{section:enrichment}

Aby nebylo nutné si na všech místech předávat instanci manežera, případně instance všech
potřebných služeb, nabízí Parasim rozšiřitelný mechanismus obohacování objektů.
To je umožněno pomocí služby \href{https://github.com/sybila/parasim/blob/master/core/src/main/java/org/sybila/parasim/core/api/enrichment/Enrichment.java}{\texttt{Enrichment}}.
Tato služba spouští objety na kontextu nezávislých služeb \href{https://github.com/sybila/parasim/blob/master/core/src/main/java/org/sybila/parasim/core/spi/enrichment/Enricher.java}{\texttt{Enricher}}
na dané instanci, které ji dokáží různým způsobem vylepšit. 

Jádro Parasimu obsahuje dvě implementace rozhraní \href{https://github.com/sybila/parasim/blob/master/core/src/main/java/org/sybila/parasim/core/spi/enrichment/Enricher.java}{\texttt{Enricher}},
které umožní zpřístupnit poskytující metody a atributy podobně jako u rozšíření 
a které injektují služby do atributů. Za tímto účelem se používají již zmíněné anotace \href{https://github.com/sybila/parasim/blob/master/core/src/main/java/org/sybila/parasim/core/annotation/Provide.java}{\texttt{Provide}} a \href{https://github.com/sybila/parasim/blob/master/core/src/main/java/org/sybila/parasim/core/annotation/Inject.java}{\texttt{Inject}}.
Tyto anotace lze použít ve dvou nastaveních, v jednom nastavení bude při obohoacování
vyhozena výjimka, pokud poskytovaný případně injektovaný objekt není k dispozici.
Pokud je anotace použita s přiřazením \texttt{required=false}, výjimka se nevyhodí.

\subsection{Vzdálený přístup}\label{section:remote}

Další části práce ukážou, jak je možné Parasim použít k distribovanému počítání.
To je umožněno pomocí rozšíření pro vzdálený přístup, které u\-mož\-ňu\-je získat
některé služby nacházející se v aplikačním kontextu nějaké vzdáleného stroje.
Na tomto stroji je nejprve nutné tuto funkcionalitu aktivovat skrze rozhraní \href{https://github.com/sybila/parasim/blob/master/core/src/main/java/org/sybila/parasim/core/api/remote/Loader.java}{\texttt{Loader.java}},
jak je ukázáno ve zdrojovém kódu \ref{code:remote}. Ukázaný kód nastartuje server
pro práci s \textit{Remote Method Invocation}~\cite{grosso2001} a na tento server vystaví \href{https://github.com/sybila/parasim/blob/master/core/src/main/java/org/sybila/parasim/core/api/remote/Loader.java}{\texttt{Loader.java}}.

\begin{lstlisting}[label={code:remote}, caption={Spuštění serveru}]
manager.resolve(Loader.class, Default.class)
	   .load(Loader.class, Default.class);
\end{lstlisting}
Ostatní stroje jsou pak schopny vynutit si vystavení služeb z aplikačního kontextu
tohoto stroje na tento server. Aby bylo možné službu vystavit, je nutné, aby implementovala
rozhraní \texttt{java.rmi.Remote}. Jakmile je server na vzdáleném stroji aktivován,
je možné s ním komunikovat podobně jako ve zdrojovém kódu \ref{code:host:control}.

\begin{lstlisting}[label={code:host:control}, caption={Přístup ke vzdálenému serveru}]
HostControl control = new HostControlImpl(
		new URI("localhost"))
if (!control.isRunning(true)) {
	throw new IllegalStateException();
}
RemoteServis servis = control
		.lookup(RemoteServis.class, Default.class);
\end{lstlisting}

Jakákoliv interakce s takto získanými službami vyvolá síťovou komunikaci a veškerá logika
služby se vyhodnocuje na straně vzdáleného stroje.

\section{Výpočetní model}

Parasim obsahuje rozšíření pro snadnější provádění jistého výpočtu vý\-poč\-tu.
Základní jednotku výpočtu zde představuje instance. Během průběhu po\-čí\-tá\-ní lze
tuto instance rozdělit na více dalších a ty počítat nezávisle na sobě. Jednotlivé instance
se mohou dále dělit a vrací mezivýsledky, které je možné slučovat pomocí asociativní
a komutativní operace. Jak přesně se výpočetní instance dělí a jak jsou mezivýsledky
slučovány, určuje ten, kdo implementuje algoritmus.

Důležitým aspektem však je, že se vývojář implementující algoritmus nemusí starat
o způsob, jakým konkrétně budou výpočetní instance provedeny. To na druhou stranu vynucuje
některá omezení, která musí vývojář při implementaci výpočtu dodržet.  Parasim zatím nabízí
jednotné rozhraní pro počítání ve sdílené a~dís\-tri\-buo\-va\-né paměti.

\subsection{Reprezentace výsledku}

Třída reprezentující výsledek pro daný výpočet musí implementovat rozhraní \href{https://github.com/sybila/parasim/blob/master/model/core/src/main/java/org/sybila/parasim/model/Mergeable.java}{\texttt{Mergeable}}. Toto rozhraní si vynucuje krom uchování dat také definci komutativní
a asociativní operaci pro slučování mezivýsledků. Sou\-čas\-ně je nezbytné, aby třída
byla schopna serializace. Serializace je nezbytná pro výpočet v dis\-tri\-buo\-va\-né paměti,
kde se data po skončení mezivýpočtu po\-sí\-la\-jí mezi počítači po síti.

Z formálního hlediska již není kladen na datový typ výsledku žádný požadavek, nicméně
je třeba mít na paměti, že mezivýsledek je v případě distribuovaného počítání se výsledek
serializuje a je mezi stroji posílán po síti, tudíž velikost mezivýsledků může velkou měrou
ovlivnit výpočetní čas. Podobně i metoda pro slučování mezivýsledků by neměla být vý\-po\-čet\-ně
příliš náročná, protože spojování výsledků má zpravidla na starost jedno vlákno, respektive
jeden stroj.

\subsection{Reprezentace výpočtu}

Jakýkoliv výpočet je v Parasimu definován jako třída implementující rozhraní \href{https://github.com/sybila/parasim/blob/master/extensions/computation-lifecycle-api/src/main/java/org/sybila/parasim/computation/lifecycle/api/Computation.java}{\texttt{Computation}}. Ve třídě vývojář
implementuje algoritmus výpočtu za pomoci injektovaných služeb. V případě potřeby
lze určit způsob, jakým se má instance zachovat ve chvíli, kdy již není za potřebí.

V případě distribuovaného počítání se instance výpočtu může i ně\-ko\-lik\-rát posílat mezi stroji po sítí. Z toho důvodu
je nutné, aby i třída definující výpočet byla serializovatelná. To v praxi znamená, že musí
být seriazovatelné i hodnoty všech atributů, které nejsou injektovány. Bohužel chybová hláška z virtuálního stroje Javy
není v případě, že třída nesplňuje podmínky serializovatelnosti, přílíš popisná, takže
zejména tento typ chyb se velice špatně opravuje.

\subsection{Životní cyklus výpočtu}

Vstupním bodem pro spuštění výpočtu je služba \href{https://github.com/sybila/parasim/blob/master/extensions/computation-lifecycle-api/src/main/java/org/sybila/parasim/computation/lifecycle/api/ComputationContainer.java}{\texttt{ComputationContainer}}. Tento kontejner rozhoduje
rozhoduje, jaké použít prostředí pro výpočet na základě dostupné konfigurace a anotací.
Pomocí anotace \href{https://github.com/sybila/parasim/blob/master/extensions/computation-lifecycle-api/src/main/java/org/sybila/parasim/computation/lifecycle/api/annotations/RunWith.java}{\texttt{RunWith}} lze zvolit mezi prostředím se sdílenou nebo distribuovanou pamětí. Pokud tato anotace
není k dispozici, je použito výchozí prostředí, které lze předefinovat globálně v konfiguraci
Parasimu.

Jakmile kontejner provedl analýzu výpočetní instance, předává tuto instance dále
ke zvolenému výpočetnímu prostředí, ve kterém se instance spustí. Toto prostředí
vytvoří kontext výpočtu, který je sdílen napříč celým výpočtem na
jednom stroji. Pozor, v případě prostředí s distribuovanou pamětí, se tento kontext
vytvoří na každém použitém stroji. Tento kontext je vhodný pro bezstavové služby
a cache, do které si jednotlivé výpočetní instance mohou ukládat informace, jejichž
získání je vý\-po\-čet\-ně náročné. Například u algoritmu pro analýzu dynamických systémů
se v tomto kontextu nacházejí již nasimulované trajektorie chování nebo služba schopná spočítat
pro trajektorii chování její lokální robustnost.

Pro každou výpočetní instanci je dále vytvořen další kontext, který již není sdílený
s žádnou další instancí a jehož délka trvání může být  mnohem kratší než u výpočetního kontextu.
Tento kontext je vhodný pro služby, které by v případě sdílení napříč více vláken musely
být synchronizovány. V Parasimu se v tomto kontextu nachází služba pro simulaci trajektorie
chování, která pro tento účel používá proces nástroje GNU Octave \cite{eaton2008}. Není
žádoucí, aby jeden proces nástroje GNU Octave byl sdílen více vlákny. 

Aby se usnadnilo použití služeb z různých rozšíření, je před samotným spuštěním výpočetní
instance použito obohacování zmíněné v sekci \ref{section:enrichment}. Až v tomto bodě
se instance výpočtu spustí. Do této chvíle se v kontejneru nenachází žádná souběžnost
a vše je prováděno pouze sekvenčně. V těle výpočtu však mohou být vytvářeny další instance výpočtu,
a ty pak pomocí asynchronního volání služby \href{https://github.com/sybila/parasim/blob/master/extensions/computation-lifecycle-api/src/main/java/org/sybila/parasim/computation/lifecycle/api/Emitter.java}{\texttt{Emitter}} emitovány do výpočetního prostředí, které je zodpovědné za jejich spuštění.
Průběh počítání je tudíž podobný jako u modelu Fork/Join \cite{lea2000} uvedeného v Javě verze 7.

Jakmile je výpočetní instance s výpočtem hotová, je vrácen výsledek a zničí se
kontext spojený s touto instancí. Kontejner sbírá mezivýsledky a postupně je slučuje
s dalšími. Až svůj výpočet ukončí poslední instance, zničí se výpočetní kontext na
všech použitých strojích. Konečný výsledek, případně i aktuální podoba sloučených
mezivýsledků je k dispozici v podobě objektu \href{https://github.com/sybila/parasim/blob/master/extensions/computation-lifecycle-api/src/main/java/org/sybila/parasim/computation/lifecycle/api/Future.java}{\texttt{Future}}.

\begin{figure}[H]
\label{figure:computation:shared:memory}
\begin{center}
\resizebox{0.9\textwidth}{!}{
\begin{tikzpicture}  	[node distance=1.8cm,start chain=going below]
\tikzset{>=stealth',
	event/.style={
    	rectangle, 
    	rounded corners, 
    	fill=green!20,
    	draw=black, very thick,
    	text width=9em, 
    	minimum height=3em, 
    	text centered},
	class/.style={
    	rectangle, 
    	rounded corners, 
    	fill=black!10,
    	draw=black, very thick,
    	text width=9em, 
    	minimum height=3em, 
    	text centered},
	package/.style={
		rectangle,
		draw=black!50, dashed,
		rounded corners,
		inner sep=0.3cm},
 	scope/.style={decorate},
	code/.style={
		rectangle,
		draw=black!50, dashed,
		rounded corners,
    	text width=12em, 
    	minimum height=3em, 
    	text centered},
	dot/.style={
		circle,
		draw=black,
		fill=black,
		inner sep=0pt,
		minimum size=1pt
	}
}
	\node[class] (computation) {Computation comp\\instance výpočtu};
	\node[code, below of=computation] (compute) {ComputationContainer\\.compute(comp)};
	\node[code, below of=compute] (execute) {Executor.submit(comp)};
	\node[event, below of=execute] (before-computation) {BEFORE\\kontext výpočtu};
	\node[event, below of=before-computation] (before-computation-instance-1) {BEFORE\\kontext\\instance výpočtu};
	\node[code, below of=before-computation-instance-1] (computation-enrichment-1) {obohacení};
	\node[code, below of=computation-enrichment-1] (thread-pool-executor-execute-1) {výpočet};
	\node[class, below of=thread-pool-executor-execute-1] (result-partial-1) {mezivýsledek};
	\node[code, below of=result-partial-1] (computation-instance-destroy-1) {zničení};
	\node[event, below of=computation-instance-destroy-1] (after-computation-instance-1) {AFTER\\kontext\\instance výpočtu};
	\node[event, below of=after-computation-instance-1] (after-computation) {AFTER\\kontext výpočtu};
	\node[class, below of=after-computation] (result) {výsledek};

	\node[class, left of=before-computation-instance-1, node distance=6.5cm] (computation-copy-2) {instance výpočtu};
	\node[event, below of=computation-copy-2] (before-computation-instance-2) {BEFORE\\kontext\\instance výpočtu};
	\node[code, below of=before-computation-instance-2] (computation-enrichment-2) {obohacení};
	\node[code, below of=computation-enrichment-2] (thread-pool-executor-execute-2) {výpočet};
	\node[class, below of=thread-pool-executor-execute-2] (result-partial-2) {mezivýsledek};
	\node[code, below of=result-partial-2] (computation-instance-destroy-2) {zničení};
	\node[event, below of=computation-instance-destroy-2] (after-computation-instance-2) {AFTER\\kontext\\instance výpočtu};

	\node[class, right of=before-computation-instance-1, node distance=6.5cm] (computation-copy-3) {instance výpočtu};
	\node[event, below of=computation-copy-3] (before-computation-instance-3) {BEFORE\\kontext\\instance výpočtu};
	\node[code, below of=before-computation-instance-3] (computation-enrichment-3) {obohacení};
	\node[code, below of=computation-enrichment-3] (thread-pool-executor-execute-3) {výpočet};
	\node[class, below of=thread-pool-executor-execute-3] (result-partial-3) {mezivýsledek};
	\node[code, below of=result-partial-3] (computation-instance-destroy-3) {zničení};
	\node[event, below of=computation-instance-destroy-3] (after-computation-instance-3) {AFTER\\kontext\\instance výpočtu};

	\node[dot, left of=thread-pool-executor-execute-1, node distance=3cm] (line11) {};
	\node[dot, left of=before-computation-instance-1, node distance=3cm, label=above left:emit] (line12) {};
	\node[dot, right of=thread-pool-executor-execute-1, node distance=3cm] (line21) {};
	\node[dot, right of=before-computation-instance-1, node distance=3cm, label=above right:emit] (line22) {};

	\begin{scope}[->, thick]
		\draw (computation) -> (compute);
		\draw (compute) -> (execute);
		\draw (execute) -> (before-computation);
		\draw (after-computation) -> (result);

		\draw (before-computation) -> (before-computation-instance-1);

		\draw (before-computation-instance-1) -> (computation-enrichment-1);
		\draw (before-computation-instance-2) -> (computation-enrichment-2);
		\draw (before-computation-instance-3) -> (computation-enrichment-3);

		\draw (computation-copy-2) -> (before-computation-instance-2);
		\draw (computation-copy-3) -> (before-computation-instance-3);

		\draw (computation-enrichment-1) -> (thread-pool-executor-execute-1);
		\draw (computation-enrichment-2) -> (thread-pool-executor-execute-2);
		\draw (computation-enrichment-3) -> (thread-pool-executor-execute-3);

		\draw (thread-pool-executor-execute-1) -> (result-partial-1);
		\draw (thread-pool-executor-execute-2) -> (result-partial-2);
		\draw (thread-pool-executor-execute-3) -> (result-partial-3);

		\draw (result-partial-1) -> (computation-instance-destroy-1);
		\draw (result-partial-2) -> (computation-instance-destroy-2);
		\draw (result-partial-3) -> (computation-instance-destroy-3);

		\draw (computation-instance-destroy-1) -> (after-computation-instance-1);
		\draw (computation-instance-destroy-2) -> (after-computation-instance-2);
		\draw (computation-instance-destroy-3) -> (after-computation-instance-3);

		\draw (after-computation-instance-1) -> (after-computation);
		\draw (after-computation-instance-2) -> (after-computation);
		\draw (after-computation-instance-3) -> (after-computation);

		\draw (line12) -> (computation-copy-2);

		\draw (line22) -> (computation-copy-3);
	\end{scope}
	\begin{scope}[thick]
		\draw (thread-pool-executor-execute-1) -- (line21);
		\draw (line21) -- (line22);
		\draw (thread-pool-executor-execute-1) -- (line11);
		\draw (line11) -- (line12);
	\end{scope}
\end{tikzpicture}}
\end{center}
\caption{Schéma průběhu výpočtu ve sdílené paměti. Zelené obdelníky představují události,
šedými obdelníkymi objekty a čerchovaně ohraničené jsou části Java kódu.}
\end{figure}

\subsection{Výpočetní prostředí}

Jak již bylo řečeno, Parasim nabízí dva typy prostředí, ve kterém lze výpočet spustit.
Výchozí prostředím, které bude použito bez jakéhokoliv nastavování, je počítání ve sdílené paměti.
Toto prostředí využívá standardního způsobu nepřímé práce s vlákny ve virtuálním stroji Javy
pomocí třídy implementující \texttt{java.util.concurrent.ExecutorService}.
Instance té\-to třídy má k dispozici určité množství vláken, kterému se posílají objekty
implementující \texttt{java.util.concurrent.Callable}, vlákna pak tyto objekty zpracovávají.

Takto odeslané objekty \texttt{java.util.concurrent.Callable} již nelze získat zpět,
což způsobuje problém. Aby bylo možné model pro sdílenou paměť převyužít i pro prestředí s distribuovanou pamětí,
je zapotřebí u\-mož\-nit balancovat výpočet napříč více stroji. V případě, že některý ze strojů není vytížený, kontejner zajistí,
aby se na tento stroj přesunula instance výpočtu ze stroje více vytíženého. Z toho důvodu
obsahuje Parasim mezivrstvu (třída \href{https://github.com/sybila/parasim/blob/master/extensions/computation-lifecycle-impl/src/main/java/org/sybila/parasim/computation/lifecycle/impl/common/Mucker.java}{\texttt{Mucker}}). Instance výpočtu jsou nejprve uloženy do fronty a objektu \texttt{java.util.concurrent.ExecutorService} jsou dále posílány teprve z této fronty 
 tak, aby dostupná vlánka byla co nejvíce vytížená.
Pokud dojde k přesouvání instancí výpočtu mezi stroji, jsou k tomu využity instance z fronty dané stroje.

Centrálním bodem výpočtu je třída implementující \href{https://github.com/sybila/parasim/blob/master/extensions/computation-lifecycle-api/src/main/java/org/sybila/parasim/computation/lifecycle/api/MutableStatus.java}{\texttt{MutableStatus}}. Skrze tento bod se dorozumívají všechny objekty
starající se o řádný průběh počítání. Využívají k tomu systém událostí a naslouchajících objektů
implementujících rozhraní \href{https://github.com/sybila/parasim/blob/master/extensions/computation-lifecycle-api/src/main/java/org/sybila/parasim/computation/lifecycle/api/ProgressListener.java}{\texttt{ProgressListener}}. Použité události jsou:

\begin{description}

    \item[emitted] 		Událost nastane ve chvíli, kdy je emitován nový výpočet pomocí služby
						\href{https://github.com/sybila/parasim/blob/master/extensions/computation-lifecycle-api/src/main/java/org/sybila/parasim/computation/lifecycle/api/Emitter.java}{\texttt{Emitter}}.

    \item[computing] 	Událost nastane ve chvíli, kdy je instance výpočtu z fronty vý\-poč\-tů poslána
						vláknů k provedení a ještě před tím, než je pro instanci vytvořen příslušný kontext.

    \item[done] 		Událost nastane ve chvíli, kdy se instance výpočtu se svým výpočtem
						hotova. Stane se tak ještě před tím, než je vrácen výsledek,
						zničen kontext a zničena instance.

	\item[finished]		Událost nastane ve chvíli, kdy již není k dispozici žádná instance výpočtu.
	
	\item[balanced]		Událost nastane ve chvíli, kdy dojde k přesunutí instance výpočtu z jednoho stroje na jiný,
						a to pouze na stroji, kam byla instance pře\-su\-nu\-ta.
\end{description}

V případě prostředí s distribuovanou pamětí je průběh výpočtu slo\-ži\-těj\-ší. Než začne samotný výpočet,
je nutné na strojích, které chceme použít, spustit podporu pro vzdálený přístup
zmíněný v sekci \ref{section:remote}. Na každém z~těch\-to strojů je výpočet zaregistrován
pod náhodným identifikátorem a vytvoří se výpočetní kontext. V celém mechanismu se tedy  nacházejí dva typy strojů.
Stroj, ze kterého byl výpočet iniciován (\textit{master}), a stroje, mezi které je distribuovaná práce (\textit{slave}).

Narozdíl od prostředí se sdílenou pamětí se zde nacházejí syn\-chro\-ni\-zač\-ní body dva.
Na každém stroji se nachází lokální status, který skrze události volá příslušné objekty
nutné k obsluze lokálního výpočtu. Tanto lokální status ještě před tím, než zavolá naslouchající objekty implementující rozhraní 
\href{https://github.com/sybila/parasim/blob/master/extensions/computation-lifecycle-api/src/main/java/org/sybila/parasim/computation/lifecycle/api/ProgressListener.java}{\texttt{ProgressListener}},
kontaktuje s příslušnou událostí vzdálený status, synchronizační bod pro všechny stroje participující na da\-ném vý\-poč\-tu.
Tanto vzdálený status se nachází na stroji, ze kterého byl výpočet iniciován, a zařizuje přesun
instancí výpočtu mezi méně a více vytíženými stroji.

\begin{figure}[h!]
\label{figure:computation:events}
\begin{center}
\resizebox{0.95\textwidth}{!}{
\begin{tikzpicture}  	[node distance=1.8cm,start chain=going below]
\tikzset{>=stealth',
	event/.style={
    	rectangle, 
    	rounded corners, 
    	fill=green!20,
    	draw=black, very thick,
    	text width=9em, 
    	minimum height=3em, 
    	text centered},
	class/.style={
    	rectangle, 
    	rounded corners, 
    	fill=black!10,
    	draw=black, very thick,
    	text width=9em, 
    	minimum height=3em, 
    	text centered},
	package/.style={
		rectangle,
		draw=black!50, dashed,
		rounded corners,
		inner sep=0.3cm},
 	scope/.style={decorate},
	code/.style={
		rectangle,
    	text width=9em, 
		draw=black!50, dashed,
		rounded corners,
    	minimum height=3em, 
    	text centered}
}
	\node[class] (computation-instance) {instance výpočtu};
	\node[event, below left of=computation-instance, node distance=3.5cm] (computing) {COMPUTING};
	\node[event, left of=computing, node distance=4.9cm] (emitted) {EMITTED};
	\node[event, below right of=computation-instance, node distance=3.5cm] (balanced) {BALANCED};
	\node[event, right of=balanced, node distance=4.9cm] (done) {DONE};
	\node[event, below of=computation-instance, node distance=7cm] (finished) {FINISHED};

	\node[code, below of=computing] (executor-service) {ExecutorService};

	\node[package, below of=emitted, node distance=4cm] (emitted-listeners) {
		\begin{tikzpicture}
			\begin{scope}[node distance=1.8cm, solid]
				\node[class] (distributed-mucker-1) {DistributedMemory\\Mucker};
				\node[class, below of=distributed-mucker-1] (mucker-1) {Mucker};
				\node[class, below of=mucker-1] (offerer) {Offerer};
			\end{scope}
		\end{tikzpicture}
	};

	\node[package, below of=done, node distance=4cm] (done-listeners) {
		\begin{tikzpicture}
			\begin{scope}[node distance=1.8cm, solid]
				\node[class] (distributed-mucker-2) {DistributedMemory\\Mucker};
				\node[class, below of=distributed-mucker-2] (mucker-2) {Mucker};
				\node[class, below of=mucker-2] (computation-future) {ComputationFuture};
			\end{scope}
		\end{tikzpicture}
	};

	\begin{scope}[->, thick]
		\draw (computation-instance) -> (computing);
		\draw (computation-instance) -> (emitted);
		\draw (computation-instance) -> (balanced);
		\draw (computation-instance) -> (done);
		\draw (emitted) -> (emitted-listeners);
		\draw (computing) -> (executor-service);
		\draw (done) -> (done-listeners);
		\draw (computation-instance) -> (finished);
	\end{scope}
\end{tikzpicture}}
\end{center}
\caption{Schéma událostí, které nastávají během výpočtu, a objektů, které na tyto události reagují.}
\end{figure}

\subsection{Možná nastavení výpočtu}

Jednotlivé výpočty se od sebe mohou velkou mírou lišit, a je proto vhodné, aby celý mechanismus
počítání v Parasimu byl co nejsnáze nastavitelný. Jednou z věcí, která může výpočet ovlivnit, je pořadí,
v jakém se jednotlivé instance budou počítat. Dalším důležitým faktorem může být vývěr instance,
která se přesune při balancování mezi méně a více vytíženým strojem. Tyto dva atributy se zdají být
kritické pro implementaci uvažovaného algoritmu pro analýzu dynamického systému, protože obsahuje
výpočetně velice náročnou numerickou simulaci, kvůli které je vhodné používat cache.

Ukládání trajektorií chování je paměťově poměrně náročné, a proto se po nějaké době
musí trajektorie z paměti vymazat. Pokud se výpočetní instance pro\-vá\-dě\-jí v pořadí daném
iterací zahušťování, je možné některé trajektorie vymazat z paměti dříve. V případě virtuálního
stroje Javy se navíc menší využití paměti projeví i na výkonu \textit{Garbage collector} \cite{printezis2000}. 

Pokud přesováme jednu výpočetní instanci z jednoho stroje na druhé, je třeba si uvědomit,
že každý z těchto strojů má již vybudovanou svoji cache. Cílem balancování je přesouvat
nejlépe takovou výpočetní instanci, která bude pracovat s trajektoriemi nenacházejícími
se v cache zdrojového stroje. Výpočet takové instance bude na cílovém stroji trvat nejvíce tak dlouho jako
na stroji zdrojovém.

Parasim k tomuto účelu nabízí rozhraní \href{https://github.com/sybila/parasim/blob/master/extensions/computation-lifecycle-api/src/main/java/org/sybila/parasim/computation/lifecycle/api/Selector.java}{\texttt{Selector}}, které z dané kolekce vybere jeden její prvek. Pomocí anotace \href{https://github.com/sybila/parasim/blob/master/extensions/computation-lifecycle-api/src/main/java/org/sybila/parasim/computation/lifecycle/api/annotations/RunWith.java}{\texttt{RunWith}} lze nastavit třídu implementující toto rozhraní
pro účely balancování nebo výběru instance pro další počítání. U těchto objektů je opět použito mechanismu obohacování ze sekce \ref{section:enrichment},
a tudíž mají k dispozici všechny služby z výpočetního kontextu na daném stroji.
Vývojář však musí mít na paměti, že se tyto objekty budou volat velice často, 
a proto by výběr z kolekce neměl být příliš výpočetně náročný.


\section{Dostupná rozšíření}

Na závěr této kapitoly je vhodné uvést rozšíření, která jsou zatím pro účely aplikace
implementující algoritmus pro analýzu dynamických systémů dostupná. Zde je uveden pouze výčet
s krátkým popisem. Možné způsoby konfigurace jsou k dispozici v příloze \ref{appendix:extensions}.

\begin{description}

	\item[computation-simulation] 		Rozšíření poskytuje numerickou simulaci. Pro se\-kven\-ci bodů vrátí
										sekvenci trajektorií chování. Umožňuje rovněž již nasimulovanou trajektorii
										chování prodloužit o zvolený čas. Současná implementace použivá volně dostupný
										nástroj GNU Octave~\cite{eaton2008}. 

    \item[computation-cycledetection]	Rozšíření pro detekci cyklu na trajektorii cho\-vá\-ní. V současné
										implementaci nástroje toto rozšížení není použito. Nicméně ostatní
										rozšíření jsou schopna s výsledky analýzy pracovat. Od detekování
										cyklu se upustilo z důvodu použití konečných intervalů u
										temporálních operátorů ve zkoumaných vlastnostech.

    \item[computation-density]			Rozšíření dokáže určit vzdálenosi mezi hlavní trajektorií a sousedními trajektoriemi chování a na
										základě této vzdá\-le\-nos\-ti zahustit prostor iniciálních podmínek. Obsahuje cache pro
										již použité trajektorie. Z důvodu nepřesnosti při počítání s reálnými čísly v Javě se
										se jako klíč v této paměti používají souřadnice obsahující zlomky, jejichž jmenovatel
										je mocninou čísla 2. Tento zlomek určuje v jaké části prostoru iniciálních podmínek,
										který je po celou dobu výpočtu stejně velký, se trajektorie chování nachází.

	\item[computation-lifecycle]		Rozšíření poskytující již dříve zmíněný výpočetní model použitý v Parasimu. 

\end{description}

V současné době Parasim obsahuje více rozšíření, než je zde zmíněno. Nicméně tato rozšíření se vztahují ne k samotnému
algoritmu, ale spíše k~samotné aplikaci. Tato aplikace umí spravovat projekty s modely a nastavením
pro analýzu. Také umožňuje výsledky analýzy zobrazovat a ukládat do souboru.

\chapter{Evaluace}\label{chapter:evaluation}

Předešlé kapitoly představily upravený algoritmus pro analýzu dy\-na\-mic\-kých systémů
a jeho implementaci v rámci nástroje Parasim. Následující kapitola prezentuje průběh 
analýzy nad vybranými modely v různě nastaveném prostředí. Nejdůležitějším měřítkem
je čas, který je nutný k analýze, avšak budou prezentovýny i data jiného typu.


Nejprve si ukážeme, jakým způsobem byly vybrány analyzované modely. Následuje
popis těchto modelů a prezentace naměřených dat. Na\-mě\-ře\-ná data obsahují především
výpočetní čas, který je nutný k provedení experimentu v různých konfiguracích výpočetního
prostředí. Na závěr se nachází interpretace měření.

\section{Možné parametry}

Průběh výpočtu popisovaného algoritmu pro analýzu dynamický systémů lze rozdělit do tří hlavních částí:

\begin{enumerate}
	\item	zahuštění inicialníhi prostoru požadovanými body, \label{item:density}
	\item	získání trajektorií chování,\label{item:simulation}
	\item	ověření dané vlastnosti nad nasimulovanými trajektoriemi. 
\end{enumerate}

Ukazuje se, že bod \ref{item:simulation} je výpočetně nejnáročnější, tvoří více než 90~\% výpočetního času a přímo závisí na 
množství  trajektorií chování nutných k~analýze úzce. Toto množství samozřejmě plyne z modelu, vlastnosti, perturbací.
V algoritmu se nové trajektorie vytváří pomocí bodu \ref{item:density}, který je parametrizován maximálním
počtem provedených iterací zahuštění.

Dále je pravděpodobné, že z hlediska roz\-dě\-le\-ní práce v paralelním a distribuovaném
prostředí bude rozdíl mezi analýzou menšího počtu dlouhých trajektorií, jejichž simulace
je náročná, a mnoha krátkých trajektorií, které lze nasimulovat rychle. Délka trajektorie
závisí na intervalech nacházejících se v ověřované formuli. 

Z pohledu vstupních dat experimentu se tedy nabízí následující parametry:

\begin{enumerate}
	\item	maximální počet iterací zahuštění,
	\item	počet perturbovaných parametrů a proměnných,
	\item	minimální délka trajektorie nutná k ověření dané vlastnosti,
	\item	vlastnosti modelu ovlivňující zahušťování.
\end{enumerate}

\section{Použité modely}

Pomocí aplikace Parasim bylo provedeno několik experimentů pokrývající výše uvedené vlastnosti, tyto experimenty
zahrnují tři modely. U dvou z nich jsou dostupné tři konfigurace. Jednotlivé konfigurace se od sebe liší 
jen v~několika detailech, ale i tyto detaily mají obrovský vliv na průběh analýzy, což bude ukázáno později
v sekci \ref{section:measurement}.

\subsection{Predátor a kořist}

Prvním analyzovaným modelem je systém predátora a kořisti již dříve popsaný v~sek\-ci \ref{section:lotkav}. Model
obsahuje dvě proměnné, proměnnou $x$ pro kořist a pro\-měn\-nou $y$ pro predátora. Analýza používá
perturbaci iniciálních hodnot  těchto proměnných v prostoru $[1, 100] \times [1, 100]$ a zkoumá oscilaci pomocí vlastnosti \ref{eq:model:stl:lotkav},
která vznikla úpravou předpisu \ref{eq:stl:lotkav:oscil}. Délka oscilace je pak dána konkrétní konfigurací,
která byla k analýze použita. Hodnoty parametrů jsou nastaveny na $\alpha = \frac{1}{10}$,  $\beta = \frac{2}{10000}$, $\gamma = \frac{1}{10}$ a $\delta = \frac{2}{10000}$.

\begin{align}
\label{eq:model:stl:lotkav}
\mathcal{F}_{[0, 100]}\mathcal{G}_{[0, \textrm{délka oscilace}]}\mathcal{F}_{[0, 50]}(x \geq 40 \wedge \mathcal{F}_{[0, 40]}x \leq 40)
\end{align}

Analýza byla provedena ve třech různých nastaveních, jejichž jména nesou pro další účely prefix \texttt{lotkav}
a jejichž konkrétní popis je k dispozici v tabulce \ref{tabular:benchmark:models}.

K zahuštění dochází zejména v extrémních hodnotách pro\-měn\-ných $x$ a $y$, kde systém vůbec neosciluje,
a ve středních hodnotách, ve kterých systém sice osciluje, ale okolo jiné hodnoty, než je požadováno v uvažované vlastnosti.
Charakter zahuštění je vidět na obrázku \ref{image:lotkav:result}. 

\begin{figure}[h!]
\begin{center}
\subfigure[analýza na vybraných parametrech]{
	\label{image:lotkav:result}
	\includegraphics[width=0.48\textwidth]{../images/generated/lotkav-analysis.pdf}
}
\subfigure[vývoj systému v čase]{
	\label{image:lorenz84:result}
	\includegraphics[width=0.48\textwidth]{../images/generated/lotkav-timeserie.pdf}
}
\caption{Pradátor a kořist}
\end{center}
\end{figure}


\subsection{Lorenzův atraktor}

Dalším analyzovaným modelem je Lorenzův atraktor \cite{lorenz2010} skládající se
ze tří stavových proměnných, který je odvozen ze zjednodušených rovnic prou\-dě\-ní
vzduchu v atmosféře, k. Ačkoliv tento model vykazuje v určitých hodnotách chaotické chování,
je možné u něj nalézt na jedné z proměnných oscilaci.

\begin{figure}[h!]
\begin{center}
\subfigure[analýza na vybraných parametrech]{
	\label{image:lorenz84:result}
	\includegraphics[width=0.48\textwidth]{../images/generated/lorenz84-analysis.pdf}
}
\subfigure[vývoj systému v čase]{
	\label{image:lorenz84:result}
	\includegraphics[width=0.48\textwidth]{../images/generated/lorenz84-timeserie.pdf}
}
\caption{Lorenzův systém}
\end{center}
\end{figure}

Konkrétní podoba modelu je dána rovnicemi \ref{eq:model:ode:lorenz84}. Uvažovány jsou perturbace
parametrů $F$ a $G$ v prostoru iniciálních podmínek $[\frac{1}{10}, 2] \times [\frac{1}{10}, 2]$.
Hodnoty fixních parametrů jsou $a = \frac{1}{4}$ a $b = 4$ a iniciální hodnoty $x_1 = x_2 = x_3 = 0$.

\begin{align}\label{eq:model:ode:lorenz84}
\begin{array}{ll}
\frac{dx_0}{dt} = a \cdot F - x_1^2 - x_2^2 - a \cdot x_0				\\
\frac{dx_1}{dt} = x_0 \cdot x_1 + G - b \cdot x_0 \cdot x_2 - x_1			\\
\frac{dx_2}{dt} = b \cdot x_0 \cdot x_1 + x_0 \cdot x_2 - x_2				\\
\end{array}
\end{align}

Analyzovanou vlastností je opět oscilace, tudíž lze s malými úpravami použít opět
vlastnost \ref{eq:stl:lotkav:oscil}. Pozorovanou proměnnou je $x_1$, která
osciluje okolo $0$. 

\begin{align}
\label{eq:model:stl:lorenz84}
\mathcal{F}_{[0, 5]}\mathcal{G}_{[0, \textrm{délka oscilace}]}\mathcal{F}_{[0, 5]}(x \geq \frac{1}{100} \wedge \mathcal{F}_{[0, 5]}x \leq -\frac{1}{100})
\end{align}

Lorenzův atraktor je velikostí modelu podobný predátoru a kořisti, nic\-mé\-ně v průběhu
analýzy nedochází k tak rovnoměrnému zahušťování prostoru iniciálních podmínek.
Popis konfigurací modelu použitých v rámci evaluace  se nacházejí v tabulce \ref{tabular:benchmark:models},
jejich názvy začínají prefixem \texttt{lorenz84}.

\subsection{Oscilace vápníku}

Posledním uvažovaným experimentem je model oscilace vápníku \cite{meyer1991} po\-chá\-ze\-jí\-cí z~volně dostupné databáze biologických modelů \cite{biomodels}.
Model lze z~této databáze stáhnout za použití identifikátoru \texttt{BIOMD0000000224}.
Hodnoty fix\-ních parametrů a výchozí koncentrace látek jsou získané ze sta\-že\-né\-ho SBML souboru,
který byl bez jakýchkoliv změn importován do aplikace Parasim.

\begin{align}\label{eq:model:ode:lorenz84}
\begin{array}{ll}
\frac{dCaI}{dt} &= (1 - g) \cdot \bigg(\frac{A \cdot \big(\frac{IP_3}{2}\big)^4}{\big(\frac{IP_3}{2} + k_1\big)^4} + L\bigg) \cdot CaS - \frac{B \cdot \big(\frac{CaI}{100}\big) ^ 2}{\big(\frac{CaI}{100}\big) ^ 2 + k_2 ^ 2}\\
\frac{dIP3}{dt} &= C \cdot (1 - \frac{k_3}{CaI \cdot \frac{1}{100} + k_3} \cdot \frac{1}{1 + R} ) - D \cdot \frac{IP_3}{2}	\\
\frac{dCaS}{dt} &= \frac{B \cdot \big(\frac{CaI}{100}\big) ^ 2}{\big(\frac{CaI}{100}\big) ^ 2 + k_2 ^ 2} - (1 - g) \cdot \bigg(\frac{A \cdot \big(\frac{IP_3}{2}\big)^4}{\big(\frac{IP_3}{2} + k_1\big)^4} + L\bigg) \cdot CaS	\\
\frac{dg}{dt}   &= E \cdot \big(\frac{CaI}{100}\big) ^ 4 \cdot (1 - g) - F
\end{array}
\end{align}


\begin{figure}[h!]
\begin{center}
\subfigure[analýza na vybraných parametrech]{
	\includegraphics[width=0.48\textwidth]{../images/generated/meyer91-analysis.pdf}
	\label{fig:meyer91:result}
}
\subfigure[vývoj systému v čase]{
	\includegraphics[width=0.48\textwidth]{../images/generated/meyer91-timeserie.pdf}
}
\caption{Oscilace vápníku}
\end{center}
\end{figure}

Model obsahuje velké množství parametrů a čtyři stavové proměnné $CaI$, $IP_3$, $CaS$ a $g$, u tří z nich je požadována oscilace.
Jak je vidět na obrázku \ref{fig:meyer91:result}, zahušťování
během analýzy navíc probíhá velice rov\-no\-měr\-ně. Během analýzy byly perturbovány čtyři parametry
$k_1$, $k_2$, $C$ a $D$ v prostoru i\-ni\-ciál\-ních podmínek $[\frac{1}{10}, \frac{9}{10}] \times [\frac{1}{10}, \frac{2}{10}] \times [\frac{6}{10}, \frac{16}{10}] \times [\frac{15}{10}, \frac{25}{10}]$.

Požadovanou vlastností je opět oscilace. I když sledujeme oscilaci více proměnných,
oscilují tyto proměnné synchronně, a proto lze opět použít upravenou vlastnost
ze schématu \ref{eq:stl:lotkav:oscil}:


\begin{align}
\begin{array}{ll}
\label{eq:model:stl:meyer91}
\mathcal{F}_{[0, 5]}\mathcal{G}_{[0, \textrm{délka oscilace}]}\mathcal{F}_{[0, 50]}(\varphi_1  \wedge \mathcal{F}_{[20, 50]}\varphi_2),\\\\
\textrm{kde~~}
\varphi_1 \equiv CaI \geq 100 \wedge IP_3 \geq \frac{1}{2} \wedge g \geq \frac{9}{10}	\\
~~~~~~~~~\varphi_2 \equiv CaI  \leq 15 \wedge IP_3 \leq \frac{2}{10} \wedge g \leq \frac{4}{10}
\end{array}
\end{align}

\begin{table}[h!]
\centering
\begin{tabular}{ l c c }
\toprule
	~ 									& počet iterací zahušťování		& délka oscilace	\\
\midrule
	\texttt{lotkav-common}				& 8								& 300				\\
	\texttt{lotkav-iterations}			& 10							& 300				\\
	\texttt{lotkav-long-property}		& 8								& 6000				\\
	\texttt{lorenz84-common}			& 8								& 15				\\
	\texttt{lorenz84-iterations}		& 10							& 15				\\
	\texttt{lorenz84-long-property}		& 8								& 150				\\
	\texttt{meyer91-common}				& 6								& 150				\\
\bottomrule
\end{tabular}
\caption{Jednotlivé konfigurace experimentů, které byly pro účely evaluace použity. Vstupní soubory k experimentům jsou k dispozici v adresáři \texttt{benchmark/experiments} repozitáře aplikace Parasim.}
\label{tabular:benchmark:models}
\end{table}

\section{Měření}\label{section:measurement}

Experimenty byly spuštěny v prostředí se sdílenou i distribuovanou pamětí s různým
počtem dostupných procesorových jader, respektive strojů. Pro každé výpočetní vlákno
Parasimu běží dva procesy nástroje Octave, který obstarává numerickou simulace. Při měření
ve sdílené paměti byla aplikace parasim spouštěna s $n$ výpočetními vlákny na $2n$
procesorových jádrech.

Na strojích pro prostředí s distribuovanou pamětí byla aplikace Parasim spouštěna se dvěma
výpočetními vlákno a konfigurace těchto počítačů se lišily od počítače použitého pro prostředí
s pamětí sdílenou. v obou prostředích měl vyrtuální stroj Javy k dispozici 4 GB paměti.

\begin{description}
	\item[sdílená paměť: ]
		 64 jader, 2.27 GHz; paměť 450 GB; Red Hat Enterprise Linux Server release 6.4; Java 1.7.0\_13-b20 (64 bit); Octave 3.4.3
	\item[distribuovaná paměť]
		4 jádra, 2.0 GHz; paměť 16 GB; Java 1.7.0\_13-b20 (64 bit); Octave 3.6.4
\end{description}

Kompletní výsledky měření pro všechny uvažované konfigurace modelů jsou k dispozici
v příloze \ref{appendix:measurement}. Ke každé konfiguraci jsou k dispozici čtyři grafy:

\begin{itemize}
	\item	čas nutný k provedení analýzy v prostředí se sdílenou a distribuovanou pamětí,
			pro názornost je v grafu červeně uvedeno ideální zrychlení,
	\item	počet simulovaných trajektorií chování během analýzy, pro názornost je v grafu
			uveden počet trajektorií, který by byl použit při naivním zahušťování odpovídajícímu
			dané iteraci,
	\item	množství neplatných přístupů do paměti v závislosti na počtu strojů v distribuovaném prostředí,
			tato pomáhá předejít po\-čí\-tá\-ní duplicitních trajektorií.
\end{itemize}

Experiment s modelem v konfiguraci \texttt{meyer91-common} byl spouštěn pouze v distribuovaném prostředí
a to minimálně na dvou strojích kvůli velkému množství trajektorií, se kterými je během analýzy nutno
pracovat. Toto množství trajektorií je dáno počtem perturbovaných parametetrů.

\section{Interpretace měření}

\subsection{Prostředí s distribuovanou pamětí}

Při pohledu na grafy času nutného k provedění analýzy plyne, že implementace
prostředí s distribuovanou pamětí pro jeden až šestnáct strojů šká\-lu\-je poměrně. Výjimku tvoří
Lorenzův systém s maximálním množ\-stvím iterací nastaveným na 8. Při analýze tohoto modelu
se prostor iniciálních podmínek zahušťuje pouze v poměrně malém regionu a z toho důvodu se při tak
malém množství iterací nestihne vytvořit dostatečné množství trajektorií tak, aby se plně
vytížily všechny stroje. 

\begin{figure}[h!]
\begin{center}
\subfigure[\texttt{lorenz84-common}]{
	\includegraphics[width=0.48\textwidth]{../images/generated/lorenz84-common-balancer.pdf}
}
\subfigure[\texttt{lorenz84-iterations}]{
	\includegraphics[width=0.48\textwidth]{../images/generated/lorenz84-iterations-balancer.pdf}
}
\caption{Množství strojů v čase, kterým náleží neprázdná fronta výpočtů. Každý tik odpovídá jednomu provedenému balancování.}
\end{center}
\end{figure}

Rozdělování práce mezi výpočetní jednotky vede k možnosti du\-pli\-cit\-ních výpočtů,
a proto Parasim používá cache, ve které ukládá již dříve analyzované trajektorie.
V prostředí s více stroji si každý počítač udržuje svou verzi paměti s již
analyzovanými trajektoriemi. Jelikož je tato cache lokální, je možné, že při balancování
výpočtu napříč stroji dojde k tomu, že se některe trajektorie analyzují vícekrát.

V současné implementaci se dává přednost rychlému výběru instance výpočtu pro
přesun kvůli balancování, protože se na výběr čeká v kritické sekci na hlavním
počítači (master), skrze který je celý výpočet řízen. Sofistikovanější, ale časově
náročnější výběr zvýšit čekací čas na komunikaci mezi výpočetními stroji a hlavním strojem.
Ukazuje se, že podíl duplicitní práce u náročnějších modelů tvoří pouze zlomek výpočtu,
na druhou stranu u menších analýz se může množství duplicitně počítaných trajektorií
vyšplhat i na 20 \%. Je na zvážení, zda by se nevyplatilo používat časově náročnější
balancovací metodu, která by se u větších a na komunikaci náročnějších analýz vypínala.

\subsection{Prostředí se sdílenou pamětí}

Ve sdílené paměti odpadá problém s duplicitními výpočty, protože cache s~již napočítanými
trajektoriemi je sdílená napříč všemi výpočetními vlákny. Při menším množství jader se
projevují nedostatky Parasimu. Na grafech výpočetního času například u experimentu \texttt{lotkav-long-property}
je vidět, že při přechodu z~jednoho na dvě vlákna dochází k většímu než dvojnásobnému zrychlení. To je pravděpodobně způsobeno
množstvím vlá\-ken vytvořených aplikací Parasim a knihovnou pro komunikaci s aplikací Octave,
kterých je více, než dokáže efektivně obsloužit daný počet jader. Tento \uv{over\-head} se
při větším jader projevuje méně.

\subsection{Pozorování nezávislá na prostředí}

Při měření se projevilo i několik faktorů nezávislých na prostředí, ve kterém se
analýza prováděla.


\chapter{Závěr}\label{chapter:conclusion}

Cílem diplomové práce bylo implementovat v prostředí s distribuovanou pamětí
algoritmus pro analýzu dynamických systémů zadaných pomocí
soustavy diferenciálních rovnic vzhledem k vlastnostem definovaným
v~tem\-po\-rál\-ní logice signálů. Implementovaný algoritmus byl
převzat z~diplomové práce Svena Dražana~\cite{drazan2011} a rozšířen o lokální robustnost, jejíž použití umožňuje
efektivnější pokrytí prostoru iniciálních pod\-mí\-nek. 
Implementace takto upraveného algoritmu vyústila ve vytvoření volně dostupného nástroje Parasim\footnote{\url{https://github.com/sybila/parasim/wiki}}.

Na základě identifikovaných parametrů, jež mohou ovlivnit výpočet, byly vybrány modely,
nad kterými se poté spustila analýza v různě nastaveném výpočetním prostředí.
Z průběhu této analýzy a jejích výsledků vyplývá, že až na výjimky implementace
ve sdílené i distribuované paměti škáluje. Pro dosažení tohoto výsledku nebylo zapotřebí
žádných sofistikovanějších metod pro balancování výpočtu. Z naměřených
dat také plyne, že použití robustnosti a prezentovaného způsobu pokrytí prostoru iniciálních podmínek má svůj význam,
protože ve většině případů se oproti naivním metodám ušetřilo velké množství práce.

Princip analýzy je ve velké míře podobný tomu, který se používá v~nás\-tro\-ji Breach~\cite{donze2010breach}.
Nicméně na rozdíl od něj je Parasim založen na volně dostupných technologiích, je snadno rozšiřitelný
a podporuje distribuované počítání. Také způsob pokrytí prostoru iniciálních podmínek
a použití robustnosti se liší.

Byl naimplementován vlastní výpočetní model, který abstrahuje od vý\-po\-čet\-ního
prostředí a umožňuje snadno přecházet z prostředí se sdí\-le\-nou pamětí do prostředí z pamětí distribuovanou
a naopak. Díky tomuto a\-spek\-tu a modulární architektuře je Parasim otevřen pro rozšiřování a
implementaci dalších algoritmů.

Ve stávající implementaci je pro numerickou simulaci použit Octave, který se
však ukázal v mnoha případech nevyhovující. Simulace trvá příliš dlouho a často ani
neposkytne požadované výsledky. Použití jiného nástroje pro řešení systému
diferenciálních rovnic nebylo předmětem této práce, nicméně do budoucna se jeví jako vhodné
místo nástroje Octave použít například nástroj COPASI~\cite{hoops2006}, který je přímo určen
pro analýzu biologických modelů.


\bibliographystyle{czechiso}
\bibliography{literature}

\appendix
\chapter{Způsob použití aplikace Parasim}\label{appendix:usage}

Pro použití aplikace Parasim je nutné mít na svém počítačí nainstalovanou Javu verze 7
a Octave ve verzi 3.6.x. Poslední verzi  Parasimu je možné si stáhnout ze
stránek projektu \footnote{\url{https://github.com/sybila/parasim/wiki}} v podobě
archivu JAR. Na těchto stránkách se rovněž nachází podrobný návod, jak aplikaci Parasim používat.
Zde je uvedeno jen několik příkladů. Stažený archiv lze spustit standardní cestou:

\begin{lstlisting}[style=Bash]
java -jar parasim-2.0.0.Final-dist;
\end{lstlisting}

V případě, že je Parasim spuštěn bez jakýchkoliv dalších parametrů, nastartuje se grafické
uživatelské rozhraní pro správu experimentů. Pro získání nápovědy k jednotlivým argumentům
je nutné použít přepínač \texttt{-h}:

\begin{lstlisting}[style=Bash]
java -jar parasim-2.0.0.Final-dist -h;
\end{lstlisting}

K nastartování serveru pro distribuované počítání lze použít přepínač \texttt{-s},
je rovněž vhodné nastavit adresu stroje v síti:

\begin{lstlisting}[style=Bash]
java -Dparasim.remote.host=pheme01 -jar \
	parasim-2.0.0.Final-dist -s;
\end{lstlisting}

Na stránkách projektu je rovněž k dispozici GIT repozitář, který krom zdrojových kódů obsahuje
i projekty experimentů připravené ke spuštění. Tento repozitář je možné si stáhnout
za použití následujícího příkazu:

\begin{lstlisting}[style=Bash]
git clone git://github.com/sybila/parasim.git;
\end{lstlisting}

V ukázkových projektech je k dispozici i model predátora a kořisti, jehož analýzu
lze po načtení spustit z grafického uživatelského rozhraní nebo z~příkazové řádky:

\begin{lstlisting}[style=Bash]
java -jar parasim-2.0.0.Final-dist\
	-e experiments/lotkav/oscil.experiment.properties;
\end{lstlisting}

\chapter{Ukázka rozšíření pro Parasim}\label{appendix:extension}

\begin{lstlisting}[caption={META-INF/services/org.sybila.parasim.core.spi.\\LoadableExtension}]
org.sybila.parasim.myextension.MyExtension
\end{lstlisting}


\begin{lstlisting}[caption={org/sybila/parasim/myextension/MyExtension.java}]
public class MyExtension implements LoadableExtension {
    public void register(ExtensionBuilder builder) {
        builder.extension(FunctionalityRegistrar.class);
    }
}
\end{lstlisting}

\begin{lstlisting}[caption={org/sybila/parasim/myextension/\\FunctionalityRegistrar.java}]
@ApplicationScope
public class FunctionalityRegistrar {

	private Context context;
	
	/**
     * Poksytni sluzbu Functionality pod kvalifikatorem
     * @Default
     */
	@Default
	@Provide
    public Functionality provideFunctionality(...) {
        return new FunctionalityImpl();
    }

	/**
     * Pokud je vytvoren kontext, oznam ostatnim
     * pozorovatelum udalost 'Hello World!'.
     */
	public void hello(
			@Observes Before event,
			EventDispatcher eventDispatcher) {
	
		if (before.getLoad().equals(OwnScope.class)) {
			eventDispatcher.fire("Hello World!");
		}
	}

	/**
     * Kontext je vytvoren v momente,
     * kdy je poskutnuta sluzna Functionality
	 */
	public void startOwnContext(
			@Observes Functionality event,
			ContextFactory contextFactory) {
		
		context = contextFactory
				.context(OwnScope.class);
	}

	/**
     * Manazer se vypina, a proto je potreba
     * vytvoreny kontext znicit.
     */
	public void stopOwnContext(
			@Observes ManagerStopping event) {

		if (context != null) {
			context.destroy();
		}
	}

	public interface Functiononality {
	}

	public static class FunctiononalityImpl
			implements Functiononality {
	}

	@Scope
	@Documented
	@Retention(RetentionPolicy.RUNTIME)
	@Target(ElementType.TYPE)
	public @interface OwnScope {
	}
}
\end{lstlisting}

\chapter{Ukázka konfigurace pro Parasim}\label{appendix:configuration}

\begin{lstlisting}[label={code:config:bean}, caption={Konfigurační třída}]
public class ExampleConfig {
	private long timeoutAmount = 10;
	private TimeUnit timeoutUnit = TimeUnit.SECONDS;

	public long timeoutAmount() {
		return timeoutAmount;
	}

	public TimeUnit timeoutUnit() {
		return timeoutUnit;
	}
}
\end{lstlisting}

\begin{lstlisting}[label={code:config:provide}, caption={Metoda poskytující konfiguraci}]
public ExampleConfig provideConfig(
		ParasimDescriptor descriptor,
		ExtensionDescriptorMapper mapper) {

	ExtensionDescriptor extDescriptor = descriptor
			.getExtensionDescriptor("example");
	ExampleConfig c = new ExampleConfig();
	if (extDescriptor != null) {
		mapper.map(extDescriptor, c);
	}
	return c;
}
\end{lstlisting}

\begin{lstlisting}[language=xml, label={code:config:xml}, caption={parasim.xml}]
<parasim>
	<extension qualifier="example">
		<property name="timeoutAmount">2</property>
		<property name="timeoutUnit">days</property>
	</extension>
</parasim>
\end{lstlisting}

\chapter{Konfigurace dostupných rozšíření}\label{appendix:extensions}

Následuje seznam rozšíření a jejich dostupných konfiguračních proměnných. V závorkách
u názvů rozšíření je klíč, pod kterým je nutné k nim přistupovat v konfiguraci.
V závorkách u názvů konfiguračních proměnných se nachází výchozí hodnota.

\section{Aplikace (\texttt{application})}

\begin{description}
	\item[warmupComputationSize] (30) \\
		Minimální počet hlavních trajektorií, se kterými pracuje jedna nezbytková výpočetní instance v zahřívací fázi výpočtu.
	\item[warmupBranchFactor] (4) \\
		Maximální počet výpočetních instancí, které se emitují z jedné instance v zahřívací fázi výpočtu.
	\item[warmupIterationLimit] (2) \\
		Číslo poslední iterace zahušťování, která ještě patří do zahřívací fáze výpočtu.
	\item[computationSize] (60) \\
		Minimální počet hlavních trajektorií, se kterými pracuje jedna nezbytková výpočetní instance.
	\item[branchFactor] (2) \\
		Maximální počet výpočetních instancí, které se emitují z~jedné instance.
	\item[showingRobustnessComputation] (\texttt{true}) \\
		Zapíná a vypíná možnost zobrazit okno s výpočtem robustnosti pro jeden iniciální bod po kliknutí na vizualizaci výsledků analýzy.
\end{description}

\section{Logování (\texttt{logging})}

\begin{description}
	\item[configFile] ~\\
		Konfigurační pro logovací knihovnu \texttt{log4j}. Pokud není specifikován použije se se konfigurační soubor \href{https://github.com/sybila/parasim/blob/2.0.0.Final/core/src/main/resources/org/sybila/parasim/log4j/log4j.properties}{\texttt{log4j.properties}} distribuovaný společně s rozšířením.
	\item[level] (\texttt{info}) \\
		Úroveň logovacích zpráv, které se zobrazí v konzoli. Možné hodnoty jsou -- \texttt{all}, \texttt{debug}, \texttt{error}, \texttt{fatal}, \texttt{info}, \texttt{off}, \texttt{trace} a \texttt{warn}.
\end{description}

\section{Vzdálená správa (\texttt{remote})}

\begin{description}
	\item[host] (\texttt{InetAddress.getLocalHost().getHostAddress()}) \\
		Adresa stroje v síti.
\end{description}

\section{Výpočetní model (\texttt{computation-lifecycle})}

\begin{description}
	\item[numberOfThreads] (počet dostupných procesorových jader) \\
		Počet použitých výpočetních vláken ve sdílené paměti.
	\item[nodeThreshold] ($\frac{3}{2} \times $ počet dostupných procesorových jader) \\
		Minimální počet současně počítaných výpočtů ve službě \texttt{ExecutorService}, který se výpočetní kontejner snaží udržet.
	\item[balancerMultiplier] ($\frac{3}{2}$) \\
		K balancování dochází pouze pokud $b > \texttt{balancerMultiplier} \cdot i$, kde $b$ je počet výpočetních instancí na nejvíce zatíženém stroji a $i$ počet instancí na nejméně vytíženém.
	\item[balancerBusyBound] (1) \\
		K balancování dochází pouze pokud je počet výpočetních instancí na nejvíce zatíženém stroji vyšší než toto číslo.
	\item[balancerIdleBound] (1) \\
		K balancování dochází pouze pokud je počet výpočetních instancí na nejméně zatíženém stroji nižší než toto číslo.
	\item[nodes] ~\\
		Stroje použité pro distribuované počítání. Na těchto strojích musí být Parasim spuštěn s přepínačem \texttt{-s}.
	\item[defaultExecutor] (org.sybila.parasim.computation.lifecycle.api.SharedMemoryExecutor) \\
		Výchozí výpočetní prostředí.
\end{description}

\section{Numerická simulace (\texttt{simulation})}

\begin{description}
	\item[lsodeIntegrationMethod] (\texttt{nonstiff}) \\
		Integrační metoda pro LSODE, k dispozici jsou -- \texttt{adams}, \texttt{nonstiff}, \texttt{bdf} a \texttt{stiff}.
	\item[odepkgFunction] ~\\
		Pokud je nastaveno, je použita integrační metoda z balíku \texttt{odepkg}, dostupné hodnoty jsou
		 -- \texttt{ode5r}, \texttt{ode78}, \texttt{odebda}, \texttt{odebdi}, \texttt{odekdi}, \texttt{oders} a \texttt{odesx}.
\end{description}

\section{Detekce cyklu (\texttt{cycledetection})}

\begin{description}
	\item[relativeTolerance] ($\frac{1}{100}$) \\
		Relativní tolerance pro ověření ekvivalence dvou bodů pro účely detekce cyklu.
\end{description}

\chapter{Výsledky měření}\label{appendix:measurement}


\begin{figure}[h!]
\begin{center}
\subfigure[čas nutný k provedení analýzy v~dis\-tri\-bu\-o\-va\-ném prostředí]{
	\includegraphics[width=0.48\textwidth]{../images/generated/lotkav-common-dist-time.pdf}
}
\subfigure[čas nutný k provedení analýzy v~pro\-stře\-dí se sdílenou pamětí]{
	\includegraphics[width=0.48\textwidth]{../images/generated/lotkav-common-shared-time.pdf}
}
\subfigure[počet simulovaných trajektorií chování v~porovnání s naivním zahušťováním odpovídajícímu danému počtu iterací]{
	\includegraphics[width=0.48\textwidth]{../images/generated/lotkav-common-shared-iterations-summary.pdf}
}
\subfigure[počet neplatných přístupů do paměti, kde se ukládají již analyzivané trajektorie chování]{
	\includegraphics[width=0.48\textwidth]{../images/generated/lotkav-common-dist-cache.pdf}
}
\end{center}
\caption{\texttt{lotkav-common}}
\end{figure}

\begin{figure}[h!]
\begin{center}
\subfigure[čas nutný k provedení analýzy v~dis\-tri\-bu\-o\-va\-ném prostředí]{
	\includegraphics[width=0.48\textwidth]{../images/generated/lotkav-iterations-dist-time.pdf}
}
\subfigure[čas nutný k provedení analýzy v~pro\-stře\-dí se sdílenou pamětí]{
	\includegraphics[width=0.48\textwidth]{../images/generated/lotkav-iterations-shared-time.pdf}
}
\subfigure[počet simulovaných trajektorií chování v~porovnání s naivním zahušťováním odpovídajícímu danému počtu iterací]{
	\includegraphics[width=0.48\textwidth]{../images/generated/lotkav-iterations-shared-iterations-summary.pdf}
}
\subfigure[počet neplatných přístupů do paměti, kde se ukládají již analyzivané trajektorie chování]{
	\includegraphics[width=0.48\textwidth]{../images/generated/lotkav-iterations-dist-cache.pdf}
}
\end{center}
\caption{\texttt{lotkav-iterations}}
\end{figure}

\begin{figure}[h!]
\begin{center}
\subfigure[čas nutný k provedení analýzy v~dis\-tri\-bu\-o\-va\-ném prostředí]{
	\includegraphics[width=0.48\textwidth]{../images/generated/lotkav-long-property-dist-time.pdf}
}
\subfigure[čas nutný k provedení analýzy v~pro\-stře\-dí se sdílenou pamětí]{
	\includegraphics[width=0.48\textwidth]{../images/generated/lotkav-long-property-shared-time.pdf}
}
\subfigure[počet simulovaných trajektorií chování v~porovnání s naivním zahušťováním odpovídajícímu danému počtu iterací]{
	\includegraphics[width=0.48\textwidth]{../images/generated/lotkav-long-property-shared-iterations-summary.pdf}
}
\subfigure[počet neplatných přístupů do paměti, kde se ukládají již analyzivané trajektorie chování]{
	\includegraphics[width=0.48\textwidth]{../images/generated/lotkav-long-property-dist-cache.pdf}
}
\end{center}
\caption{\texttt{lotkav-long-property}}
\end{figure}

\begin{figure}[h!]
\begin{center}
\subfigure[čas nutný k provedení analýzy v~dis\-tri\-bu\-o\-va\-ném prostředí]{
	\includegraphics[width=0.48\textwidth]{../images/generated/lorenz84-common-dist-time.pdf}
}
\subfigure[čas nutný k provedení analýzy v~pro\-stře\-dí se sdílenou pamětí]{
	\includegraphics[width=0.48\textwidth]{../images/generated/lorenz84-common-shared-time.pdf}
}
\subfigure[počet simulovaných trajektorií chování v~porovnání s naivním zahušťováním odpovídajícímu danému počtu iterací]{
	\includegraphics[width=0.48\textwidth]{../images/generated/lorenz84-common-shared-iterations-summary.pdf}
}
\subfigure[počet neplatných přístupů do paměti, kde se ukládají již analyzivané trajektorie chování]{
	\includegraphics[width=0.48\textwidth]{../images/generated/lorenz84-common-dist-cache.pdf}
}
\end{center}
\caption{\texttt{lorenz84-common}}
\end{figure}

\begin{figure}[h!]
\begin{center}
\subfigure[čas nutný k provedení analýzy v~dis\-tri\-bu\-o\-va\-ném prostředí]{
	\includegraphics[width=0.48\textwidth]{../images/generated/lorenz84-iterations-dist-time.pdf}
}
\subfigure[čas nutný k provedení analýzy v~pro\-stře\-dí se sdílenou pamětí]{
	\includegraphics[width=0.48\textwidth]{../images/generated/lorenz84-iterations-shared-time.pdf}
}
\subfigure[počet simulovaných trajektorií chování v~porovnání s naivním zahušťováním odpovídajícímu danému počtu iterací]{
	\includegraphics[width=0.48\textwidth]{../images/generated/lorenz84-iterations-shared-iterations-summary.pdf}
}
\subfigure[počet neplatných přístupů do paměti, kde se ukládají již analyzivané trajektorie chování]{
	\includegraphics[width=0.48\textwidth]{../images/generated/lorenz84-iterations-dist-cache.pdf}
}
\end{center}
\caption{\texttt{lorenz84-iterations}}
\end{figure}

\begin{figure}[h!]
\begin{center}
\subfigure[čas nutný k provedení analýzy v~dis\-tri\-bu\-o\-va\-ném prostředí]{
	\includegraphics[width=0.48\textwidth]{../images/generated/lorenz84-long-property-dist-time.pdf}
}
\subfigure[čas nutný k provedení analýzy v~pro\-stře\-dí se sdílenou pamětí]{
	\includegraphics[width=0.48\textwidth]{../images/generated/lorenz84-long-property-shared-time.pdf}
}
\subfigure[počet simulovaných trajektorií chování v~porovnání s naivním zahušťováním odpovídajícímu danému počtu iterací]{
	\includegraphics[width=0.48\textwidth]{../images/generated/lorenz84-long-property-shared-iterations-summary.pdf}
}
\subfigure[počet neplatných přístupů do paměti, kde se ukládají již analyzivané trajektorie chování]{
	\includegraphics[width=0.48\textwidth]{../images/generated/lorenz84-long-property-dist-cache.pdf}
}
\end{center}
\caption{\texttt{lorenz84-long-property}}
\end{figure}


\begin{figure}[h!]
\begin{center}
\subfigure[čas nutný k provedení analýzy v~dis\-tri\-bu\-o\-va\-ném prostředí]{
	\includegraphics[width=0.48\textwidth]{../images/generated/meyer91-common-dist-time.pdf}
}
\subfigure[počet simulovaných trajektorií chování v~porovnání s naivním zahušťováním odpovídajícímu danému počtu iterací]{
	\includegraphics[width=0.48\textwidth]{../images/generated/meyer91-common-dist-iterations-summary.pdf}
}
\subfigure[počet neplatných přístupů do paměti, kde se ukládají již analyzivané trajektorie chování]{
	\includegraphics[width=0.48\textwidth]{../images/generated/meyer91-common-dist-cache.pdf}
}
\end{center}
\caption{\texttt{meyer91-common}}
\end{figure}


\end{document}

