\documentclass[xcolor=svgnames]{beamer}
\usepackage[utf8]{inputenc}
\usepackage[czech]{babel}
\usepackage{tikz}
\usetikzlibrary{calc,trees,positioning,arrows,chains,shapes.geometric,%
    decorations.pathreplacing,decorations.pathmorphing,shapes,%
    matrix,shapes.symbols, decorations.markings}

\usetheme{Parasim}

\title[Parasim]{Analýza robustnosti spojitých dynamických systémů v distribuovaném prostředí}
\author{Jan Papoušek}
\institute{Masaryk University Brno}
\date{\today}

\begin{document}
% --------------------------- SLIDE --------------------------------------------
\frame[plain]{\titlepage}
% ------------------------------------------------------------------------------
% --------------------------- SLIDE --------------------------------------------
\begin{frame}
	\frametitle{Dynamické systémy}
	\begin{center}
		soustava diferenciálních rovnic

		\bigskip
		\bigskip

		{\Huge$\frac{\textbf{y}}{dt} = f(\textbf{y})$}

		\bigskip
		$t \geq t_o, \textbf{y}(t_o) = y_0$
	\end{center}
\end{frame}
% ------------------------------------------------------------------------------
% --------------------------- SLIDE --------------------------------------------
\begin{frame}
	\frametitle{Chování dynamického systému}
	\begin{center}
		numerická simulace

		\bigskip
		\bigskip

		{\large$t_{n+1} = t_n + h$}

		\bigskip
		{\Huge$\textbf{y}_n \sim \textbf{y}(t_n)$}

		\bigskip
		\bigskip
		(trejektorie chování, signál)
	\end{center}
\end{frame}
% ------------------------------------------------------------------------------

% --------------------------- SLIDE --------------------------------------------
\begin{frame}
	\frametitle{Chování dynamického systému}
	\begin{center}
		\includegraphics<1>[width=0.8\textwidth]{../images/generated/piecewise-constant-a.pdf}
		\includegraphics<2>[width=0.8\textwidth]{../images/generated/piecewise-constant-b.pdf}
	\end{center}
\end{frame}
% ------------------------------------------------------------------------------
% --------------------------- SLIDE --------------------------------------------
\begin{frame}
	\frametitle{Vlastnost -- STL}
	\begin{center}
		{\Huge$U = \{\mu_1, \mu_2, \ldots, \mu_k\}$}

		\bigskip
		\bigskip
		{\Large $\mu_i: \mathbb{R}^n \rightarrow \{T, F\}$}

		\medskip
		{$(y_i \geq k)$}

		\bigskip
		\bigskip

		\visible<2->{
			atomické propozice

			\bigskip

			{\Huge$P = \{1, \ldots, k\}$}
		}
	\end{center}
\end{frame}
% ------------------------------------------------------------------------------
% --------------------------- SLIDE --------------------------------------------
\begin{frame}
	\frametitle{Vlastnost -- STL}

	\begin{center}
		{\Large $\varphi := T~|~p~|~\neg\varphi~|~\varphi_1 \wedge \varphi_2~|~\varphi_1\mathbf{U}_{[a,b]}\varphi_2$}

		\visible<2->{
			\bigskip
			\hrule
			\bigskip

			\begin{align*}\begin{array}{ll}
				(\mathbf{y}, t)	\models p
						&\Longleftrightarrow \mu_p(\mathbf{y}(t)) = T			\\
				(\mathbf{y}, t) \models \neg \varphi
						&\Longleftrightarrow (\mathbf{y}, t) \not\models \varphi	\\
				(\mathbf{y}, t) \models \varphi_1 \wedge \varphi_2
						&\Longleftrightarrow (\mathbf{y}, t)  \models \varphi_1 \textrm{ a současně } (\mathbf{y}, t) \models \varphi_2 \\
				(\mathbf{y}, t) \models \varphi_1 \mathbf{U}_{[a,b]} \varphi_2
						&\Longleftrightarrow \exists t' \in [t+a, t+b] . (\mathbf{y}, t') \models \varphi_2	\\
						&~~~~~~\textrm{ a současně } \forall t'' \in [t, t'] . (\mathbf{y}, t'') \models \varphi_1
			\end{array}\end{align*}
		}

	\end{center}
\end{frame}
% ------------------------------------------------------------------------------
% --------------------------- SLIDE --------------------------------------------
\begin{frame}
	\frametitle{Vlastnost -- STL}

	\begin{center}
		{\Large $\varphi_1 \mathbf{U}_{[a,b]} \varphi_2$}

		\bigskip

		\begin{tikzpicture}
			\tikzstyle{main}=[circle,fill=black,minimum size=5pt,inner sep=0pt, draw=black]
			\node[main,label=below:t]		(t) {};
			\node[main, right of=t, label=below:t', node distance=5.5cm]		(t2){};
			\node[main, right of=t2, label=below:t+b, node distance=1.5cm]		(tb){};
			\node[main, minimum size=1pt, right of=tb, node distance=3cm]		(end){};

			\node at (3,0.3) {$\varphi_1$};

			\begin{scope}[-, shorten <=1pt]
				\draw[green, very thick]	(t) -> (t2) node[above] {$\varphi_1 \wedge \varphi_2$};
				\draw[black]		(t2) -> (end);
			\end{scope}
			\node[main, right of=t, label=below:t+a, node distance=4cm]		(ta){};
		\end{tikzpicture}

	\end{center}
\end{frame}
% ------------------------------------------------------------------------------
% --------------------------- SLIDE --------------------------------------------
\begin{frame}
	\frametitle{Parametry}
	$$\frac{dY_1}{dt}=\nu Y_1-\alpha Y_1 \cdot Y_2	\qquad	\frac{dY_2}{dt}=\alpha Y_1 \cdot Y_2-\mu Y_2$$
	\begin{description}
		\item[proměnné] ($Y_1$, $Y_2$) -- počáteční hodnoty
		\item[koeficienty] ($\mu$, $\nu$, $\alpha$)
	\end{description}
\end{frame}
% ------------------------------------------------------------------------------
% --------------------------- SLIDE --------------------------------------------
\begin{frame}
	\frametitle{Cíl analýzy}
	\begin{center}
		\visible<2>{\alert{spojitá doména hodnot}}
		\includegraphics[width=.7\textwidth]{../images/generated/analysis-goal.pdf}
	\end{center}
\end{frame}
% ------------------------------------------------------------------------------
% --------------------------- SLIDE --------------------------------------------
\begin{frame}
	\frametitle{Lokální robustnost}
	\begin{columns}[2]
		\column{.35\textwidth}
		\begin{center}
			{\large$U' = \{\mu'_1, \mu'_2, \ldots, \mu'_k\}$}

			\medskip
			{\large $\mu'_i: \mathbb{R}^n \rightarrow \mathbb{R}$}

			\medskip
			{$(y_i - k)$}
		\end{center}

		\column{.65\textwidth}
		\begin{center}
			\includegraphics<1>[width=0.95\textwidth]{../images/generated/tube.pdf}
		\end{center}
	\end{columns}
\end{frame}
% ------------------------------------------------------------------------------
% --------------------------- SLIDE --------------------------------------------
\begin{frame}
	\frametitle{Průběh analýzy}

	\begin{center}
		\only<1>{
		\begin{tikzpicture}[node distance=2cm]
			\draw[thin,color=black,dotted] (0,0) grid (4,4);
			\tikzstyle{main}=[circle,fill=black,minimum size=5pt,inner sep=0pt, draw=black]
			\tikzstyle{neigh}=[circle,fill=white,minimum size=5pt,inner sep=0pt,draw=black]
			\node[main, label=left:D] 				(D) {};
			\node[neigh,right of=D, label=below:S] 	(S) {};
			\node[main, right of=S, label=right:E] 	(E) {};
			\node[neigh, above of=D, label=left:Q] 	(Q) {};
			\node[main, right of=Q, label=above right:C] 	(C) {};
			\node[neigh, right of=C, label=right:R]	(R) {};
			\node[main, above of=Q, label=left:A] 	(A) {};
			\node[neigh, right of=A, label=above:P] (P) {};
			\node[main, right of=P, label=right:B] 	(B) {};

			\begin{scope}[->, thick, shorten <=1pt]
				\draw	(A) -> (P);
				\draw	(A) -> (Q);
				\draw	(B) -> (P);
				\draw	(B) -> (R);
				\draw	(C) -> (P);
				\draw	(C) -> (Q);
				\draw	(C) -> (R);
				\draw[label={[above]{$f(x)=x^2$}}]	(C) -> (S);
				\draw	(D) -> (Q);
				\draw	(D) -> (S);
				\draw	(E) -> (S);
				\draw	(E) -> (R);
			\end{scope}
		\end{tikzpicture}
		}
		\only<2>{
		\begin{tikzpicture}[node distance=2cm]
			\draw[thin,color=black,dotted] (0,0) grid (4,4);
			\tikzstyle{main}=[circle,fill=black,minimum size=5pt,inner sep=0pt, draw=black]
			\tikzstyle{neigh}=[circle,fill=white,minimum size=5pt,inner sep=0pt,draw=black]
			\node[main, label=left:D] 				(D) {};
			\node[neigh,right of=D, label=below:S] 	(S) {};
			\node[main, right of=S, label=right:E] 	(E) {};
			\node[neigh, above of=D, label=left:Q] 	(Q) {};
			\node[neigh, right of=Q, label=above right:C] 	(C) {};
			\node[neigh, right of=C, label=right:R]	(R) {};
			\node[main, above of=Q, label=left:A] 	(A) {};
			\node[neigh, right of=A, label=above:P] (P) {};
			\node[main, right of=P, label=right:B] 	(B) {};

			\node[main, label=above right:F] (F) at (2,1) {};
			\node[neigh, label=above left:M] (M) at (1,1) {};
			\node[neigh, label=above right:N] (N) at (3,1) {};
			\begin{scope}[->, thick, shorten <=1pt]
				\draw	(F) -> (C);
				\draw	(F) -> (N);
				\draw	(F) -> (M);
				\draw	(F) -> (S);
			\end{scope}
		\end{tikzpicture}
		}
	\end{center}
\end{frame}
% ------------------------------------------------------------------------------
% --------------------------- SLIDE --------------------------------------------
\begin{frame}
	\frametitle{Ukázka}
	\begin{center}
		\includegraphics[width=0.8\textwidth]{../images/generated/lotkav-analysis7.pdf}
	\end{center}
\end{frame}
% ------------------------------------------------------------------------------
% --------------------------- SLIDE --------------------------------------------
\begin{frame}
	\frametitle{Implementace}
	\begin{center}
		{\Huge \alert{parasim}}

		\bigskip
		nástroj pro paralelní simulaci a verifikaci
	\end{center}
\end{frame}
% ------------------------------------------------------------------------------
% --------------------------- SLIDE --------------------------------------------
\begin{frame}
	\frametitle{Systém rozšíření}
	\begin{itemize}
		\item<1->	\alert{jádro}
				\begin{itemize}
					\item	životní cyklus
					\item	kontexty + služby
					\item	jednotná konfigurace
					\item	obohacování objektů
					\item	vzdálený přístup
				\end{itemize}

		\item<1->	\alert{rozšíření}
				\begin{itemize}
					\item	numerická simulace
					\item	výpočet robustnosti
					\item	zahušťování
					\item	výpočetní model
					\item	vizualizace
					\item	uživatelské rozhraní
					\item	a další \ldots
				\end{itemize}
	\end{itemize}
\end{frame}
% ------------------------------------------------------------------------------
% --------------------------- SLIDE --------------------------------------------
\begin{frame}
	\frametitle{Výpočetní model}

	\begin{center}
		abstrakce nad konkrétním výpočetním prostředím

		\bigskip
		\hrule
		\bigskip
		\bigskip

		{\large $\textrm{V} = V_1 \oplus V2 \oplus \ldots \oplus V_k$}
	\end{center}

	\begin{description}
		\item[$V$]	konečný výsledek
		\item[$V_i$]	mezivýsledky
		\item[$\oplus$]	asociativní a komutativní operace spojování
	\end{description}

\end{frame}
% ------------------------------------------------------------------------------
% --------------------------- SLIDE --------------------------------------------
\begin{frame}
	\frametitle{Výpočetní model}

	\begin{itemize}
		\item<1->	2 výpočetní prostředí:
				\begin{itemize}
					\item	sdílená paměť
					\item	distribuovaná paměť
				\end{itemize}
		\item<1->	výpočet má na starost kontejner $\longrightarrow$ výpočetní prostředí
		\item<1->	na začátku 1 výpočetní instance
		\item<1->	instance může kdykoliv vytvářet nové $\longrightarrow$ kontejner
		\item<1->	kontejner sbírá výsledky z vypočtených instancí\\ $\longrightarrow$ konečný výsledek
	\end{itemize}
\end{frame}
% ------------------------------------------------------------------------------
% --------------------------- SLIDE --------------------------------------------
\begin{frame}
	\frametitle{Distribuované prostředí}

	\begin{center}
		\begin{tikzpicture}[node distance=3cm, text width=1.5cm, text centered]
			\node[rectangle, rounded corners, draw=black!50, inner sep=0.1cm, text width=2cm, dashed] 	(master-package)	{
				\begin{tikzpicture}\begin{scope}[text width=1.5cm]
					\node[rectangle]					(master-master)		{master};
					\node[rectangle, fill=green!50, below of=master-master, node distance=0.5cm]	(master-parasim) 	{parasim};
				\end{scope}\end{tikzpicture}
			};
			\node[rectangle, rounded corners, draw=black!50, inner sep=0.1cm, text width=2cm, dashed, below of=master-package] 	(slave-package1)	{
				\begin{tikzpicture}\begin{scope}[text width=1.5cm]
					\node[rectangle]					(slave-slave)		{slave};
					\node[rectangle, fill=green!50, below of=slave-slave, node distance=0.5cm]	(slave-parasim) 	{parasim};
					\node[rectangle, fill=black!10, below of=slave-parasim, node distance=0.6cm]	(slave-queue)		{fronta};
					\node[rectangle, fill=red!10, below of=slave-queue, node distance=0.6cm]	(slave-cpu1)		{CPU};
					\node[rectangle, fill=red!10, below of=slave-cpu1, node distance=0.6cm]	(slave-cpu2)		{CPU};
				\end{scope}\end{tikzpicture}
			};
			\node[rectangle, rounded corners, draw=black!50, inner sep=0.1cm, text width=2cm, dashed, left of=slave-package1] 	(slave-package2)	{
				\begin{tikzpicture}\begin{scope}[text width=1.5cm]
					\node[rectangle]					(slave-slave)		{slave};
					\node[rectangle, fill=green!50, below of=slave-slave, node distance=0.5cm]	(slave-parasim) 	{parasim};
					\node[rectangle, fill=black!10, below of=slave-parasim, node distance=0.6cm]	(slave-queue)		{fronta};
					\node[rectangle, fill=red!10, below of=slave-queue, node distance=0.6cm]	(slave-cpu1)		{CPU};
					\node[rectangle, fill=red!10, below of=slave-cpu1, node distance=0.6cm]	(slave-cpu2)		{CPU};
				\end{scope}\end{tikzpicture}
			};
			\node[rectangle, rounded corners, draw=black!50, inner sep=0.1cm, text width=2cm, dashed, right of=slave-package1] 	(slave-package3)	{
				\begin{tikzpicture}\begin{scope}[text width=1.5cm]
					\node[rectangle]					(slave-slave)		{slave};
					\node[rectangle, fill=green!50, below of=slave-slave, node distance=0.5cm]	(slave-parasim) 	{parasim};
					\node[rectangle, fill=black!10, below of=slave-parasim, node distance=0.6cm]	(slave-queue)		{fronta};
					\node[rectangle, fill=red!10, below of=slave-queue, node distance=0.6cm]	(slave-cpu1)		{CPU};
					\node[rectangle, fill=red!10, below of=slave-cpu1, node distance=0.6cm]	(slave-cpu2)		{CPU};
				\end{scope}\end{tikzpicture}
			};
			\begin{scope}[<->, thick, shorten <=1pt]
				\draw	(master-package) -> (slave-package1);
				\draw	(master-package) -> (slave-package2);
				\draw	(master-package) -> (slave-package3);
			\end{scope}
		\end{tikzpicture}
	\end{center}

	\vspace{-1cm}

	\begin{center}{\small
	\begin{columns}[t]
		\begin{column}{.49\textwidth}
			\begin{itemize}
				\item	vytvoření instance
				\item	začátek výpočtu instance
				\item	konec výpočtu instance
			\end{itemize}
		\end{column}
		\begin{column}{.49\textwidth}
			\begin{itemize}
				\item	balancováno
				\item	konec výpočtu
			\end{itemize}
		\end{column}
	\end{columns}
	}\end{center}
\end{frame}
% ------------------------------------------------------------------------------
% --------------------------- SLIDE --------------------------------------------
\begin{frame}
	\frametitle{Počet potřebných primárních bodů}

	\begin{columns}[2]
		\begin{column}{0.5\textwidth}
			\begin{center}
				\textbf{Lorenz 84}

				\includegraphics[width=\textwidth]{../images/generated/lorenz84-iterations-shared-iterations-primary-summary.pdf}
			\end{center}
		\end{column}
		\begin{column}{0.5\textwidth}
			\begin{center}
				\textbf{Lotka-Volterra}

				\includegraphics[width=\textwidth]{../images/generated/lotkav-iterations-shared-iterations-primary-summary.pdf}
			\end{center}
		\end{column}
	\end{columns}
\end{frame}
% ------------------------------------------------------------------------------
% --------------------------- SLIDE --------------------------------------------
\begin{frame}
	\frametitle{Škálovatelnost}

	\begin{columns}[2]
		\begin{column}{0.5\textwidth}
			\begin{center}
				\textbf{Lorenz 84}

				\includegraphics[width=\textwidth]{../images/generated/lorenz84-iterations-dist-time.pdf}
			\end{center}
		\end{column}
		\begin{column}{0.5\textwidth}
			\begin{center}
				\textbf{Lotka-Volterra}

				\includegraphics[width=\textwidth]{../images/generated/lotkav-iterations-dist-time.pdf}
			\end{center}
		\end{column}
	\end{columns}
\end{frame}
% ------------------------------------------------------------------------------
% --------------------------- SLIDE --------------------------------------------
\begin{frame}
	\frametitle{Závěr}
	\begin{itemize}
		\item	algoritmus pro analýzu dynamických systémů rozšířený o~lokální robustnost
		\item	implementace v nástroji Parasim
		\item	paralelizace ve sdílené a distribuované paměti
		\item	velký důraz na rozšiřitelnost
	\end{itemize}
\end{frame}
% ------------------------------------------------------------------------------
% --------------------------- SLIDE --------------------------------------------
\begin{frame}
	\begin{center}
		{\Huge Otázky}
	\end{center}
\end{frame}
% ------------------------------------------------------------------------------
% --------------------------- SLIDE --------------------------------------------
\begin{frame}
	\frametitle{Balancování}
	\begin{center}
		{\large
			$\textrm{busy} \geq \textsc{Busy\_Bound}$

			$\wedge$

			$\textrm{idle} \leq \textsc{Idle\_Bound}$

			$\wedge$

			$\textrm{busy} \geq K \cdot \textrm{idle}$
		}

		\bigskip
		\hrule
		\bigskip

		balancuje se instace s nejvyšší úrovní zahuštění

		\bigskip

		alternativou je instance s nejnižším cache hit\\
		(výpočetně náročnější)
	\end{center}
\end{frame}
%% ------------------------------------------------------------------------------
%% --------------------------- SLIDE --------------------------------------------
%\begin{frame}
%	\frametitle{Podíl numerické simulace na času výpočtu (90 \%)}
%	\begin{center}
%		jvisualvm (odhad) / vlastní rozšíření
%		\includegraphics[width=.8\textwidth]{../images/jvisualvm.png}
%	\end{center}
%\end{frame}
%% ------------------------------------------------------------------------------
% --------------------------- SLIDE --------------------------------------------
\begin{frame}
	\frametitle{Globální robustnost}
	\begin{center}
		aktuálně vypnuté z důvodu výkonu

		\bigskip

		\begin{tikzpicture}[node distance=2cm]
			\draw[thin,color=black,dotted] (0,0) grid (4,4);
			\tikzstyle{main}=[circle,fill=black,minimum size=5pt,inner sep=0pt, draw=black]
			\tikzstyle{neigh}=[circle,fill=white,minimum size=5pt,inner sep=0pt,draw=black]

			\node[main, label=above:A]	(A)	at (2,2)	{};
			\node[main, label=above:O]	(A)	at (2,3)	{};
			\node[main, label=above:P]	(P)	at (3,2)	{};
			\node[main, label=above:Q]	(Q)	at (2,1)	{};
			\node[main, label=above:R]	(R)	at (1,2)	{};

			\begin{scope}[-]
				\draw	(1,1) -> (3,3);
				\draw	(1,3) -> (3,1);
				\draw	(2,0) -> (4,2);
				\draw	(4,2) -> (2,4);
				\draw	(2,4) -> (0,2);
				\draw	(0,2) -> (2,0);
			\end{scope}
		\end{tikzpicture}

		\bigskip

		\only<1>{$f(A) = \frac{R_A + f(O) + f(P) + f(Q) + f(R)}{5}$}
		\only<2>{$f(A) = \frac{R_A + \sum_{i=1}^{|N|}f(N_i) + (2d - |N|) \cdot R_A}{1 + 2d}$}
	\end{center}
\end{frame}
% ------------------------------------------------------------------------------
% --------------------------- SLIDE --------------------------------------------
\begin{frame}
	\frametitle{Systém rozšíření}
	\begin{itemize}
		\item<1->	\alert{jádro}
				\begin{itemize}
					\item	životní cyklus
					\item	kontexty + služby
					\item	jednotná konfigurace
					\item	obohacování objektů
					\item	vzdálený přístup
				\end{itemize}

		\item<1->	\alert{rozšíření}
				\begin{itemize}
					\item	numerická simulace
					\item	výpočet robustnosti \only<1->{(2 implementace, Tomáš Vejpustek)}
					\item	zahušťování
					\item	výpočetní model
					\item	vizualizace \only<1->{(Tomáš Vejpustek)}
					\item	uživatelské rozhraní \only<1->{(Tomáš Vejpustek)}
					\item	a další \ldots
				\end{itemize}
	\end{itemize}
\end{frame}
% ------------------------------------------------------------------------------
% --------------------------- SLIDE --------------------------------------------
\begin{frame}
	\frametitle{Kritická místa aplikace a možná vylepšení I}

	\begin{itemize}
		\item	numerická simulace (nyní GNU Octave)
		\item	plná podpora SBML
		\item	možnost navázat na předchozí výpočty
		\item	vizualizace pro více parametrů
		\item	refaktorování za účelem výpočtu globální robustnosti
	\end{itemize}
\end{frame}
% ------------------------------------------------------------------------------
% --------------------------- SLIDE --------------------------------------------
\begin{frame}
	\frametitle{Kritcká místa a možná vylepšení II}
	\begin{center}
		\includegraphics[width=.8\textwidth]{../images/generated/meyer91-analysis-wrong-density.pdf}
	\end{center}
\end{frame}
% ------------------------------------------------------------------------------
\end{document}
