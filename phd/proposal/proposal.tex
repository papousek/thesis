	\documentclass[table,color]{fithesis3/fithesis3}
\usepackage[english]{babel}
\usepackage{amsmath}
\usepackage{tabularx}
\usepackage{booktabs}
\thesissetup{
	faculty=fi,
	title=Adaptive Practice of Facts,
	type=r,
	author=Jan Papou\v{s}ek,
	advisor=Radek Pel\'{a}nek
}

\begin{document}

\chapter{Introduction}
Using of computers and other electronic devices becomes more and more common in
the educational process. This allows

\begin{itemize}
	\item	wide usage of computers $\rightarrow$ help teachers with routine tasks
				and provide personalized approach for students;
	\item	teacher's aspect vs. student's aspect;
	\item	providing content, complex exercises vs. practicing atomic tasks;
	\item	domains where you need to learn facts -- vocabulary, anatomy, \ldots;
	\item	fixed amount of time at school vs. online environment with a lot of
				possible activities and high fluctuation;
	\item	the system providing practice needs to attract and keep the attention
				of users and spend the given amount of time in the most effective way
				with respect to learning;
	\item	different users have different background - not only knowledge, but
				also the motivation of the practice can differ (vs. school where the
				motivation is $\pm$ the same);
	\item	systems motivating users to stay active $\neq$ systems effective in
				learning;
	\item	there is no obvious technique for practicing in online environment;
	\item	to build a good adaptive you need at least:
		\begin{itemize}
			\item	a good model describing the user's knowledge;
			\item	good strategy for practicing -- may differ for different types of
						users (some users want to learn for the tommorrow exam, some users want to
						learn in long term);
		\end{itemize}
	\item	we can't design a good strategy for practice when we can not measure
				its impact on the user's motivation to practice and the effectivenes for
				learning -- big methodological issue in online environment;
	\item	when we build a widely used system for practice
\end{itemize}

\chapter{State of the Art}

\section{Memory and Models}

\begin{itemize}
	\item spacing effect~\cite{ostrow2015blocking,maass2015how,kornell2008learning,kornell2010spacing}
	\item flashcards (\textbf{TODO})~\cite{kornell2014focusing,bjork2013self}
	\item learning and forgetting curves~\cite{streeter2015mixture}
	\item recall vs. recognition
	\item BKT, IRT, \ldots
\end{itemize}

\section{Systems}

\begin{itemize}
	\item Duolingo~\cite{von2013duolingo, garcia2013learning}
	\item Fact and Concept Training~\cite{pavlik2007fact,pavlik2008using}
\end{itemize}

\section{Evaluation}

\begin{itemize}
	\item only student model and its accuracy~\cite{pelanek2014brief}
	\item saved time for getting mastery~\cite{yudelson2015small}
	\item combination of historical data and simulation~\cite{gonzalez2015your}
	\item exploring the interaction loop using synthetic data~\cite{niznan2015exploring}
	\item learning vs. time spent in a system~\cite{lomas2013optimizing, monterrat2015player},
	\item online multivariate (AB) testing~\cite{lomas2013optimizing,liu2014towards,stamper2012rise}
	\item multi-armed bandits~\cite{liu2014trading,lopes2015multi}
\end{itemize}

\section{

\section{Aims of the Thesis}

\

\bibliographystyle{plain}
\bibliography{proposal}
\end{document}
