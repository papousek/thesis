\chapter{Algoritmus pro analýzu dynamických systémů}\label{chapter:algorithm}

V této kapitole ukážeme jednu z možností, jak přistoupit k analýze modelů zadaných
pomocí obyčejných diferenciálních rovnic. Zde uvedená analýza se snaží řešit následující
problémy:
\begin{enumerate}
	\item\label{item:init}	Máme k dispozici již hotový model, jehož chování sedí se skutečným
			systémem pro jedno konkrétní nastavení iniciálních hodnot stavovým
			proměnným. Splňuje tento model požadované vlastnosti i pro jiná nastavení?
	\item	Máme kostru modelu, v němž se vyskytuje několik parametrů, jejichž hodnota
			není známá. Jak nastavit parametry modelu tak, aby splňoval dané chování~\cite{aster2012}?
\end{enumerate}

V~bodu \ref{item:init} lze jít ještě dál než ke kontrole modelů. Můžeme si představit
poměrně přesný model, který využijeme k analýze v~podmínkách, které nelze
navodit u reálného systému. Typickým příkladem může být živý organismus v~toxickém
prostředí nebo extrémně vysoká nákaza šířící se ce\-lo\-svě\-to\-vou populací.

\section{Definice problému}

Nechť je dynamický systém $\mathcal{DS} = (\mathbf{X}, f)$, kde $\mathbf{X} = (x_1, \ldots, x_n)$
je vektor stavových proměnných a dynamiku systému popisují obyčejné diferenciální rovnice $\frac{dx_i}{dt} = f_i(\mathbf{X})$.
Tento systém rovnic souhrnně označíme jednou rovnicí $\frac{d\mathbf{X}}{dt} = f(\mathbf{X})$. Narozdíl
od obecného modelu zadaného pomocí systému obyčejných diferenciálních rovnic budeme předpokládat,
že funkce stojící na pravé straně nezávisí na čase. Nechť $\mathbf{P}  = (p_1, \ldots, p_m)$ značí
parametry dostupné v~systému rovnic. Systém rovnic s konkrétním ohodnocení parametrů $[\mathbf{P}]$
označíme $f_{\mathbf{P} \leftarrow [\mathbf{P}]}$.

Pro každou stavovou proměnnou $x_i$ je dán interval $\mathcal{I}_{x_i} = [\iota_{x_i}^{min}, \iota_{x_i}^{max}]$
a pro každý parametr $p_j$ interval $\mathcal{I}_{p_j} = [\iota_{p_j}^{min}, \iota_{p_j}^{max}]$
Tyto intervaly omezují nastavení iniciálních hodnot $[x_i]_0$ stavových proměnných $x_i$, respektive o\-hod\-no\-ce\-ní $[p_j]$
parametrů $p_j$, a určují prostor iniciálních podmínek \cite[str. 23]{drazan2011}.

\begin{align}
\mathcal{I} &= \mathcal{I}_{x_1} \times \mathcal{I}_{x_n} \times \ldots \times \mathcal{I}_{p_1} \times \ldots \times \mathcal{I}_{p_m}
\end{align}

$\mathcal{I} \models [\mathbf{X}]_0$ značí, že množina iniciálních hodnot $[\mathbf{X}]_0$ splňuje
omezení prostoru iniciálních podmínek, podobné označení $\mathcal{I} \models [\mathbf{P}]$ zavedeme
pro ohodnocení parametrů $\mathbf{P}$.

Dále je dána numerická metoda $\mathcal{M}_\varepsilon (f, [\mathbf{X}]_{t_0}, \Delta t) = [\mathbf{X}]_{t_0 + \Delta t}$,
která pro daný systém diferenciálních rovnic $f$,
ohodnocení stavových proměnných $[\mathbf{X}]_{t_0}$ v čase $t_0$ a časový krok $\Delta t$
vrátí stav $\mathbf{X}_{t_0 + \Delta t}$ v čase $t_0 + \Delta t$ s relativní chybou $\varepsilon$.
Díky numerické metodě lze sestrojit již dříve zmíněnou posloupnost vektorů
$\mathcal{M}^{\tau}_\varepsilon(f, [\mathbf{X}]_{t_0}, \Delta t) = [\mathbf{X}]_{t_0}, [\mathbf{X}]_{t_0 + \Delta t}, \ldots, [\mathbf{X}]_{t_0 + \tau}$, kde $\tau$
je množství času, po~který model simulujeme.

Je-li dáná formule temporální logiky signálů $\varphi$, dynamický systém $\mathcal{DS}$
a prostor iniciálních podmínek $\mathcal{I}$, pak řešeným problémem je najít
části prostoru $\mathcal{S}, \mathcal{N} \subseteq \mathcal{I}$ takové, že platí
vztah \ref{eq:initial:value:problem} \cite[str. 23]{drazan2011}.

\begin{align}\label{eq:initial:value:problem}
\forall [\mathbf{X}]_0 \forall [\mathbf{P}] . (\mathcal{S} \models [\mathbf{X}]_0 \wedge \mathcal{S} \models [\mathbf{P}])
\Rightarrow \mathcal{M}^\tau_\varepsilon(f_{\mathbf{P} \leftarrow [\mathbf{P}]}, [\mathbf{X}]_0, \Delta t) \models \varphi \\
\forall [\mathbf{X}]_0 \forall [\mathbf{P}] . (\mathcal{N} \models [\mathbf{X}]_0 \wedge \mathcal{N} \models [\mathbf{P}])
\Rightarrow \mathcal{M}^\tau_\varepsilon(f_{\mathbf{P} \leftarrow [\mathbf{P}]}, [\mathbf{X}]_0, \Delta t) \not\models \varphi
\end{align}

Tyto části prostoru popisují ohodnocení počátečních stavů a parametrů, ze kterých se daný systém vyvíjí
s požadovanou vlastností $\varphi$ a bez po\-ža\-do\-vané vlastnosti. Naivní algoritmus řešící
tento problém do prostoru iniciálních podmínek vloží určité množství bodů a tyto body
použije pro simulaci chování modelu, nad kterým se následně provede ověření vlastnosti.
Počet bodů, tedy míra zahuštění prostoru iniciálních podmínek, závisí na požadované přesnosti
analýzy. I přes nesporné výhody tohoto přístupu snadno narazíme na výpočetní limity
potřebného množství bodů, a proto je vhodné pokusit se počet bodů omezit.


\section{Původní algoritmus}

Původní algoritmus, na kterém staví tato práce vychází z velice důležitého předpokladu.
\uv{Většina řešení začínajících v iniciálních bodech blízko sebe zůstávají blízko sebe
i v průběhu času~\cite[str. 25]{drazan2011}.} Předpokládá se tedy, že chování určená blízkými
hodnotami z prostoru iniciálních podmínek mají i blízkou míru platnosti dané formule.
Není třeba tedy zjišťovat chování pro všechny body prostoru inicálních podmínek,
ale jen pro určitou množinu reprezentantů, pro kterou platí \cite[str. 25]{drazan2011}:

\begin{enumerate}
	\item	chování blízké reprezentatovi zůstane blízké na celém časovém intervalu,
			na kterém je daná formule ověřována,
	\item	množina reprezenantů pokrývá celý prostor iniciálních podmínek.
\end{enumerate}

Algoritmus do prostoru iniciálních podmínek vkládá body tak dlouho, dokud si trajektorie chování
určených blízkými body jsou vzdálenější než daná vzdálenost $\delta$. Nad simulovanými chováními
se ověří platnost formule. Výsledkem je určité množství bodů, u kterých dostáváme informace,
zda z nich simulované chování splňuje či nesplňuje danou vlastnost. Tyto body nastíní
hranice regionů platnosti a neplatnosti se zvolenou přes\-nos\-tí~$\delta$.

%\begin{algorithm}
%\caption{Analýza prostoru inicálních podmínek}
%\begin{algorithmic}
%\REQUIRE 	$\mathcal{DS} = (\mathbf{X}, f), \mathcal{I}, \varphi$
%\ENSURE 	$\mathcal{F} = \{([\mathbf{F}_0]_0, s_1), \ldots ([\mathbf{F}_k]_0, s_k)\}$ -- body a platnosti formule $\varphi$
%\STATE
%\STATE		$main_{new} 	\gets $ počáteční zahuštění $\mathcal{I}$
%\WHILE{$main_{new} \neq \emptyset$}
%	\STATE	$neigh 	\gets $ vedlejší body pro body z $P$
%	\STATE 	$MAIN 	\gets $ nasimuluj všechny body z $main_{new}$
%	\STATE	$NEIGH	\gets $ nasimuluj všechny body z $neigh$
%	\STATE	$
%\ENDWHILE
%\STATE 		
%
%\STATE		
%\end{algorithmic}
%\end{algorithm}

Je samozřejme otázkou, jakým způsobem se ověří, zda si jsou trajektorie bližší než
dané $\delta$. Tomuto tématu se podrobně zabývá \cite{drazan2011}. Tato
práce se spíše zaměřovat na otázku, jakou zvolit hodnotu $\delta$.

\section{Robustnost}\label{section:robustness}
\section{Upravený algoritmus}
