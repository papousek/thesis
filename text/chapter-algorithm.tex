\chapter{Algoritmus pro analýzu dynamických systémů}\label{chapter:algorthm}

V této kapitole ukážeme jednu z možností, jak přistoupit k analýze modelů zadaných
pomocí obyčejných diferenciálních rovnic. Zde uvedená analýza se snaží řešit následující
problémy:
\begin{enumerate}
	\item\label{item:init}	Máme k dispozici již hotový model, jehož chování sedí se skutečným
			systémem pro jedno konkrétní nastavení iniciálních hodnot stavovým
			proměnným. Splňuje tento model požadované vlastnosti i pro jiná nastavení?
	\item	Máme kostru modelu, v němž se vyskytuje několik parametrů, jejichž hodnota
			není známá. Jak nastavit parametry modelu tak, aby splňoval dané chování~\cite{aster2012}?
\end{enumerate}

V~bodu \ref{item:init} lze jít ještě dál než ke kontrole modelů. Můžeme si představit
poměrně přesný model, který využijeme k analýze v~podmínkách, které nelze
navodit u reálného systému. Typickým příkladem může být živý organismus v~toxickém
prostředí nebo extrémně vysoká nákaza šířící se ce\-lo\-svě\-to\-vou populací.

\section{Definice problému}\label{section:initial:condtion:problem:definition}

Nechť je dynamický systém $\mathcal{DS} = (\mathbf{X}, f)$, kde $\mathbf{X} = (x_1, \ldots, x_n)$
je vektor stavových proměnných a dynamiku systému popisují obyčejné diferenciální rovnice $\frac{dx_i}{dt} = f_i(\mathbf{X})$.
Tento systém rovnic souhrnně označíme jednou rovnicí $\frac{d\mathbf{X}}{dt} = f(\mathbf{X})$. Narozdíl
od obecného modelu zadaného pomocí systému obyčejných diferenciálních rovnic budeme předpokládat,
že funkce stojící na pravé straně nezávisí na čase. Nechť $\mathbf{P}  = (p_1, \ldots, p_m)$ značí
parametry dostupné v~systému rovnic. Systém rovnic s konkrétním ohodnocení parametrů $[\mathbf{P}]$
označíme $f_{\mathbf{P} \leftarrow [\mathbf{P}]}$.

Pro každou stavovou proměnnou $x_i$ je dán interval $\mathcal{I}_{x_i} = [\iota_{x_i}^{min}, \iota_{x_i}^{max}]$
a pro každý parametr $p_j$ interval $\mathcal{I}_{p_j} = [\iota_{p_j}^{min}, \iota_{p_j}^{max}]$
Tyto intervaly omezují nastavení iniciálních hodnot $[x_i]_0$ stavových proměnných $x_i$, respektive o\-hod\-no\-ce\-ní $[p_j]$
parametrů $p_j$, a určují prostor iniciálních podmínek \cite[str. 23]{drazan2011}.

\begin{align}
\mathcal{I} &= \mathcal{I}_{x_1} \times \mathcal{I}_{x_n} \times \ldots \times \mathcal{I}_{p_1} \times \ldots \times \mathcal{I}_{p_m}
\end{align}

$\mathcal{I} \models [\mathbf{X}]_0$ značí, že množina iniciálních hodnot $[\mathbf{X}]_0$ splňuje
omezení prostoru iniciálních podmínek, podobné označení $\mathcal{I} \models [\mathbf{P}]$ zavedeme
pro ohodnocení parametrů $\mathbf{P}$.

Dále je dána numerická metoda $\mathcal{M}_\varepsilon (f, [\mathbf{X}]_{t_0}, \Delta t) = [\mathbf{X}]_{t_0 + \Delta t}$,
která pro daný systém diferenciálních rovnic $f$,
ohodnocení stavových proměnných $[\mathbf{X}]_{t_0}$ v čase $t_0$ a časový krok $\Delta t$
vrátí stav $\mathbf{X}_{t_0 + \Delta t}$ v čase $t_0 + \Delta t$ s relativní chybou $\varepsilon$.
Díky numerické metodě lze sestrojit již dříve zmíněnou posloupnost vektorů
$\mathcal{M}^{\tau}_\varepsilon(f, [\mathbf{X}]_{t_0}, \Delta t) = [\mathbf{X}]_{t_0}, [\mathbf{X}]_{t_0 + \Delta t}, \ldots, [\mathbf{X}]_{t_0 + \tau}$, kde $\tau$
je množství času, po~který model simulujeme.

Je-li dáná formule temporální logiky signálů $\varphi$, dynamický systém $\mathcal{DS}$
a prostor iniciálních podmínek $\mathcal{I}$, pak řešeným problémem je najít
části prostoru $\mathcal{S}, \mathcal{N} \subseteq \mathcal{I}$ takové, že platí
vztah \ref{eq:initial:value:problem} \cite[str. 23]{drazan2011}.

\begin{align}\label{eq:initial:value:problem}
\forall [\mathbf{X}]_0 \forall [\mathbf{P}] . (\mathcal{S} \models [\mathbf{X}]_0 \wedge \mathcal{S} \models [\mathbf{P}])
\Rightarrow \mathcal{M}^\tau_\varepsilon(f_{\mathbf{P} \leftarrow [\mathbf{P}]}, [\mathbf{X}]_0, \Delta t) \models \varphi \\
\forall [\mathbf{X}]_0 \forall [\mathbf{P}] . (\mathcal{N} \models [\mathbf{X}]_0 \wedge \mathcal{N} \models [\mathbf{P}])
\Rightarrow \mathcal{M}^\tau_\varepsilon(f_{\mathbf{P} \leftarrow [\mathbf{P}]}, [\mathbf{X}]_0, \Delta t) \not\models \varphi
\end{align}

Tyto části prostoru popisují ohodnocení počátečních stavů a parametrů, ze kterých se daný systém vyvíjí
s požadovanou vlastností $\varphi$ a bez po\-ža\-do\-vané vlastnosti. Naivní algoritmus řešící
tento problém do prostoru iniciálních podmínek vloží určité množství bodů a tyto body
použije pro simulaci chování modelu, nad kterým se následně provede ověření vlastnosti.
Počet bodů, tedy míra zahuštění prostoru iniciálních podmínek, závisí na požadované přesnosti
analýzy. I přes nesporné výhody tohoto přístupu snadno narazíme na výpočetní limity
potřebného množství bodů, a proto je vhodné pokusit se počet bodů omezit.


\section{Původní algoritmus}

Původní algoritmus, na kterém staví tato práce vychází z velice důležitého předpokladu.
\uv{Většina řešení začínajících v iniciálních bodech blízko sebe zůstávají blízko sebe
i v průběhu času~\cite[str. 25]{drazan2011}.} Předpokládá se tedy, že chování určená blízkými
hodnotami z prostoru iniciálních podmínek mají i blízkou míru platnosti dané formule.
Není třeba tedy zjišťovat chování pro všechny body prostoru inicálních podmínek,
ale jen pro určitou množinu reprezentantů, pro kterou platí \cite[str. 25]{drazan2011}:

\begin{enumerate}
	\item	chování blízké reprezentatovi zůstane blízké na celém časovém intervalu,
			na kterém je daná formule ověřována,
	\item	množina reprezenantů pokrývá celý prostor iniciálních podmínek.
\end{enumerate}

Algoritmus do prostoru iniciálních podmínek vkládá body tak dlouho, dokud si trajektorie chování
určených blízkými body jsou vzdálenější než daná vzdálenost $\delta$. Nad simulovanými chováními
se ověří platnost formule. Výsledkem je určité množství bodů, u kterých dostáváme informace,
zda z nich simulované chování splňuje či nesplňuje danou vlastnost. Tyto body nastíní
hranice regionů platnosti a neplatnosti se zvolenou přes\-nos\-tí~$\delta$.

\begin{algorithm}
\caption{Analýza prostoru inicálních podmínek}
\begin{algorithmic}
\Require 	$\mathcal{DS} = (\mathbf{X}, f), \mathcal{I}, \varphi, \Delta t, \delta, \varepsilon$
\Ensure 	$\textsc{Result} = \{([\mathbf{F}_0]_0, s_1), \ldots ([\mathbf{F}_k]_0, s_k)\}$ -- body a splněnost $\varphi$
\State		$\textsc{M}_{new} 	\gets $ počáteční zahuštění $\mathcal{I}$
\State		$\textsc{Result} \gets \emptyset$
\While{$\textsc{M}_{new} \neq \emptyset$}
	\State $\textsc{M}_{old} \gets \textsc{M}_{new}, \textsc{M}_{new} \gets \emptyset$
	\For{$([\mathbf{X}_{main}]_0, [\mathbf{P}_{main}]) \in \textsc{M}_{old}$}
		\State $\textsc{Trajectory}_{main} \gets \mathcal{M}^{||\varphi||}_\varepsilon(f_{\mathbf{P} \gets [\mathbf{P}_{main}]}, [\mathbf{X}_{main}]_0, \Delta t)$
		\State $\textsc{Satisfied}_{main} \gets $ splněnost $\varphi$ nad $\textsc{Trajectory}_{main}$
		\State $\textsc{Result} \gets \textsc{Result} \cup \{(\textsc{Satisfied}_{main}, [\mathbf{X}_{main}]_0, [\mathbf{P}_{main}])\}$
		\State $\textsc{Neigh} \gets $ body sousedící s bodem $([\mathbf{X}]_0, [\mathbf{P}])$
		\For{$([\mathbf{X}_{neigh}]_0, [\textbf{P}_{neigh}]) \in \textsc{Neigh}$}
			\State $\textsc{Trajectory}_{neigh} \gets \mathcal{M}^{||\varphi||}_\varepsilon(f_{\mathbf{P} \gets [\mathbf{P}_{neigh}]}, [\mathbf{X}_{neigh}]_0, \Delta t)$
			\State $\textsc{Satisfied}_{neigh} \gets $ splněnost $\varphi$ nad $\textsc{Trajectory}_{neigh}$
			\State $\textsc{Result} \gets \textsc{Result} \cup \{(\textsc{Satisfied}_{main}, [\mathbf{X}_{main}]_0, [\mathbf{P}_{main}])\}$
			\State $\textsc{Distance} \gets $ vzdálenost $\textsc{Trajectory}_{main}$ a $\textsc{Trajectory}_{neigh}$
			\If{$\textsc{Distance} > \delta$}
				\State	$\textsc{M}_{new} \gets \textsc{M}_{new} \cup \{(\frac{[\mathbf{X}_{main}]_0 + [\mathbf{X}_{neigh}]_0}{2}, \frac{[\mathbf{P}_{main}] + [\mathbf{P}_{neigh}]}{2})\}$
			\EndIf
		\EndFor
	\EndFor
\EndWhile
\end{algorithmic}
\end{algorithm}

Je samozřejme otázkou, jakým způsobem konkrétně probíhá počáteční zahuštění prostoru
iniciálních podmínek, co přesně znamená obsahuje mno\-žina sousedů daného bodu a jak
se se změří vzdálenost dvou chování. Poslední zmíněné otázce se podrobně zabývá \cite{drazan2011}.
Zbytek bude ještě rozveden v sekci \ref{section:algorithm:updated} společně s tím,
jak zvolit hodnotu $\delta$.

\section{Robustnost}\label{section:robustness}

Platnost formule lze chápat trochu šířeji než prosté \uv{platí}/\uv{neplatí}.
Nemusíme se pouze ptát, zda dané chování splňuje danou formuli, otázku lze posunout dál.
Jak moc dané chování splňuje danou formuli? Jak moc je vlastnost nesplňena? Pro účely
této práce tyto otázky zcela postačují, ale samozřejmě je možné požadovat ještě víc.
Jak moc je daný systém robustní vůči změně podmínek, změnám teplot nebo koncentrací
chemických látek?

V této sekci zavedeme pojem \textit{lokální a globální robustnosti}. Lokální robustnosti
rozumíme míru, do jaké je daná vlastnost splněna na jednom chování. Globální robustnost
na druhé straně vztáhneme na celý systém, tzn. že zahrnuje míru platnosti dané vlastnosti
nad větším množství chování, která vzniknou tzv. \textit{perturbacemi}. Existuje jedno
referenční chování za ideálních podmínek, a pak mnoho perturbovaných chování za podmínek
ne tak i\-deál\-ních. Perturbace lze chápat různě, v této práci odpovídají prostoru i\-ni\-ciál\-ních
podmínek definovaného v sekci \ref{section:initial:condtion:problem:definition}. 

Jednou z věcí, které je třeba předem zmínit, je, že pro účely vylepšení algoritmu pro
analýzu dynamických systémů, se na robustnost díváme z pohledu chování systému \cite{donze2011}.
To znamená, že se při výpočtu lokální robustnosti snažíme ohraničit prostor okolo jednoho chování,
ve kterém má daná vlastnost stejnou platnost. Podobně lze k problému přistoupit z opačné strany \cite{rizk2009}.
Ve formuli identifikovat parametry, napočítat podprostor parametrického prostoru, ve kterém je formule
pro dané chování splněna splněna a určit vzdálenost již konkrétní formule s dosazenými parametry
od tohoto podprostoru.

\subsection{Lokální robustnost}

Nechť $s_1$, $s_2$ jsou signály nad doménou $\mathbb{D}$ a $d: \mathbb{D}^n \rightarrow \mathbb{R}^{+}$
funkce určující vzdálenost dvou bodů v prostoru $\mathbb{R}^n$. 
Vzdálenost dvou signálů zavedeme předpisem \ref{eq:signal:distance}.

\begin{align}\label{eq:signal:distance}
\sigma(s_1, s_2) = {\displaystyle \sup_{t \in \mathbb{R}^{+}}} \{d(s_1(t), s_1(t))\}
\end{align}

Od robustnosti $\rho$ budeme požadovat konzistentní chování s již zavedenou dvouhodnotovou platností
formule. 

\begin{align}\label{eq:signal:distance}
\rho(\varphi, s) = \left\{\begin{array}{r@{\quad}c}
\inf\{\sigma(s, s') | \forall s'. s' \not\models \varphi \}	& \textrm{pokud } s \models \varphi	\\
- \inf\{\sigma(s, s') | \forall s'. s' \models \varphi \}	& \textrm{pokud } s \not\models \varphi
\end{array} \right.
\end{align}

Oborem hodnot funkcí $\mu_i$ odpovídajících atomickým prozicím tentokrát není množina $\{T, F\}$,
nýbrž množina reálných čísel $\mathbb{R}$. Toto chápání je spíše technického charakteru a převod
ze starého chápání je přímočarý.

\begin{align}\label{eq:stl:semantics}
\begin{array}{ll}
\mu_i = f(\mathbf{X}) \geq k		\leadsto \mu_i = f(\mathbf{X}) - k							\\
\mu_i = f(\mathbf{X}) \leq k		\leadsto \mu_i = k - f(\mathbf{X})
\end{array}
\end{align}

Obdobně jako při výpočtu dvouhodnotové platnosti formule i u robustnosti je potřeba se odkazovat
na robustnost v určitém čase $\rho(\varphi, s, t)$, která se definuje induktivně ke struktuře
formule. Výslednou robustnost dostaneme položením $\rho(\varphi, s) = \rho(\varphi, s, 0)$.

\begin{align}\label{eq:stl:semantics}
\begin{array}{ll}
\rho(p, s, t)											&= \pi_p(s)[t]											\\
\rho(\neg\varphi, s, t)									&= - \rho(\varphi, s, t)								\\
\rho(\varphi_1 \wedge \varphi_2, s, t)					&= \min\Big(\rho(\varphi_1, s, t), \rho(\varphi_1, s, t)\Big)	\\
\rho(\varphi_1 \mathcal{U}_{[a, b]} \varphi_2, s, t)	&= {\displaystyle \max_{t' \in [t + a, t + b]}} min\Big(\rho(\varphi_2, s, t'), {\displaystyle\min_{t'' \in [t, t']}}\rho(\varphi_1, s, t'')\Big)
\end{array}
\end{align}

Pro odvozené temporální operátory $\mathcal{F}$ a $\mathcal{G}$ lze definovat robustnost
samostatně, což umožňuje naimplementovat její výpočet efektivněji.

\begin{align}\label{eq:stl:semantics}
\begin{array}{ll}
\rho(\mathcal{F}_{[a, b]}\varphi)		&= {\displaystyle \max_{t' \in [t + a, t + b]}} \Big(\rho(\varphi, s, t')\Big)		\\
\rho(\mathcal{G}_{[a, b]}\varphi)		&= {\displaystyle \min_{t' \in [t + a, t + b]}} \Big(\rho(\varphi, s, t')\Big)		
\end{array}
\end{align}

Jestliže máme k dispozici analyzovaný primární signál, lze použit funkci robustnosti pro definici tzv. sekundárních
signálů příslušejících každé podformuli $\varphi'$ dané formule $\varphi$, kde $s_{\varphi'}[t] = \rho(\varphi', s, t)$.
a $|s_{\varphi'}| = ||\varphi'||$. Tyto se\-kun\-dár\-ní signály nám dávají informaci o tom, jak moc se lze od primárního
signálu vzdálit, aby daná podformule zůstala ještě platná v případě klad\-ných hodnot sekundárního signálů
nebo neplatná v případě hodnot zá\-por\-ných. Krajní hodnotou je signál nulový, který značí hranici platnosti formule.

%\subsection{Příklad sekundárního signálu}

Vraťme se k příkladu formule popisující oscilaci systému s jednou stavovou proměnnou ze 
sekce \ref{section:stl:example}. Ukážeme si, jak vypadá sekundární signál pro podformuli
$\mathcal{F}_{[0, \frac{1}{2}]}x \leq -k$.

\begin{figure}[h!]
\begin{center}
\subfigure[$x \leq -k$]{
	\includegraphics[width=.48\textwidth]{../images/robustness-example-leq-limit.pdf}
}
\subfigure[$\mathcal{F}_{[0, \frac{1}{2}]}x \leq -k$]{
	\includegraphics[width=.48\textwidth]{../images/robustness-example-future.pdf}
}
\caption{Znázornění sekundárních signálů pro některé z podformulí vlastnosti oscilace. Zelené podbarvení
značí platnost dané formule v daném čase, červené naopak neplatnost.}
\end{center}
\end{figure}

\subsection{Výpočet lokální robustnosti}

Za zmínku stojí způsob, jakým lze výpočet robustnosti naimplementovat. Od nejvíce zanořených
podformulí se počítají sekundární signály, které se použijí pro výpočet sekundárních signálů
pro nadřazené podformule. Není překvapením, že sekundární signál pro základní predikátové operátory
$\neg$ a $\wedge$ lze spočítat v lineárním čase vzhledem k délce signálu. Pro hledání maxima,
respektive minima pro všechny podsekvence dané sekvence

prvků existuje algoritmus s lineární časovou složitostí~\cite{lemire2006}. Díky tomu lze
výpočet sekundárního signálu i odvozených operátorů $\mathcal{F}$ a $\mathcal{G}$ pro\-vést
rovněž v lineárním čase. Použitím pomocné funkce, jejíž struktura je popsaná v algoritmu \ref{algorithm:unary:signal}
získáme předpis 
\begin{align}\label{eq:future:gloabally:robustness:impl}
\begin{array}{ll}
s_{\mathcal{F}_{[a,b]}\varphi} = \textsc{Signal}([a,b], \varphi, >)				\\
s_{\mathcal{G}_{[a,b]}\varphi} = \textsc{Signal}([a,b], \varphi, <)
\end{array}
\end{align}

Výpočet sekundárního signálu pro operátor $\mathcal{U}$ má kvadratickou časovou složitost,
čehož lze docílit přímočarou implementací.

\begin{algorithm}
\caption{datová struktura \textsc{Lemire-Queue}\cite{lemire2006}}
\begin{algorithmic}
\Require 	$\prec \subseteq \mathbb{R}^2$ \Comment{ostré uspořádání}
\State $deque$ \Comment{fronta s přístupem k oběma koncům}
\Function{Lemire-Queue.offer}{$time, value$}
	\While{$\neg deque.\textsc{isEmpty()} \wedge value \prec \textsc{Dequeue.getLast()}.value$}
		\State $deque$.\textsc{removeLast()}
	\EndWhile
	\State $deque$.\textsc{offer(}$time, value$\textsc{)}
\EndFunction
\Function{Lemire-Queue.peek}{~}
	\State\Return $deque$.\textsc{getFirst()}
\EndFunction
\Function{Lemire-Queue.Poll}{~}
	\State\Return $deque$.\textsc{removeFirst()}
\EndFunction
\end{algorithmic}
\end{algorithm}

\begin{algorithm}\label{algorithm:unary:signal}
\caption{pomocná funkce pro sekundárního signálu}
\begin{algorithmic}
\Function{Signal}{$[a, b]$, $s_\varphi$, $\prec$}
	\State	$queue \gets $ nová \textsc{Lemire-Queue} s uspořádáním $\prec$
	\State	$monitor \gets ()$
	\State	$t' \gets t_0$
	\While{$t'\leq t_0 + b - \Delta t$}
		\State $queue$.\textsc{Offer}($s_\varphi[t']$)
		\State $t' \gets t' + \Delta t$
	\EndWhile
	\State	$t' \gets t_0$
	\While{$t' \leq |s|$}
		\State	$qeue$.\textsc{offer}($s_\varphi[t']$)
		\State	$monitor[t'] \gets queue$.\textsc{peek()}
		\If{$queue$.\textsc{peek()}.$time = t'$}
			\State	$queue$.\textsc{poll()}
		\EndIf
		\State	$t' \gets t' + \Delta t$
	\EndWhile
	\State\Return $monitor$
\EndFunction
\end{algorithmic}
\end{algorithm}

\subsection{Globální robustnost}

\begin{align}\label{eq:future:gloabally:robustness:impl}
\begin{array}{ll}
R_{\varphi, P}^\mathcal{S} = {\displaystyle\int_{p \in P}}prob(p) \cdot D_\varphi^\mathcal{S}dp
\end{array}
\end{align}

\begin{align}\label{eq:future:gloabally:robustness:impl}
\begin{array}{ll}
R_{\varphi, P}^\mathcal{S} = {\displaystyle\int_{p \in P}}prob(p) \cdot \rho(\varphi, s_p)dp
\end{array}
\end{align}

% Kitano - globalni robustnost

% STL - lokalni robustnost

% Priklad

% Vypocet robustnosti

\section{Upravený algoritmus}\label{section:algorithm:updated}
