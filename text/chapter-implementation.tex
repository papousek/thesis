\chapter{Implementace}
Následující kapitola se věnuje aplikaci Parasim \cite{TODO}, která implementuje algoritmus pro
analýzu dynamických systémů zmíněný v sekci \ref{section:algorithm:updated}. Aplikace Parasim
vznikla na základě prototypu dostupného v \cite{drazan2011}. Cílem prototypu je názorně
zobrazit průběh výpočtu původního algoritmu. Uživateli se ukazuje, jakým způsobem
se počítá vzdálenost mezi trajektoriemi chování, kde je nutné zahušťovat a jak dopadlo
ověření platnosti. Vzhledem k tomu, že prototyp například používá pouze jednoduchou metodu
numerické simulace k nalezení trajektorií chování, není vhodné jej použít k analýze složitějších modelů.

V nové implementaci bylo oproti prototypu třeba zahrnout následující:

\begin{enumerate}
	\item	úprava algoritmu dle sekce \ref{section:algorithm:updated},
	\item	paralelní výpočet ve sdílené nebo distribuované paměti,
	\item	rozšiřitelnost, modularita a otevřenost k rozdílné implementaci již naimplementovaných částí,\label{item:features:extensibility}
	\item	zobrazení výsledků analýzy pro vícedimenzionální prostory ini\-ciál\-ních podmínek.\label{item:features:visualisation}
\end{enumerate}

Tato práce se zabývá všemi zmíněnými body kromě bodu \ref{item:features:visualisation},
který je však v Parasimu již také vyřešen. Vzhledem k monolitické implementaci prototypu
nedošlo k jeho úpravám a rozšíření, ale vytvořila se zcela od základu nová aplikace Parasim.
Bod \ref{item:features:extensibility} je důležitý z několika důvodů. Pro různé části algoritmu
existuje více způsobů, jak je naimplementovat. Existuje mnoho nástrojů umožňující
numerickou simulací na základě systému diferenciálních rovnic získat trajektorii chování.
Mnoho věcí zmíněných dříve se vyvíjí. V případě výpočtu robustnosti se nemusíme omezit
pouze na temporální logiku signálů, ale můžeme zavést logiku novou, expresivněj\-ší.

\section{Architektura}

Parasim je aplikace napsaná v Javě 7 složená z většího množství artefaktů pro
sestavovací nástroj Maven. Zdrojové kódy jsou pod licencí GNU GPL verze 3~\cite{gpl} k dispozici
v Git \cite{chacon2009} repozitáři \url{https://github.com/sybila/parasim}. Artefakty sestavené
z poslední verze těchto zdrojových jsou publikovány do Maven repozitáře
\url{http://repository-sybila.forge.cloudbees.com/snapshot/}, o což se stará veřejná
instance nástroje Jenkins \cite{jenkins} \url{http://www.cloudbees.com/}. Artefakty
vydaných verzí jsou k nalezení v Maven repozitáři \url{http://repository-sybila.forge.cloudbees.com/release}.
Hlavní aplikaci Parasimu představuje artefakt \texttt{org.sybila.parasim.application:parasim}.
Pou\-ži\-tí aplikace je popsáno v příloze \ref{appendix:usage}

Jádro Parasimu tvoří artefakt \texttt{org.sybila.parasim:core}, který řídí ži\-vot\-ní
cyklus všech modulů a zajišťuje základní funkcionalitu. To zajišťuje kontejner,
jehož instance je vytvořena a spuštěn v rámci aplikace. Kontejner je velkou mírou inspi\-ro\-ván
konceptem \textit{Context and Dependency Injection} zavedeným v~Java~EE~6~\cite{jendrock2010} specifikací
JSR-299 \cite{jsr299}. V žádném případě se nejedná o~implementaci tohoto standardu,
nýbrž jen o volnou inspiraci některými zá\-klad\-ní\-mi myšlenkami.

Existují zde kontexty, které mají různou délku a je možné je za\-no\-řo\-vat.
Rozšíření pro jednotlivé kontexty nabízejí služby a aplikace definuje na různých
místech v kódu závislosti na těchto službách. Kontejner se stará o to,
aby v případě požadavku na nějakou službu bylo zavoláno příslušné rozříření,
které ji poskytuje, aby vytvořilo objekt služby, a v případě vy\-pr\-še\-ní kontextu,
aby objekt služby vhodným způsobem zničilo.

Za vytvoření a zničení objektu služby je zodpovědný autor rozšíření, který také definuje rozhraní
služby. Aplikace není téměř žádným způsobem závislá na implementaci rozhraní,
a jednotlivé implementace lze tak jednoduše zaměňovat bez toho, aniž bychom
v~aplikačním kódu cokoliv mě\-ni\-li. Kontejner obsahuje i některé základní služby,
mezi které patří například jednotný přístup ke konfiguraci nebo způsob, jak injektovat
služby do atributů daných objektů.

\subsection{Životní cyklus}

Aby bylo možné jádro Parasimu používat, je nutné nastartovat jeho životní cyklus pomocí třídy
\href{https://github.com/sybila/parasim/blob/master/core/src/main/java/org/sybila/parasim/core/impl/ManagerImpl.java}{\texttt{ManagerImpl}}.
Tato třída představuje vstupní bod pro použití všech dále popsaných služeb.
Životní cyklus řízený vytvořením, nastartováním a zničením manažera je
znázorněn diagramem \ref{figure:parasim:lifecycle}. Jakmile je manažer vytvořen,
jsou načtena zá\-klad\-ní rozšíření, kterým je oznámena událost \href{https://github.com/sybila/parasim/blob/master/core/src/main/java/org/sybila/parasim/core/event/ManagerProcessing.java}{\texttt{ManagerProcessing}}.

\begin{figure}[h!]
\label{figure:parasim:lifecycle}
\begin{center}
\resizebox{\textwidth}{!}{
\begin{tikzpicture}  	[node distance=.5cm,start chain=going below]
\tikzset{>=stealth',
	event/.style={
    	rectangle, 
    	rounded corners, 
    	fill=green!20,
    	draw=black, very thick,
    	text width=9em, 
    	minimum height=3em, 
    	text centered, 
    	on chain},
  	line/.style={draw, thick, <-},
	class/.style={
    	rectangle, 
    	rounded corners, 
    	fill=black!10,
    	draw=black, very thick,
    	text width=8em, 
    	minimum height=3em, 
    	text centered,
		on chain},
	package/.style={
		rectangle,
		draw=black!50, dashed,
		rounded corners,
		inner sep=0.3cm,
		on chain},
  	every join/.style={->, thick,shorten >=1pt}, 	
 	scope/.style={decorate},
	code/.style={
		rectangle,
		draw=black!50, dashed,
		rounded corners,
		text width=15em,
    	minimum height=3em, 
    	text centered,
		node distance=7cm}
}
	\node[event, join] (before-app-context) {BEFORE\\kontext aplikace};
	\node[event, join] (processing) {PROCESSING};
	\node[code, left of=before-app-context] (code-create) {Manager m = Manager.create()};
	\node[package] (processing-package) {
		\begin{tikzpicture}
		\begin{scope}[solid, start branch=venstre, every join/.style={->, thick, shorten <=1pt}]
			\node[class] (enrichment) {obohacování};
			\node[class, on chain=going left] (configuration) {konfigurace};
			\node[class, on chain=going below] (lifecycle) {životní cyklus};
			\node[class, on chain=going right] (extension-loader) {načtení rozšíření};
			\node[class, on chain=going right] (remote) {vzdálená správa};
			\node[class, on chain=going above] (logging) {logování};
		\end{scope}
		\end{tikzpicture}
	};
	\node[event, join] (started) {STARTED};
	\node[code, left of=started] (code-start) {m.start()};
	\node[package, join, inner sep=0.5cm, text width=15em, text centered] (other-extensions) {interakce s načtenými rozšířeními};
	\node[package, join, inner sep=0.5cm, text width=15em, text centered] (main-app) {hlavní kód aplikace};
	\node[event, join] (stopping) {STOPPING};
	\node[event, join] (after-app-context) {AFTER\\kontext aplikace};
	\node[code, left of=stopping] (code-shutdown) {m.destroy()};

	\begin{scope}[->, thick, shorten <=1pt] 
		\draw	(processing)	-> (processing-package);
	\end{scope}

	\begin{scope}[->, dashed, shorten <=1pt] 
		\draw	(code-create)	-> (before-app-context);
		\draw	(code-start)	-> (started);
		\draw	(code-shutdown)	-> (stopping);
	\end{scope}
\end{tikzpicture}}
\end{center}
\caption{Životní cyklus Parasimu. Zelené obdelníky představují události, které se propagují napříč rozšířeními.
Šedými obdelníkymi je znázorněna základní funkcionalita. Čerchovaně ohraničeně jsou vyznačeny části Java kódu.}
\end{figure}

Poté je možné manažera nastartovat, což vyústí ve vytvoření aplikačního kontextu.
S vytvořením každého kontextu je spojena událost \href{https://github.com/sybila/parasim/blob/master/core/src/main/java/org/sybila/parasim/core/event/Before.java}{\texttt{Before}}, která je propagovaná
do všech rozšíření náležejícím tomuto kontextu a v případě jiného než
aplikačního kontextu i do rozšíření kontextu rodičovského.
Po vytvoření aplikačního kontextu následuje událost \href{https://github.com/sybila/parasim/blob/master/core/src/main/java/org/sybila/parasim/core/event/ManagerStarted.java}{\texttt{ManagerStarting}},
na kterou mohou reagovat rozšíření definovaná mimo jádro Parasimu. Jakmile aplikace skončí,
je nutné manažera zničit. Následuje vyvolání poslední u\-dá\-los\-ti, na kterou mohou reagovat načtená rozšíření, \href{https://github.com/sybila/parasim/blob/master/core/src/main/java/org/sybila/parasim/core/event/ManagerStopping.java}{\texttt{ManagerStopping}}
a zničení aplikačního konextu spojené s událostí \href{https://github.com/sybila/parasim/blob/master/core/src/main/java/org/sybila/parasim/core/event/After.java}{After}, případně všech další dosud nezničených kontextů.

\subsection{Kontexty}

Manažer a objekty kontextů umožňují vytvářet nové kontexty skrze rozhraní \href{https://github.com/sybila/parasim/blob/master/core/src/main/java/org/sybila/parasim/core/api/ContextFactory.java}{ContextFactory}.
Aby mohl Parasim od se\-be rozlišit jednotlivé kontexty, používá speciální anotace
rozsahů. Deklarace takové anotace je samotná označená anotace \href{https://github.com/sybila/parasim/blob/master/core/src/main/java/org/sybila/parasim/core/annotation/Scope.java}{\texttt{Scope}},
jak je ukázáno ve zdrojovém kódu \ref{code:scope}.

\begin{lstlisting}[label={code:scope}, caption={Anotace rozsahu}]
@Scope
@Documented
@Retention(RetentionPolicy.RUNTIME)
@Target(ElementType.TYPE)
public @interface Application {}
\end{lstlisting}

Když je k dispozici anotace rozsahu rozlišující kontext, je možné nový kontext vytvořit
nový, jak je ukázáno v ukázce kódu \ref{code:context}. Vývojář je zodpovědný za to,
aby kontext zničil, když už jej nebude potřebovat, aby roz\-ší\-ře\-ním umožnil uvolnit zdroje.

\begin{lstlisting}[label={code:context}, caption={Vytvoření kontextu}]
// vytvori kontext Scope1
// s rodicovskym kontextem Application
Context context1 = manager.context(Scope1.class); 
// vytvori kontext Scope2
// s rodicovskym kontextem Scope1
Context context2 = context1.context(Scope2.class);
...
context2.destroy();
context1.destroy();
\end{lstlisting}

\subsection{Služby}

Základní funkcionalitou jádra Parasimu je poskytovat instance služeb, které jsou
vytvořeny pomocí rozšíření. Služby se definují pomocí rozhraní a může pro ně existovat
více implementací. Aby byl Parasim schopen od sebe rozlišit jednotlivé implementace
různých vlastností používá kvalifikátory. Kva\-li\-fi\-ká\-tor je anotace, v jejíž deklaraci byla použita anotace
\href{https://github.com/sybila/parasim/blob/master/core/src/main/java/org/sybila/parasim/core/annotation/Qualifier.java}{\texttt{Qualifier}}.

Pokud na daném místě aplikace není důležité, jakou typ implementace služby požadovat,
případě není známo, jaké kvalifikátory jsou vůbec k dispozici, je možné použít kvalifikátor
\href{https://github.com/sybila/parasim/blob/master/core/src/main/java/org/sybila/parasim/core/annotation/Default.java}{\texttt{Default}}
pro výchozí implementaci. Je na autorovi rozšíření, aby poskytl smysluplnou výchozí implementaci
poskytované služby.

Příkladem vhodného použití kvalifikátorů je rozšíření poskytující
vý\-počet robustnosti pro danou formuli nad danou trajektorii chování. Zde je možné zavést
různé kvalikátory pro různé temporální logiky. Výchozí implementace takové služby
by měla být schopna spočítat robustnost pro všechny podporované typy temporálních logik,
Avšak je možné, že se v takovéto implementaci bude nacházet méně efektivní algoritmus,
případě nějaká analýza předložené formule, což zbrzdí výpočet.

\begin{lstlisting}[label={code:qualifier}, caption={Kvalikátor}]
@Qualifier
@Target({
	ElementType.FIELD,
	ElementType.METHOD,
	ElementType.PARAMETER})
@Retention(RetentionPolicy.RUNTIME)
@Documented
public @interface Default {}
\end{lstlisting}

Manažer i kontexty implementují rozhraní \href{https://github.com/sybila/parasim/blob/master/core/src/main/java/org/sybila/parasim/core/api/Resolver.java}{\texttt{Resolver}},
které umožňuje na základě rozhraní a kvalifikátoru získat instance dané služby.
Pokud se v daném kontextu nenachází žádné rozšíření poskytující danou službu,
je zavolán rodičovský kontext. 

\begin{lstlisting}[label={code:resolve}, caption={Získání instance služby}]
Manager manager = ...
Enrichment enrichment = manager.resolve(
		Enrichment.class,
		Default.class);
\end{lstlisting}

Parasim používá ještě jeden jednodušší typ služeb. Tyto služby jsou dostupné pouze
pomocí manažera a nelze je od sebe odlišit pomocí kvalifikátoru. Manežer poskytuje
všechny dostupné implementace daného rozhraní v jedné kolekci. Tento typ služeb
primárně  slouží k ovlivňování cho\-vání rozšíření. Lze pomocí nich například
poslouchat událostem z logování.

\subsection{Rozšíření}

\subsection{Konfigurace}

\subsection{Obohacování}

\subsection{Vzdálená přístup}

\section{Výpočetní model}
