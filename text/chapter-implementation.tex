\chapter{Implementace}
Následující kapitola se věnuje aplikaci Parasim \cite{TODO}, která implementuje algoritmus pro
analýzu dynamických systémů zmíněný v sekci \ref{section:algorithm:updated}. Aplikace Parasim
vznikla na základě prototypu dostupného v \cite{drazan2011}. Cílem prototypu je názorně
zobrazit průběh výpočtu původního algoritmu. Uživateli se ukazuje, jakým způsobem
se počítá vzdálenost mezi trajektoriemi chování, kde je nutné zahušťovat a jak dopadlo
ověření platnosti. Vzhledem k tomu, že prototyp například používá pouze jednoduchou metodu
numerické simulace k nalezení trajektorií chování, není vhodné jej použít k analýze složitějších modelů.

V nové implementaci bylo oproti prototypu třeba zahrnout následující:

\begin{enumerate}
	\item	úprava algoritmu dle sekce \ref{section:algorithm:updated},
	\item	paralelní výpočet ve sdílené nebo distribuované paměti,
	\item	rozšiřitelnost, modularita a otevřenost k rozdílné implementaci již naimplementovaných částí,\label{item:features:extensibility}
	\item	zobrazení výsledků analýzy pro vícedimenzionální prostory ini\-ciál\-ních podmínek.\label{item:features:visualisation}
\end{enumerate}

Tato práce se zabývá všemi zmíněnými body kromě bodu \ref{item:features:visualisation},
který je však v Parasimu již také vyřešen. Vzhledem k monolitické implementaci prototypu
nedošlo k jeho úpravám a rozšíření, ale vytvořila se zcela od základu nová aplikace Parasim.
Bod \ref{item:features:extensibility} je důležitý z několika důvodů. Pro různé části algoritmu
existuje více způsobů, jak je naimplementovat. Existuje mnoho nástrojů umožňující
numerickou simulací na základě systému diferenciálních rovnic získat trajektorii chování.
Mnoho věcí zmíněných dříve se vyvíjí. V případě výpočtu robustnosti se nemusíme omezit
pouze na temporální logiku signálů, ale můžeme zavést logiku novou, expresivněj\-ší.

\section{Architektura}

Parasim je aplikace napsaná v Javě 7 složená z většího množství artefaktů pro
sestavovací nástroj Maven. Zdrojové kódy jsou pod licencí GNU GPL verze 3~\cite{gpl} k dispozici
v Git \cite{chacon2009} repozitáři \url{https://github.com/sybila/parasim}. Artefakty sestavené
z poslední verze těchto zdrojových jsou publikovány do Maven repozitáře
\url{http://repository-sybila.forge.cloudbees.com/snapshot/}, o což se stará veřejná
instance nástroje Jenkins \cite{jenkins} \url{http://www.cloudbees.com/}. Artefakty
vydaných verzí jsou k nalezení v Maven repozitáři \url{http://repository-sybila.forge.cloudbees.com/release}.
Hlavní aplikaci Parasimu představuje artefakt \texttt{org.sybila.parasim.application:parasim}.
Pou\-ži\-tí aplikace je popsáno v příloze \ref{appendix:usage}

Jádro Parasimu tvoří artefakt \texttt{org.sybila.parasim:core}, který řídí ži\-vot\-ní
cyklus všech modulů a zajišťuje základní funkcionalitu. To zajišťuje kontejner,
jehož instance je vytvořena a spuštěn v rámci aplikace. Kontejner je velkou mírou inspi\-ro\-ván
konceptem \textit{Context and Dependency Injection} zavedeným v~Java~EE~6~\cite{jendrock2010} specifikací
JSR-299 \cite{jsr299}. V žádném případě se nejedná o~implementaci tohoto standardu,
nýbrž jen o volnou inspiraci některými zá\-klad\-ní\-mi myšlenkami.

Existují zde kontexty, které mají různou délku a je možné je za\-no\-řo\-vat.
Rozšíření pro jednotlivé kontexty nabízejí služby a aplikace definuje na různých
místech v kódu závislosti na těchto službách. Kontejner se stará o to,
aby v případě požadavku na nějakou službu bylo zavoláno příslušné rozříření,
které ji poskytuje, aby vytvořilo objekt služby, a v případě vy\-pr\-še\-ní kontextu,
aby objekt služby vhodným způsobem zničilo.

Za vytvoření a zničení objektu služby je zodpovědný autor rozšíření, který také definuje rozhraní
služby. Aplikace není téměř žádným způsobem závislá na implementaci rozhraní,
a jednotlivé implementace lze tak jednoduše zaměňovat bez toho, aniž bychom
v~aplikačním kódu cokoliv měnili.


\subsection{Životní cyklus}

\begin{figure}[h!]
\begin{center}
\resizebox{\textwidth}{!}{
\begin{tikzpicture}  	[node distance=.5cm,start chain=going below]
\tikzset{>=stealth',
	event/.style={
    	rectangle, 
    	rounded corners, 
    	fill=green!20,
    	draw=black, very thick,
    	text width=9em, 
    	minimum height=3em, 
    	text centered, 
    	on chain},
  	line/.style={draw, thick, <-},
	class/.style={
    	rectangle, 
    	rounded corners, 
    	fill=black!10,
    	draw=black, very thick,
    	text width=8em, 
    	minimum height=3em, 
    	text centered,
		on chain},
	package/.style={
		rectangle,
		draw=black!50, dashed,
		rounded corners,
		inner sep=0.3cm,
		on chain},
  	every join/.style={->, thick,shorten >=1pt}, 	
 	scope/.style={decorate},
	code/.style={
		rectangle,
		draw=black!50, dashed,
		rounded corners,
		text width=15em,
    	minimum height=3em, 
    	text centered,
		node distance=7cm}
}
	\node[event, join] (processing) {PROCESSING};
	\node[code, left of=processing] (code-create) {Manager m = Manager.create()};
	\node[package] (processing-package) {
		\begin{tikzpicture}
		\begin{scope}[solid, start branch=venstre, every join/.style={->, thick, shorten <=1pt}]
			\node[class] (enrichment) {obohacování};
			\node[class, on chain=going left] (configuration) {konfigurace};
			\node[class, on chain=going below] (lifecycle) {životní cyklus};
			\node[class, on chain=going right] (extension-loader) {načtení rozšíření};
			\node[class, on chain=going right] (remote) {vzdálená správa};
			\node[class, on chain=going above] (logging) {logování};
		\end{scope}
		\end{tikzpicture}
	};
	\node[event, join] (before-app-context) {BEFORE\\kontext aplikace};
	\node[code, left of=before-app-context] (code-start) {m.start()};
	\node[event, join] (started) {STARTED};
	\node[package, join, inner sep=0.5cm, text width=15em, text centered] (other-extensions) {interakce s načtenými rozšířeními};
	\node[package, join, inner sep=0.5cm, text width=15em, text centered] (main-app) {hlavní kód aplikace};
	\node[event, join] (stopping) {STOPPING};
	\node[event, join] (after-app-context) {AFTER\\kontext aplikace};
	\node[code, left of=stopping] (code-shutdown) {m.shutdown()};

	\begin{scope}[->, thick, shorten <=1pt] 
		\draw	(processing)	-> (processing-package);
	\end{scope}

	\begin{scope}[->, dashed, shorten <=1pt] 
		\draw	(code-create)	-> (processing);
		\draw	(code-start)	-> (before-app-context);
		\draw	(code-shutdown)	-> (stopping);
	\end{scope}

%	\draw[scope, decoration={brace}] let \p1=(before-app-context.north), \p2=(after-app-context.south) in
%		($(2, \y1)$) -- ($(2, \y2)$) node[scopenode] {application scope};
\end{tikzpicture}}
\end{center}
\end{figure}

\subsection{Konfigurace}

\subsection{Rozšíření}

\subsection{Vzdálená přístup}

\section{Výpočetní model}
