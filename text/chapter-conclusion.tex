\chapter{Závěr}\label{chapter:conclusion}

Cílem diplomové práce bylo implementovat v prostředí s distribuovanou pamětí
algoritmus pro analýzu dynamických systémů zadaných pomocí
soustavy diferenciálních rovnic vzhledem k vlastnostem definovaným
v~tem\-po\-rál\-ní logice signálů. Implementovaný algoritmus byl
převzat z~\cite{drazan2011} a rozšířen o lokální robustnost, jejíž použití umožňuje
efektivnější pokrytí prostoru iniciálních pod\-mí\-nek. 
Implementace takto upraveného algoritmu vyústila ve vytvoření volně dostupného nástroje Parasim\footnote{\url{https://github.com/sybila/parasim/wiki}}.

Na základě identifikovaných parametrů, jež mohou ovlivnit výpočet, byly vybrány modely,
nad kterými se poté spustila analýza v ruzně nastaveném výpočetním prostředí.
Z průběhu této analýzy a jejích výsledků vyplývá, že až na výjimky implementace
ve sdílené i distribuované paměti škáluje. Pro dosažení tohoto výsledku nebylo zapotřebí
žádných sofistikovanějších metod pro balancování výpočtu. Z naměřených
dat také plyne, že použití robustnosti a prezentovaného způsobu pokrytí prostoru iniciálních podmínek má svůj význam,
protože ve většině případů se oproti naivním metodám ušetřilo velké množství práci.

Princip analýzy je ve velké míře podobný tomu, který se používá v~nás\-tro\-ji Breach~\cite{donze2010breach}.
Nicméně narozdíl od něj je Parasim založen na volně dostupných technologiích, je snadno rozšiřitelný
a podporuje distribuované počítání. Také způsob pokrytí prostoru iniciálních podmínek
a použití robustnosti se liší.

Byl naimplmentován vlastní výpočetní model, který abstrahuje od vý\-po\-čet\-ního
prostředí a umožňuje snadno přecházet z prostředí se sdí\-le\-nou pamětí do prostředí z pamětí distribuovanou
a naopak. Díky tomuto a\-spek\-tu a modulární architektuře je Parasim otevřen pro rozšiřování a
implementaci dalších algoritmů.

Ve stávající implementaci je pro numerickou simulaci použit Octave, který se
však ukázal v mnoha případech nevyhovující. Simulace trvá příliš dlouho a často ani
neposkytne požadované výsledky. Použití jiného nástroje pro řešení systému
diferenciálních rovnic nebylo předmětem této práce, nicméně do budoucna se jeví jako vhodné
místo nástroje Octave použít například nástroj COPASI~\cite{hoops2006}, který je přímo určen
pro analýzu biologických modelů.
