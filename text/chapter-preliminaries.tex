\chapter{Pojmy a východiska}

Následující kapitola se věnuje základním pojmům nutným k~pochopení
algoritmu pro analýzu dynamických systémů uvedeného v kapitole \ref{chapter:algorithm}
a~kontextu, v jakém byl navržen. Jedná se zejména o popis reprezentace zkou\-ma\-ných
modelů a požadovaných vlastností.

%\section{Systémová biologie}

\section{Modelování pomocí obyčejných diferenciálních rovnic}

Před popisem samotného modelování pomocí obyčejných diferenciálních rovnic je
nutné říci, co se od vytvářených modelů zpravidla očekává. Model zpro\-středko\-vá\-vá
zjednodušený pohled na zkoumaný systém a umožňuje tím systém snáze pochopit a
případně předpovídat některé jevy. Fakt, že model je jen zjednodušením, znamená,
že se vždy od reality liší a nabízí pouze jen některé aspekty chování zkoumaného
systému. [str. 48]

% Priklad modelu mapa, modelujeme problem ne system

Jednoduchým příkladem modelu je mapa. Zřejmě nemůžeme očekávat od mapy, aby obsahovala
všechny skutečnosti zahrnuté ve skutečném světě. Svým způsobem je mapa již od počátku
\uv{špatně}, nabízí pouze jistou abstrakci systému a může i zkreslovat náš pohled.
Přesto nelze popírat její užitečnost. Na mapě si lze také ukázat fakt, že se
nemodeluje systém, ale problém. Existuje celá řada druhů map od turistických, automap
až po map podloží a každá z nich má svůj specifický účel. \cite[str. 47 -- 58]{pelanek2012}

Pro tuto práci jsou důležité modely, které lze simulovat. Model definuje pravidla, podle
kterých se systém chová a simulace umožňuje se podívat na chování systému v čase, ať už
diskrétním či spojitém. Pro simulace je sa\-mozřej\-mě potřeba znát stav systému, od kterého
se jeho další chování odvíjí.

% Spojity cas -> diferencialni rovnice

Hojně užívaným způsobem modelování, kde vystupuje spojitý čas, jsou diferenciální rovnice.
Stav systému se vyjádří pomocí stavových pro\-měn\-ných $\mathbf{X} = (\mathit{x_1}, \mathit{x_2}, \dots \mathit{x_n})$.
Každé stavové proměnné přísluší diferenciální rovni\-ce prvního řádu, ve které vystupue
Lipschitzovsky spojitá \cite[str. 149 -- 163]{eriksson2004} funkce
$f_i: [t, \infty) \times \mathbb{R}^n \rightarrow \mathbb{R}$, která popisuje,
jak se stavová proměnná mění v čase. Tvar takové rovnice je vidět ve sché\-ma\-tu~\ref{eq:ode}.

\begin{align}\label{eq:ode}
\frac{d\mathit{x_i}}{dt} = f_i(t, \mathbf{X})
\end{align}

Pro účely simulce není nutné znát úplné řešení této soustavy rovnic, ale postačuje pouze znalost
vývoje systému od počatečního času $t_0$, kterému odpovídá počáteční stav $\mathbf{X}(t_0)$.
V praxi se setkáváme s tím, že neznáme ani tento přesný vývoj, nýbrž pouze jeho aproximaci,
kterou poskytují metody pro řešení problému výchozích podmínek~\cite{iserles1996}.

\begin{figure}[h!]
\begin{center}
\subfigure[vývoj systému v čase]{
	\includegraphics[width=.48\textwidth]{../images/lotkav-timeserie.png}
}
\subfigure[stavy, kterými systém prochází]{
	\includegraphics[width=.48\textwidth]{../images/lotkav-oscil.png}
}
\end{center}
\caption{Model systému obsahující predátora a kořist.}
\end{figure}

Tyto metody hledají aproximaci v diskrétním čase a chyba, s níž se vy\-po\-čí\-ta\-ná
aproximace liší od skutečného řešení, je shora ohraničená uživatelem danou hodnotou.
Fakt, že si může uživatel takto nadefinovat toleranci chyby, je je jednou z nejdůležitějších
vlastností těchto metod. Nastavení chyby může samozřejmě v případě nízké tolerance
a některých systémů vyústit ve výkonostní problémy. Jestliže tedy zvolíme časový
krok $h \in \mathbb{R}$, pak výstupem těchto metod je sekvence bodů $\mathbf{X_0}, \mathbf{X_1}$, \ldots,
pro kterou platí vztah \ref{eq:ode:aprox}.

\begin{align}\label{eq:ode:aprox}
\mathbf{X}_i \sim \mathbf{X}(t_i) \textrm{ kde } t_i = t_0 + i \cdot h, i \in \mathbb{N}
\end{align}

Známým modelem využívající soustavy diferenciálních rovnic popisuje vztah predátora
a kořisti \cite{lotka1925}.

\begin{align}\label{eq:lotkav}
\begin{array}{ll}
\frac{dx}{dt} &= x\cdot(\alpha - \beta \cdot y)			\\
\frac{dy}{dt} &= -y \cdot (\gamma - \delta \cdot x)
\end{array}
\end{align}

% Priklady modelu ve spojitem case

% Moznost prevodu chemickych rovnic do diferencialnich rovnic

% Celkem by to melo byt na 3 - 4 stranky


\section{Temporální logika signálů}

% Celkem 3 - 4 stranky
